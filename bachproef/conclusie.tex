%%=============================================================================
%% Conclusie
%%=============================================================================

\chapter{Conclusie}%
\label{ch:conclusie}

% TODO: Trek een duidelijke conclusie, in de vorm van een antwoord op de
% onderzoeksvra(a)g(en). Wat was jouw bijdrage aan het onderzoeksdomein en
% hoe biedt dit meerwaarde aan het vakgebied/doelgroep? 
% Reflecteer kritisch over het resultaat. In Engelse teksten wordt deze sectie
% ``Discussion'' genoemd. Had je deze uitkomst verwacht? Zijn er zaken die nog
% niet duidelijk zijn?
% Heeft het onderzoek geleid tot nieuwe vragen die uitnodigen tot verder 
%onderzoek?
Als men terug blikt op de vraag die gesteld werd in de inleiding: welk platform het meest geschikt is voor Quivvy Solutions, had men verwacht dat het Softr zou zijn. Maar men kan concluderen dat Bubble beter zou passen bij het bedrijf. 
Deze conclusie heeft men kunnen trekken door de zeer uitgebreide methodologie; alternatieven, vergelijkende analyse, en Proof of Concept. De reden dat Bubble aangeraden wordt, is omdat het voor Quivvy Solutions noodzakelijk is dat het platform
kan integreren met zowel AirTable als MAKE.com. Bubble heeft deze mogelijkheden en heeft ook een groot aantal plugins. Daarnaast is Bubble ook superieur bij het gebruik van diverse databases en platformflexibiliteit.
Vervolgens is Bubble stabieler op lange termijn en interessanter voor grote bedrijven. De prijs van Bubble is tegenover de twee andere platformen minder duur. Het platform is ook veel transparanter naar hun 
gebruikers toe op het vlak van updatebeleid. De enigste minpunten van Bubble is dat Softr en Stacker beter scoorde op snelheid van applicatieontwikkeling en gebruiksvriendelijkheid. Helaas weten we niet hoe goed elke platform is bij het maken
van complexe apps. Dit omdat we enkel een Proof of Concept hebben gemaakt waarin simpele applicaties werden gebouwd. We kunnen concluderen dat volgens de programmeur van de Proof of Concept Softr en Stacker minder geschikt zou zijn voor complexe apps. Hieruit volgt dat men verder kan onderzoeken 
hoe deze platformen presteren bij het maken van complexe apps. Ten slotte is het belangrijk om te vermelden dat de uitgevoerde Proof of Concept niet voldoende is om een definitieve beslissing te nemen.


