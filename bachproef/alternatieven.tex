\chapter{\IfLanguageName{dutch}{Evaluatie en Selectie van Alternatieven}{Evaluation and Selection of Alternatives}}%
\label{ch:evaluatie-en-selectie-van-alternatieven}

\section{Inleiding}
\label{sec:inleiding}
In dit hoofdstuk selecteert en analyseert men de alternatieve low-code en no-code platformen die in aanmerking komen voor Quivvy Solutions.
Vooraleer we een vergelijkende analyse en een proof of concept uitvoeren is het essentieel om deze geselecteerde platformen te analyseren.
De selectie van de alternatieven is gebaseerd op een aantal belangrijke vereisten die door de copromotor is opgesteld. Deze vereisten omvatten de integratiemogelijkheden en hun platformflexibiliteit.\\\\
Daarnaast is het belangrijk om de populariteit van de platformen te bekijken, aangezien dit vaak een indicatie is van de kwaliteit van het platform. Elk geselecteerde platform zal vervolgens uitgebreid geanalyseerd worden op basis van 
hun functies, voordelen, nadelen, integratiemogelijkheden, beoordelingen, en kostprijs.\\\\
Deze gedetailleerde analyse stelt het in staat om een goede keuze te maken voor een alternatief platform dat het meest geschikt is voor de behoeften en eisen van Quivvy Solutions.
\section{Platformen}%
\label{sec:platformen}
\subsection{Zoho Creator}%
\label{subsec:zoho-creator}

Zoho Creator is een Low-Code platform dat werknemers van bedrijven maar ook mensen zonder programmeerervaring toestaan
om eenvoudige krachtige bedrijfsapplicaties te ontwikkelen \autocite{Computer2022}. Dit platform is gemaakt door Zoho Corporation.
Zoho Corporation is een bedrijf dat zich focust op het verder ontwikkelen van hun product, zoals Zoho Creator, en hun klantensupport \autocite{ZohoCorporation2024a}. Dit toont aan dat Zoho Corporation
het updatebeleid volgens hun website goed is. Volgens \textcite{ZohoCorporation2024a} biedt het bedrijf meer dan alleen een product aan, het bedrijf noemt het 'the operating system for business' \autocite{ZohoCorporation2024a}.
Dit bevat maar liefst 55 integreerde applicaties voor elke bedrijfsnood.

\paragraph{Functies}
In 2022 was er nog geen Low-Code platform dat toeliet om businessgebruikers en IT een end to end business oplossing te laten maken \autocite{Computer2022}.
De Zoho Creator platform zorgt ervoor dat je applicatie development, business intelligence en analytics, integraties, en automatiseringen kan doen in één platform, wat men ook een end to end business oplossing noemt \autocite{Computer2022}.
Dit geeft verschillende voordelen voor het bedrijf dat het platform gebruikt. Zoals het verhogen van veiligheid, maar ook omdat het platform een end to end business oplossing is. Hierdoor is het mogelijk om uniforme oplossingen te maken via de low-code platform,
waardoor elke werknemer de mogelijkheid krijgt om te innoveren. Volgens \textcite{Computer2022} zorgt Zoho Creator er ook voor dat de business gebruikers snel een schaalbare low-code oplossingen kunnen ontwikkelen zoals het maken van een app of automatisering.
Het ontwikkelen van een low-code oplossing is zelfs 10 keer sneller dan eender andere oplossing op de markt \autocite{Computer2022}. Het platform heeft ook hun eigen AI, genaamd Zia, die de gebruiker helpt bij het maken van applicaties. Deze AI
kan ontwikkelaars helpen bij het importeren van data door data te analyseren en te structureren. Daarnaast kan het ook relaties tussen data detecteren. 
Zoho Creator geeft u de mogelijkheid om zelf code te schrijven, wat een groot voordeel is voor bedrijven die toch nog high-code willen implementeren \autocite{Computer2022}. 
De code kan geschreven worden in Deluge, Java of Node.js en zijn ook herbruikbaar om duplicatie te voorkomen.
De gebruikers van het platform Zoho Creator kunnen ook zien hoe goed integraties aan het werken zijn met de 'Integration Status Dashboard', wat helpt om te analyseren waar er een probleem kan zijn \autocite{Computer2022}.
\\ %TODO (nu enkel info van Express Computer, moet nog andere bronnen zoeken)
\\
Volgens het bedrijf \textcite{ZohoCorporation2024b} neemt het platform 90\% van de complexiteit weg bij het ontwikkelen van applicaties. Het is zo dat de gebruikers van het platform
zich kunnen focussen op de functies, businesswaarde en de eindklant \autocite{ZohoCorporation2024b}.  Met Zoho Creator kan je apps maken zoals volledig jouw eigen app, mobile apps, en een online portal.
Maar Zoho Creator kan je voor meer dan app development gebruiken, je kan businessprocessen automatiseren, BI \& Analytics, en integraties maken met andere systemen. Daarnaast zorgt het ook voor een veilige omgeving en hoef je u geen zorgen 
te maken over middleware, authenticatie, en business transacties. Want Zoho Creator heeft hier namelijk services voor. Vervolgens heeft Zoho Creator functies zoals het opslaan van data,
betalingstransacties, het updaten van je CRM, en e-mails en rapporten sturen \autocite{ZohoCorporation2024b}.

\paragraph*{Voordelen}
\begin{enumerate}
    \item Het heeft een zeer simpele en straightforward interface \autocite{Marvin2017Zoho}.
    \item Zoho Creator is betaalbaar  \autocite{Marvin2017Zoho}.
    \item Het heeft een grote selectie aan pre-built app templates en componenten \autocite{Marvin2017Zoho}.
    \item Je kan je eigen workflows creëren \autocite{Marvin2017Zoho}.
    \item Het heeft ingebouwde automatische vertaling \autocite{Marvin2017Zoho}.
    \item Het ondersteund scannen van streepjescodes \autocite{Marvin2017Zoho}.
\end{enumerate}


\paragraph*{Nadelen}
\begin{enumerate}
    \item Het heeft een gelimiteerd aantal features \autocite{G22024}.
    \item Het heeft database limitaties \autocite{G22024}.
    \item Zoho Creator heeft een zwakke Customer Support \autocite{G22024}.
\end{enumerate}

\paragraph{Integratie mogelijkheden}
Zoho Creator heeft een lijst met maar liefst 600+ apps waarmee het kan integreren \autocite{ZohoCorporation2024b}. Ze hebben integraties voor Sales automation, IT Security en Team Collaboration.
Helaas heeft Zoho Creator nog geen integratie met AirTable, maar wel met \textcite{MAKE.com2024}.

\paragraph{Beoordelingen}
Op \textcite{Gartner2024} heeft Zoho Creator een beoordeling van 4.6/5.0, gebaseerd op 205 beoordelingen. Hieruit volgt dat 38\% van bedrijven met minder dan een waarde van 50 miljoen dollar Zoho Creator aanbevelen, 
en 37\% van bedrijven met meer dan 50 miljoen dollar tot 1 miljard Zoho Creator aanbevelen. Vervolgens waren de laagste beoordelingen met 4.3/5, op business logica en workflow, integratie met API's, en platformflexibiliteit \autocite{Gartner2024}.
\paragraph{Kostprijs}
%per abonnement; herstelbeheer, integratie, automatiseringen (ai)
Zoho Corporation heeft diverse abonnementen voor hun platform, deze zijn Standard, Professional, Enterprise, en Flex \autocite{ZohoCorporation2024}.
De standaard abonnement bedraagt 12 euro, per gebruiker, per maand. Hiermee kan je maar maximaal 1 business applicatie maken.
Daarbij wordt ook je data opgeslagen en geback-upt in de cloud en kan u 20 AI models maken om te trainen voor specifieke taken \autocite{ZohoCorporation2024}. Op vlak van integratie
kan je maar maximum 5 databronnen integreren en 5 eigen connecties maken met systemen die niet ondersteund worden door Zoho Creator \autocite{ZohoCorporation2024}. Vervolgens heb je dan het
abonnement Professional, wat 30 euro per gebruiker, per maand bedraagt. Hiermee kan je unlimited business applicaties maken en 100 AI models trainen \autocite{ZohoCorporation2024}. 
Daarbij kan je 15 databronnen integreren en 10 eigen connecties maken \autocite{ZohoCorporation2024}. Daarnaast heb je dan Enterprise abonnement, wat 37 euro per gebruiker per maand bedraagt.
Het voordeel van dit abonnement is dat je integratie mogelijkheden hebt met 650+ business apps, 30 databronnen kan integreren, en 20 eigen connecties kan maken \autocite{ZohoCorporation2024}. Vervolgens heb je ook
Business Intelligence and Analytics bij Enterprise om eenvoudig analyses te maken \autocite{ZohoCorporation2024}. Het handige is dat je ook de mogelijkheid hebt om hun te contacteren in geval 
van gepersonaliseerde requirements via het Flex abonnement \autocite{ZohoCorporation2024}.

\subsection{OutSystems}%
\label{subsec:outsystems}
OutSystems is een Low-Code platform dat developers 'pixel-perfect' applicaties laat maken met hun drag-and-drop functionaliteiten \autocite{Ranosys2023} \autocite{Payne2023}.
OutSystems maakt gebruik van AI, cloud technologie, DevOps, en visuele ontwikkeling om citizen developers deel te laten nemen aan applicatieontwikkeling \autocite{Ranosys2023}.

\paragraph{Functies}
Deze Low-code platform heeft een visuele omgeving dat het ontwikkelingsproces versnelt en de complexiteit van het ontwikkelen van applicaties verminderd \autocite{Payne2023}.
Hierdoor gaat de productiviteit naar omhoog, verminderd de leercurve, en verminderd het aantal errors and bugs \autocite{Payne2023}. OutSystems heeft ook ingebouwde templates 
en componenten die het ontwikkelen van applicaties versnelt en geïmplementeerde high-code vermindert.
Vervolgens laat het toe om eigen gebouwde componenten te gebruiken voor consistentie en schaalbaarheid \autocite{Ranosys2023}.
OutSystems heeft ook Agile Software Development functies zoals version control, programmer collaboration en geautomatiseerde testing tools \autocite{Ranosys2023}.
Het Low-Code platform heeft ook een AI om bijvoorbeeld errors te verminderen, om pre-built componenten te genereren of zoeken en het analyseren van code om bugs te detecteren \autocite{Ranosys2023}.
Het bedrijf OutSystems laat ook bedrijven toe om de prestaties en groei van uw bedrijf visueel te monitoren en te beheren \autocite{Ranosys2023}.

\paragraph*{Voordelen}
\begin{enumerate}
    \item Het reduceert de kost van software development, omdat men zeer snel applicaties kan maken \autocite{Payne2023}.
    \item Het platform ondersteunt Agile Software Development \autocite{Payne2023}.
    \item Zeer goed platform voor samenwerking tussen developers \autocite{Payne2023}.
    \item Het platform is makkelijk te gebruiken \autocite{G22024OutSystems}.
\end{enumerate}


\paragraph*{Nadelen}
\begin{enumerate}
    \item OutSystems is een duur platform \autocite{G22024OutSystems}.
    \item Het heeft een gelimiteerd aantal features \autocite{G22024OutSystems}.
    \item Het platform bedraagt volgens \textcite{G22024OutSystems} een hoge leercurve.
\end{enumerate}

\paragraph{Integratie mogelijkheden}
OutSystems bevat integratie mogelijkheden zoals het connecteren met externe databronnen, enterprise systemen, en API's \autocite{Payne2023}.
Hiervoor biedt het Low-Code platform connecties en tools aan om het integratieproces te vergemakkelijken en tijd te besparen \autocite{Payne2023}. Helaas
is er geen informatie gevonden over integratie met AirTable en MAKE.com.

\paragraph{Beoordelingen}
Volgens \textcite{Gartner2024} heeft OutSystems een beoordeling van 4.5/5.0, gebaseerd op 885 beoordelingen. 
Hieruit volgt dat 22\% van bedrijven met minder dan een waarde van 50 miljoen dollar OutSystems aanbevelen en dat 44\% van bedrijven met meer dan 50 miljoen dollar tot 1 miljard OutSystems aanbevelen.
Vervolgens zijn de beoordeling op schaalbaarheid (4.3/5), aanpasbaarheid (4.3/5), en een goede platform voor overheid (4.2/5) de laagste beoordelingen \autocite{Gartner2024}.

\paragraph{Kostprijs}
OutSystems heeft drie abonnementen, namelijk Free, Multiple apps, en Large app portfolio's \autocite{OutSystems}. Het Free abonnement is gratis en is bedoeld voor het maken van één app.
Bij Free plan kan je enkel de app uitvoeren in development en niet in production \autocite{OutSystems}. Vervolgens kunnen er ook maar 100 eindgebruikers gebruikmaken van de app \autocite{OutSystems}.
Het Multiple apps abonnement heeft meer voordelen, maar wel tegen een prijs van 1.250 euro per maand. Deze voordelen zijn dat het zowel in development als in production modus kan uitgevoerd worden \autocite{OutSystems}.
Daarnaast heb je een uptime van 99.5\% en is er geen maximum aantal eindgebruikers. Het laatste abonnement, Large app portfolio, is bedoeld voor bedrijven met meer nodige functies dan gegeven bij Multiple apps \autocite{OutSystems}. 
\subsection{Microsoft PowerApps}%
\label{subsec:microsoft-powerapps}
Microsoft PowerApps is gezien als één van de bekendste Low-Code platformen \autocite{Nguyen} \autocite{Gupta2023}.
Microsoft PowerApps is meer dan alleen een Low-Code platform, het bestaat uit verschillende app features, templates voor ontwerp, pre-built connectors, en tools om ontwikkelaars te helpen bij het maken van een Low-Code applicatie \autocite{Nguyen}.

\paragraph{Functies}
Microsoft PowerApps heeft een robuuste veiligheid om zowel de applicatie als de data op elk niveau te beschermen \autocite{Nguyen}.
Gebruikers van de gemaakt applicatie worden beveiligd door Office 365 of Azure. Daarnaast is er ook access control, app-level security, form-level security, record-level security, en field-level security  \autocite{Nguyen}.
Het Low-Code platform support ook een wijd bereik aan apparaten van zowel Android, iOS, en Windows voor mobiele apps \autocite{Nguyen}.
Voor de web versie van de app kan het op eender moderne web browser draaien. Dat terzijde beschikt volgens \textcite{Gupta2023} Microsoft PowerApps over unieke functies. Als eerste heb je 
streamline operations, dit is voor het traceren van de werknemerskosten tot automatiseren van communicaties, analyse van data en het introduceren van AI in de bedrijfsprocessen \autocite{Gupta2023}.
Als tweede heb je snellere ontwikkeling want met Microsoft PowerApps kan je een applicatie maken in dagen \autocite{Gupta2023}. Vervolgens is het integreren met Office 365 suite eenvoudig met Microsoft PowerApps,
 dus kortom alles van Microsoft kan geïntegreerd worden.

 \paragraph*{Voordelen}
\begin{enumerate}
    \item Een grote reeks aan functies \autocite{Marvin2018}.
    \item Het platform biedt heel veel integratie mogelijkheden aan \autocite{Marvin2018}.
    \item Het is een zeer makkelijk platform om te gebruiken \autocite{Marvin2018}.
\end{enumerate}


\paragraph*{Nadelen}
\begin{enumerate}
    \item Het platform kan soms traag zijn \autocite{Marvin2018}.
    \item De gebruikersinterface van PowerApps kan in het begin overweldigend zijn \autocite{Marvin2018}.
    \item Het platform heeft een redelijk steile leercurve \autocite{Marvin2018}.
\end{enumerate}

\paragraph{Integratie mogelijkheden}
Volgens \textcite{Nguyen} kan PowerApps integreren met over de 400 connectors (applicaties/services). De meest bekende 
zijn Office 365, SQL Server en Azure,  en OpenAI \autocite{Nguyen}. Volgens het bedrijf \textcite{Microsoft2024} kan PowerApps integreren met 
AirTable, maar helaas is er geen integratie mogelijkheid met \textcite{MAKE.com2024a}.
\paragraph*{Limitaties en nadelen}
Microsoft PowerApps heeft 3 belangrijke limitaties waar bedrijven rekening mee moeten houden. 
Volgens \textcite{Gupta2023} is de eerste limitatie dat het platform gelimiteerd is op vlak van platformflexibiliteit.
Daarbij is het ook lastig om het low-code platform te integreren met externe systemen want het heeft maar een paar 
third-party applicaties of services waarmee het kan integreren \autocite{Gupta2023}. Ten slotte kunnen de applicaties die gemaakt zijn met
PowerApps niet gepubliceerd worden op de Apple App Store, Google Play Store en Windows Store \autocite{Gupta2023}.
Volgens \textcite{Nguyen} is het moeilijk om complexe applicaties te maken met PowerApps.
De reden hiervoor is omdat je dan high-code moet implementeren, wat betekent dat je de programmeertaal Power FX zal moeten leren.
Daarnaast zijn er volgens \textcite{Nguyen} reviews die zeggen dat het pricing plan alleen voordelig is voor kleinere specifieke apps.
\paragraph{Beoordelingen}
Microsoft PowerApps heeft op \textcite{Gartner2024} een beoordeling van 4.5/5.0, gebaseerd op 295 beoordelingen.
Uit deze beoordeling zijn er 12\% van bedrijven met minder dan een waarde van 50 miljoen dollar die Microsoft PowerApps gebruiken en aanbevelen \autocite{Gartner2024}.
Dit is anders voor bedrijven met een waarden tussen 50 miljoen en 1 miljard want daar is het 41\%.
Microsoft PowerApps scoort het slechtste voor de overheid (4.0/5) en DevOps Practices (4.2/5).

\paragraph{Kostprijs}
Volgens \textcite{Gupta2023} heeft het Low-Code platform 3 abonnementen, namelijk de eerste is ongeveer 4,60 euro per maand per gebruiker om één applicatie te runnen.
Als tweede heb je een abonnement Premium van 18,70 euro per maand per gebruiker, om meerdere applicaties te runnen. Als laatste heb je een pay-as-you-go abonnement
waarbij het ongeveer 9,20 euro per actieve gebruiker per app per maand is, maar hiervoor heb je een Azure abonnement nodig \autocite{Gupta2023}.
\subsection{Appian}%
\label{subsec:appian}
Volgens \textcite{Shala} is Appian Corporation een cloud-based business software bedrijf. Het bedrijf biedt een Platform as a Service (PaaS) aan voor het maken van bedrijfsapplicaties.
Appian Corporation focust op Low-Code development, Business Process, en Case Management Markets. Het Low-Code platform Appian wordt gebruikt om bedrijven te helpen
bij het maken van applicaties, het automatiseren van bedrijfsprocessen en case management \autocite{Shala}.
\paragraph{Functies}
De hoofdfunctie van Appian is om snel een enterprise-ready applicatie te maken met een schitterende user interface \autocite{Shala}. Appian maakt het ook makkelijk 
om je middelen en mensen, technologie, data, en systemen te combineren in een uniform proces om de productiviteit te verhogen \autocite{Shala}.
Het Low-Code platform Appian doet een goede onderscheiding tussen de niet-programmeerbare delen en de zware processen \autocite{Marvin2017}.
Volgens \textcite{Marvin2017} is de interface en de mogelijkheden van Appian Quick App, dat een onderdeel is van het Low-Code platform, niet zo goed in vergelijking met dat van Zoho Creator of Microsoft PowerApp. Maar Appian is wel de enige die een echte no-code ervaring biedt.

\paragraph*{Voordelen}
\begin{enumerate}
    \item Met Appain Quick Apps heb je een echte no-code ervaring \autocite{Marvin2017}.
    \item Het kan gebruikt worden om native mobiele applicaties te maken \autocite{Marvin2017}.
    \item Bevat een drag-and-drop procesontwerper \autocite{Marvin2017}.
    \item Ingebouwde team samenwerking, taakbeheer systeem, en bedrijfsnetwerkplatform \autocite{Marvin2017}.
\end{enumerate}


\paragraph*{Nadelen}
\begin{enumerate}
    \item Tegenover andere Low-Code platformen is het zeer duur \autocite{Marvin2017}.
    \item Gebruikersinterface is niet meer up-to-date \autocite{Marvin2017}.
\end{enumerate}

\paragraph{Integratie mogelijkheden}
Volgens \textcite{Marvin2017} heb je ervaring nodig in het programmeren om Appian te integreren met andere systemen. 
Helaas kan het ook niet integreren met \textcite{MAKE.com2024a}.

\paragraph{Beoordelingen}
Appian scoort op \textcite{Gartner2024} een beoordeling van 4.5/5.0, gebaseerd op 525 beoordelingen. Hieruit volgt dat 85\% van de reviews het aanbevelen \autocite{Gartner2024}.
Enkel 12\% van de reviews zijn kleine bedrijven dat het aanbevelen \autocite{Gartner2024}. Appian wordt volgens de recommandaties gebruikt in grote bedrijven met een waarde van meer dan 10 miljard dollar, dit is namelijk 28\% \autocite{Gartner2024}.
\paragraph{Kostprijs}
Helaas is er zeer weinig info gevonden over de kostprijs van Appian. Ze hebben drie abonnementen, namelijk de eerste is 
Standard, de tweede is Advandced en de laatste is Premium \autocite{Appian2024}. De prijs van deze abonnementen is niet bekend omdat dit kan verschillen van bedrijf to bedrijf want het is namelijk 
per gebruiker per maand per app. Het grootste verschil tussen de abonnementen is dat je bij Premium niet gelimiteerd meer bent op het aantal portals en bots \autocite{Appian2024}. Maar volgens
\textcite{Shala} kost het Standard abonnement ongeveer 7 euro per gebruiker per maand.


% \subsection{Wix}%
% \label{subsec:wix}
% Volgens \textcite{Ryan2024} is Wix de beste website bouwer voor 2024, dat beter was dan 16 andere platformen dat ze hebben getest. Vervolgens raden ze 
% Wix aan voor simpele websites en kleine bedrijven om hun merk online op te bouwen, maar niet voor grote online winkels door het gebrek aan schaalbaarheid \autocite{Ryan2024}.
% Wix geeft de gebruikers een goede verhouding tussen prijs en kwaliteit door de functies die het platform aanbiedt tegenover een eerlijke prijs  \autocite{Singleton2024}.
% Bij een abonnement van Wix krijg je een website editor, verkoop tools, video hosting, een manier om agenda's te implementeren in de app, en nog meer.
% \paragraph{Functies}
% Zoals vermeld heeft Wix heel wat functies. Eén van zijn beste functies is zoekmachineoptimalisatie (SEO) tools, dit omvat het personaliseren van de URL en meta data \autocite{Ryan2024}.
% Op Wix zijn er ook integraties die kunnen helpen met zoekmachineoptimalisatie zoals Google My Business \autocite{Ryan2024}. Vervolgens heb je een andere functie genaamd email marketing
% om de website eigenaar een connectie te laten leggen met zijn klanten. Deze functie bevat ook templates die je volledig kan personaliseren naar uw eigen stijl met de drag-and-drop editor.
% Met Wix kan je ook je website synchroniseren met je sociale media accounts zoals Facebook, LinkedIn, en X. Hiermee kan je dan via uw Wix Dashboard nieuwe social media posts maken en publiceren.
% Meestal moet een website meertalig zijn, met Wix kan je dit eenvoudig realiseren want het ondersteund 180 talen en vertaalt automatisch uw website \autocite{Ryan2024}. Het Low-Code platform biedt ook een afspraakmaker aan, waardoor je geen 
% gebruik moet maken van een externe afspraakmaker zoals Google Calendar \autocite{Singleton2024}. 
% \paragraph*{Voordelen}
% \begin{enumerate}
%     \item Wix bevat een drag-and-drop editor \autocite{Ryan2024}.
%     \item Maar liefst 900+ professionele ontworpen templates voor uw website \autocite{Ryan2024} \autocite{Singleton2024}.
%     \item Beschikt over AI tools om u te helpen bij het maken van uw website \autocite{Ryan2024}.
%     \item Beschikt over een ingebouwde email marketing tool \autocite{Singleton2024}.
% \end{enumerate}


% \paragraph*{Nadelen}
% \begin{enumerate}
%     \item Ongelimiteerde opslag is enkel verkrijgbaar bij het duurste abonnement \autocite{Ryan2024}.
%     \item De verschillende abonnementen van Wix zijn wat duurder dan andere website bouwers \autocite{Ryan2024}.
%     \item Door de vele functies kan het platform overweldigend zijn voor beginners \autocite{Ryan2024}.
%     \item Templates zijn niet volledig responsive \autocite{Singleton2024}.
% \end{enumerate}

% \paragraph{Integratie mogelijkheden}
% Volgens \textcite{Singleton2024} bevat Wix een app store genaamd Wix App Market. 
% Hierin kan je maar liefst 821 apps vinden van het bedrijf zelf maar ook van andere bedrijven, die je kan integreren met uw website \autocite{Singleton2024}.
% Het is ten slotte mogelijk om Wix te integreren met  \textcite{MAKE.com2024a}.

% \paragraph{Kostprijs}
% De prijzen variëren van €11 tot €149 per maand (jaarlijks gefactureerd) \autocite{Wix2024} \autocite{Ryan2024}. 
% Er bestaan 4 abonnementen, namelijk Light, Core, Business, en Business Elite \autocite{Wix2024}.
% Het Light abonnement is het goedkoopste abonnement en kost €11 per maand. Light wordt gebruikt voor informele websites \autocite{Ryan2024}.
% Het Core abonnement kost €22 per maand \autocite{Wix2024}. Dit abonnement is voor als jouw website betalingen moet accepteren en online verkoopt \autocite{Ryan2024}.
% Het Business abonnement kost €34 per maand \autocite{Wix2024}. Dit abonnement is voor groeiende kleine bedrijven \autocite{Ryan2024}.
% Het Business Elite abonnement kost €149 per maand \autocite{Wix2024}. Dit abonnement is voor succesvolle online stores \autocite{Ryan2024}.
\subsection{Bubble}%
\label{subsec:bubble}
Bubble is een No-Code development platform dat u toelaat om web applicaties te ontwerpen, te maken, en te hosten zonder enige lijntje code te schrijven \autocite{Sharma2022}.
Je hebt rechtstreeks toegang tot je browser via Bubble.io, zoals de meeste LCNC platformen \autocite{Minor2022}. Het maken van een applicatie in Bubble kan zeer ruim zijn, van een simpele blog tot een CRM systeem \autocite{Sharma2022}.
Het platform kan gebruikt worden als interactieve website bouwer met SEO tools en analytics, maar zou ook kunnen gebruikt worden om een 2D game te maken zoals Wordle \autocite{Minor2022}.

\paragraph{Functies}
Dit No-Code platform maakt het mogelijk om elk soort web applicatie te maken zonder code te schrijven \autocite{Bubble2024b}.
Dit kan gaan tot een interactieve en multi-user apps voor desktop en mobile web browsers, daarnaast bevat het alles om een site zoals Facebook of Airbnb te maken \autocite{Bubble2024b}.
Bubble maakt het makkelijk om je applicatie te ontwerpen en te maken met hun drag-and-drop editor.
Vervolgens zorgt het platform zelf voor het veilig opstellen en hosten van de applicatie waarbij er geen limiet is op het aantal gebruikers en volume van verkeer of data opslag \autocite{Bubble2024b}.
Daarnaast kan je ook in Bubble samenwerken in real time, wat betekent dat je de wijzigingen van je collega's direct kan zien.
\paragraph*{Voordelen}
\begin{enumerate}
    \item Het is een visuele programming language \autocite{Minor2022}.
    \item Geen enkele ervaring van programmeren nodig \autocite{Minor2022}.
    \item Het heeft tutorials over verschillende onderwerpen binnen het platform \autocite{Minor2022}.
    \item Bevat een community marketplace voor templates en plugins \autocite{Minor2022}.
\end{enumerate}


\paragraph*{Nadelen}
\begin{enumerate}
    \item Het kan snel duur worden op langere termijn \autocite{Minor2022}.
    \item Door het gebrek aan coderen kan het platform beperkt zijn \autocite{Minor2022}.
    \item Met Bubble.io kan je geen native mobiele applicaties maken, dus rechtstreeks publiceren op bijvoorbeeld Google Play Store is niet mogelijk
    maar hiervoor zijn workarounds zoals het gebruikmaken van een third-party service \autocite{Sharma2022}.
    \item Je kan geen python of andere scripts uitvoeren \autocite{Sharma2022}.
\end{enumerate}

\paragraph{Integratie mogelijkheden}
Er is geen exacte nummer voor hoeveel integraties er mogelijk zijn met Bubble.io maar op de officiële website \textcite{Bubble2024a} 
staat er dat het mogelijk is om zowel met MAKE.com en AirTable te integreren. Daarnaast kan het ook integreren met Google Maps, Figma, Zapier, Discord, Google Drive en nog veel meer.

\paragraph{Beoordelingen}
Op \textcite{Gartner2024} heeft Bubble.io een beoordeling van 4.4/5.0, gebaseerd op 33 beoordelingen. Volgens \textcite{Gartner2024}
wordt het platform vooral gebruikt in bedrijven met een waarde tussen 50 miljoen en 1 miljard dollar en minder dan 50 miljoen.
Bubble scoort het slechtste op een goed platform voor overheid (4.0/5), DevOps Practices (4.1/5), en security \& QoS  (4.1/5). Maar het scoort wel
het beste op User Experience Design (4.8/5).

\paragraph{Kostprijs}
Bubble.io heeft 5 abonnementen, namelijk Free, Starter, Growth, Team, en Enterprise \autocite{Bubble2024}.
Het Free abonnement is gratis en bedoeld voor het maken van projecten die nog in aanbouw zijn \autocite{Bubble2024}.
Het Starter abonnement kost ongeveer 27 euro per maand (\$ 29) en is bedoeld voor Minimum Viable Products (MVPs) en simpele tools.
Het Growth abonnement kost ongeveer 111 euro per maand (\$ 119). Het is goed voor consumer projects met complexe functionaliteiten.
Het Team abonnement kost ongeveer 327 euro per maand (\$ 349) en is goed voor schaalbare projecten met hoge gebruikersfrequentie. Als laatste heb je dan nog
het Enterprise abonnement, waarvan de prijs niet bekend is, bedoeld voor bedrijven met specifieke requirements.

\section{Samenvatting}%
\label{sec:samenvatting}
\begin{longtable}{p{2.5cm} p{5.5cm} p{3.5cm} p{2.5cm}}
    \caption{Samenvatting van alternatieven} \label{samenvatting-alternatieven} \\
    \toprule
    \textbf{Platform} & \textbf{Voordelen|Nadelen} & \textbf{Integraties} & \textbf{Prijs} \\
    \midrule
    \endfirsthead

    % Repeat the headers on the next page
    \multicolumn{4}{c}{{\bfseries \tablename\ \thetable{} -- vervolg van de vorige pagina}} \\
    \toprule
    \textbf{Platform} & \textbf{Voordelen|Nadelen} & \textbf{Integraties} & \textbf{Prijs} \\
    \midrule
    \endhead

    % Footer at the end of each page, except the last page
    \midrule
    \multicolumn{4}{r}{{Vervolg op volgende pagina}} \\
    \endfoot

    % Footer at the end of the table
    \bottomrule
    \endlastfoot

    % Table content
    Zoho Creator &
    \vspace{-\topsep}\vspace{-\partopsep} 
    \begin{enumerate}[leftmargin=2pt, topsep=0pt,parsep=0pt,noitemsep]
        \item[] AI Zia die helpt bij data te structureren.
        \item[] Je kan je eigen workflows creëren.
        \item[] Grote selectie aan pre-built templates en componenten.
    \end{enumerate}
    \begin{enumerate}[leftmargin=2pt, topsep=8pt,parsep=0pt,noitemsep]
        \item[] Gelimiteerd aantal features.
        \item[] Database limitaties.
        \item[] Zwakke Customer Support.
    \end{enumerate}
    & 
    \vspace{-\topsep}\vspace{-\partopsep} 
    \begin{itemize}[leftmargin=2pt, topsep=0pt,parsep=0pt,noitemsep]
        \item[] 600+ apps
        \item[] Airtable: Neen
        \item[] MAKE.com: Ja
    \end{itemize}
     &
    €12 - €37 (per gebruiker per maand)\\

    OutSystems & 
    \vspace{-\topsep}\vspace{-\partopsep} 
    \begin{itemize}[leftmargin=2pt, topsep=0pt,parsep=0pt,noitemsep]
        \item[] Ingebouwde templates en componenten.
        \item[] Heeft een AI die heel wat dingen kan doen zoals componenten genereren, errors vinden en bugs detecteren. 
        \item[] Zeer goed platform voor samenwerking tussen developers.
    \end{itemize} 
    \begin{itemize}[leftmargin=2pt, topsep=8pt,parsep=0pt,noitemsep]
        \item[]  Duur platform.
        \item[]  Gelimiteerd aantal features.
        \item[]  Bedraagt een hoge leercurve.
    \end{itemize} &
    \vspace{-\topsep}\vspace{-\partopsep} 
    \begin{itemize}[leftmargin=2pt, topsep=0pt,parsep=0pt,noitemsep]
        \item[] externe databronnen, enterprise systemen, en API's. 
    \end{itemize}
    &
    Multiple Apps abonnement kost €1.250 (per maand)\\

    Microsoft PowerApps & 
    \vspace{-\topsep}\vspace{-\partopsep} 
    \begin{itemize}[leftmargin=2pt, topsep=0pt,parsep=0pt,noitemsep]
        \item[] Robuuste veiligheid voor zowel de applicatie als data op elke laag.
        \item[] Makkelijk platform om te gebruiken.
        \item[] Een robuuste reeks aan functies.
    \end{itemize}
    \begin{itemize}[leftmargin=2pt, topsep=8pt,parsep=0pt,noitemsep]
        \item[] Het platform kan soms traag zijn.
        \item[] Moeilijk om met externe systemen te integreren.
        \item[] Gemaakte apps kunnen niet rechtstreeks gepubliceerd worden op Apple App Store, Google play store en Windows Store.
    \end{itemize} &
    \vspace{-\topsep}\vspace{-\partopsep} 
    \begin{itemize}[leftmargin=2pt, topsep=0pt,parsep=0pt,noitemsep]
        \item[]  400+ apps
        \item[]  Airtable: Ja
        \item[]  MAKE.com: Neen
    \end{itemize} &
    €4,60 - €18,70 (per gebruiker per maand)\\


    Appian & 
    \vspace{-\topsep}\vspace{-\partopsep} 
    \begin{itemize}[leftmargin=2pt, topsep=0pt,parsep=0pt,noitemsep]
        \item[] Met Appain Quick Apps heb je een echte no-code ervaring.
        \item[] Kan gebruikt worden om native mobiele applicaties te maken.
        \item[] Ingebouwde teamsamenwerking, taakbeheer systeem en bedrijfsnetwerkplatform.
    \end{itemize} 
    \begin{itemize}[leftmargin=2pt, topsep=8pt,parsep=0pt,noitemsep]
        \item[] Tegenover andere Low-Code platformen is het zeer duur.
        \item[] Gebruikersinterface is niet meer up-to-date. 
        \item[] Integreren vraagt ervaring in het programmeren.
    \end{itemize} &
    \vspace{-\topsep}\vspace{-\partopsep} 
    \begin{itemize}[leftmargin=2pt, topsep=0pt,parsep=0pt,noitemsep]
        \item[]  MAKE.com: Neen
    \end{itemize}
    &
    Standard abonnment kost €7 per gebruiker per maand\\

    % Wix & 
    % \vspace{-\topsep}\vspace{-\partopsep} 
    % \begin{itemize}[leftmargin=2pt, topsep=0pt,parsep=0pt,noitemsep]
    %     \item[] Beschikt over AI tools om u te helpen bij het maken van een website.
    %     \item[] Beschikt over een ingebouwde email marketing tool.
    %     \item[] Maar liefst 900+ professionele ontworpen templates voor uw website.
    % \end{itemize} 
    % \begin{itemize}[leftmargin=2pt, topsep=8pt,parsep=0pt,noitemsep]
    %     \item[] Het platform is meer bedoeld voor winkels.
    %     \item[] Door de vele functies kan het platform overweldigend zijn voor beginners.
    %     \item[] Ongelimiteerde opslag is enkel verkrijgbaar bij het duurste abonnement.
    % \end{itemize} &
    % \vspace{-\topsep}\vspace{-\partopsep} 
    % \begin{itemize}[leftmargin=2pt, topsep=0pt,parsep=0pt,noitemsep]
    %     \item[]  821+ apps
    %     \item[]  MAKE.com: Ja
    % \end{itemize}
    %  &
    % €11 tot €149 (per maand)\\

    Bubble & 
    \vspace{-\topsep}\vspace{-\partopsep} 
    \begin{itemize}[leftmargin=2pt, topsep=0pt,parsep=0pt,noitemsep]
        \item[] Heeft voor elk onderwerp tutorials.
        \item[] Geen enkele ervaring van programmeren nodig.
        \item[] Bevat een community marketplace voor templates en plugins.
    \end{itemize}
    \begin{itemize}[leftmargin=2pt, topsep=8pt,parsep=0pt,noitemsep]
        \item[] Door het gebrek aan coderen kan het platform beperkt zijn.
        \item[] Kan snel duur worden op langere termijn.
        \item[] Je kan geen python of andere scripts uitvoeren.
    \end{itemize} &
    \vspace{-\topsep}\vspace{-\partopsep} 
    \begin{itemize}[leftmargin=2pt, topsep=0pt,parsep=0pt,noitemsep]
        \item[]  Airtable: Ja
        \item[]  MAKE.com: Ja
    \end{itemize}
    &
    €27 - €327 (per maand)\\
\end{longtable}
\section{Conclusie Alternatieven}%
\label{sec:conclusie-alternatieven}
Uit de samenvatting van de alternatieven heeft mijn copromotor kunnen concluderen dat Bubble het beste alternatief is om verder uit te werken.
Dit komt omdat Bubble het beste past voor applicaties dat het bedrijf Quivvy Solutions BV bouwt, dit is niet het geval bij Zoho Creator en Wix. 
Bubble kan namelijk ook eenvoudig met zowel AirTable als MAKE.com integreren. Daarbij valt de prijs ook goed mee voor het bedrijf, wat niet het geval was bij OutSystems (1.250€ per maand).
