%%=============================================================================
%% Samenvatting
%%=============================================================================

% TODO: De "abstract" of samenvatting is een kernachtige (~ 1 blz. voor een
% thesis) synthese van het document.
%
% Een goede abstract biedt een kernachtig antwoord op volgende vragen:
%
% 1. Waarover gaat de bachelorproef?
% 2. Waarom heb je er over geschreven?
% 3. Hoe heb je het onderzoek uitgevoerd?
% 4. Wat waren de resultaten? Wat blijkt uit je onderzoek?
% 5. Wat betekenen je resultaten? Wat is de relevantie voor het werkveld?
%
% Daarom bestaat een abstract uit volgende componenten:
%
% - inleiding + kaderen thema
% - probleemstelling
% - (centrale) onderzoeksvraag
% - onderzoeksdoelstelling
% - methodologie
% - resultaten (beperk tot de belangrijkste, relevant voor de onderzoeksvraag)
% - conclusies, aanbevelingen, beperkingen
%
% LET OP! Een samenvatting is GEEN voorwoord!

%%---------- Nederlandse samenvatting -----------------------------------------
%
% TODO: Als je je bachelorproef in het Engels schrijft, moet je eerst een
% Nederlandse samenvatting invoegen. Haal daarvoor onderstaande code uit
% commentaar.
% Wie zijn bachelorproef in het Nederlands schrijft, kan dit negeren, de inhoud
% wordt niet in het document ingevoegd.

\IfLanguageName{english}{%
\selectlanguage{dutch}
\chapter*{Samenvatting}
\lipsum[1-4]
\selectlanguage{english}
}{}

%%---------- Samenvatting -----------------------------------------------------
% De samenvatting in de hoofdtaal van het document

\chapter*{\IfLanguageName{dutch}{Samenvatting}{Abstract}}

De zeer snelle opmars van digitalisatie zorgt ervoor dat kleine softwarebedrijven zoals Quivvy Solutions zorgvuldig 
moeten bepalen welke platformen men gebruikt om software zo snel mogelijk te ontwikkelen.
 De platformen die hier van toepassing zijn, maken het mogelijk om mobile en web toepassingen 
 te creëren aan de hand van een drag-and-drop systeem. Maar welke platform is nu het meest geschikt voor Quivvy Solutions waarbij het 
 platform noodzakelijk moet kunnen integreren met MAKE.com en Airtable?\\\\ 
 
 Om dit te kunnen bepalen, werd er eerst een requirements analyse gedaan om de belangrijkste vereisten te bepalen.
 Vervolgens is er een analyse gedaan van no-code en low-code platforms die 
 vergelijkbaar zijn aan Softr en Stacker. Daarnaast worden de drie platformen uitgebreid geanalyseerd door middel van een vergelijkende analyse en Proof of Concept. 
 Deze Proof of Concept bestaat uit het maken van een eenvoudige applicatie, op elk platform, door drie niet-programmeurs en een programmeur.\\\\

 Uit deze methodologie bleek dat het alternatief platform genaamd Bubble op verschillende criteria superieur was zoals; platformflexibiliteit, integratie mogelijkheden 
 en kostprijs. Hieruit kunnen we concluderen dat Bubble het meest geschikt zou kunnen zijn voor een softwarebedrijf als Quivvy Solutions. 
 Ook belangrijk om te vermelden is dat de uitgevoerde Proof of Concept niet voldoende is om een definitieve beslissing te 
 nemen over welk platform het beste is voor Quivvy Solutions, door het kleine aantal testpersonen.
