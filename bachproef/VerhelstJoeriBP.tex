%===============================================================================
% LaTeX sjabloon voor de bachelorproef toegepaste informatica aan HOGENT
% Meer info op https://github.com/HoGentTIN/latex-hogent-report
%===============================================================================

\documentclass[dutch,dit,thesis]{hogentreport}

% TODO:
% - If necessary, replace the option `dit`' with your own department!
%   Valid entries are dbo, dbt, dgz, dit, dlo, dog, dsa, soa
% - If you write your thesis in English (remark: only possible after getting
%   explicit approval!), remove the option "dutch," or replace with "english".

\usepackage{lipsum} % For blind text, can be removed after adding actual content

%% Pictures to include in the text can be put in the graphics/ folder
\graphicspath{{graphics/}}

%% For source code highlighting, requires pygments to be installed
%% Compile with the -shell-escape flag!
\usepackage[section]{minted}
%% If you compile with the make_thesis.{bat,sh} script, use the following
%% import instead:
%% \usepackage[section,outputdir=../output]{minted}
\usemintedstyle{solarized-light}
\definecolor{bg}{RGB}{253,246,227} %% Set the background color of the codeframe

%% Change this line to edit the line numbering style:
\renewcommand{\theFancyVerbLine}{\ttfamily\scriptsize\arabic{FancyVerbLine}}

%% Macro definition to load external java source files with \javacode{filename}:
\newmintedfile[javacode]{java}{
    bgcolor=bg,
    fontfamily=tt,
    linenos=true,
    numberblanklines=true,
    numbersep=5pt,
    gobble=0,
    framesep=2mm,
    funcnamehighlighting=true,
    tabsize=4,
    obeytabs=false,
    breaklines=true,
    mathescape=false
    samepage=false,
    showspaces=false,
    showtabs =false,
    texcl=false,
}

% Other packages not already included can be imported here

%%---------- Document metadata -------------------------------------------------
% TODO: Replace this with your own information
\author{Joeri Verhelst}
\supervisor{Dhr. F. Van Houte}
\cosupervisor{Mevr. S. Beeckman}
\title[Optionele ondertitel]%
    {Titel van de bachelorproef}
\academicyear{\advance\year by -1 \the\year--\advance\year by 1 \the\year}
\examperiod{1}
\degreesought{\IfLanguageName{dutch}{Professionele bachelor in de toegepaste informatica}{Bachelor of applied computer science}}
\partialthesis{false} %% To display 'in partial fulfilment'
%\institution{Internshipcompany BVBA.}

%% Add global exceptions to the hyphenation here
\hyphenation{back-slash}

%% The bibliography (style and settings are  found in hogentthesis.cls)
\addbibresource{bachproef.bib}            %% Bibliography file
\addbibresource{../voorstel/voorstel.bib} %% Bibliography research proposal
\defbibheading{bibempty}{}

%% Prevent empty pages for right-handed chapter starts in twoside mode
\renewcommand{\cleardoublepage}{\clearpage}

\renewcommand{\arraystretch}{1.2}

%% Content starts here.
\begin{document}

%---------- Front matter -------------------------------------------------------

\frontmatter

\hypersetup{pageanchor=false} %% Disable page numbering references
%% Render a Dutch outer title page if the main language is English
\IfLanguageName{english}{%
    %% If necessary, information can be changed here
    \degreesought{Professionele Bachelor toegepaste informatica}%
    \begin{otherlanguage}{dutch}%
       \maketitle%
    \end{otherlanguage}%
}{}

%% Generates title page content
\maketitle
\hypersetup{pageanchor=true}

%%=============================================================================
%% Voorwoord
%%=============================================================================

\chapter*{\IfLanguageName{dutch}{Woord vooraf}{Preface}}%
\label{ch:voorwoord}

Hedendaags spelen softwarebedrijven een cruciale rol in de digitale wereld. 
Door de exponentiele groei van digitalisatie moeten softwarebedrijven, vooral kleine bedrijven, 
een stapje hogerop op het vlak van snelle applicatieontwikkeling. Hierdoor hebben softwarebedrijven nood aan 
een platform die toelaat om web en mobile toepassingen te creëren zonder diepgaande kennis voor programmeren. 
Twee prominente spelers zijn Stacker en Softr. Deze platformen bieden de mogelijkheid aan om krachtige apps te bouwen 
door middel van een drag-and-drop systeem met integratie mogelijkheden. In deze studie worden deze platformen op de proef gezet, 
met nog een alternatief platform, om te bepalen welke platform het meest geschikt is voor Quivvy Solutions.
\\
\\
Het onderwerp vond ik interessant doordat mijn opleiding Toegepaste Informatica aan de Hogeschool Gent niks geeft over Low-Code en No-Code platformen.
Dit bracht me in een onbekend terrein waar ik zelf moest uitzoeken wat deze platformen inhouden en hoe ze werken. Dit onderwerp werd dan ook toegelicht aan mij via 
het softwarebedrijf Quivvy Solutions, waar ik mijn stage heb gedaan.
\\
Ik zou graag als eerste mijn copromotor bedanken die mij geholpen heeft met het opstellen van de vereisten en de testen maar ook
in het algemeen professionele advies gaf. Vervolgens wil ik mijn promotor bedanken die mij heeft bijgestaan met het opstellen en verbeteren van de bachelorproef. Daarnaast 
wil ten zeerste mijn ouders en broer bedanken voor hun steun en hulp bij mijn Proof of Concept. Als laatste wil ik mijn moeder bedanken voor het nalezen en advies geven op vlak van grammatica van mijn bachelorproef.
%% TODO:
%% Het voorwoord is het enige deel van de bachelorproef waar je vanuit je
%% eigen standpunt (``ik-vorm'') mag schrijven. Je kan hier bv. motiveren
%% waarom jij het onderwerp wil bespreken.
%% Vergeet ook niet te bedanken wie je geholpen/gesteund/... heeft
%%=============================================================================
%% Samenvatting
%%=============================================================================

% TODO: De "abstract" of samenvatting is een kernachtige (~ 1 blz. voor een
% thesis) synthese van het document.
%
% Een goede abstract biedt een kernachtig antwoord op volgende vragen:
%
% 1. Waarover gaat de bachelorproef?
% 2. Waarom heb je er over geschreven?
% 3. Hoe heb je het onderzoek uitgevoerd?
% 4. Wat waren de resultaten? Wat blijkt uit je onderzoek?
% 5. Wat betekenen je resultaten? Wat is de relevantie voor het werkveld?
%
% Daarom bestaat een abstract uit volgende componenten:
%
% - inleiding + kaderen thema
% - probleemstelling
% - (centrale) onderzoeksvraag
% - onderzoeksdoelstelling
% - methodologie
% - resultaten (beperk tot de belangrijkste, relevant voor de onderzoeksvraag)
% - conclusies, aanbevelingen, beperkingen
%
% LET OP! Een samenvatting is GEEN voorwoord!

%%---------- Nederlandse samenvatting -----------------------------------------
%
% TODO: Als je je bachelorproef in het Engels schrijft, moet je eerst een
% Nederlandse samenvatting invoegen. Haal daarvoor onderstaande code uit
% commentaar.
% Wie zijn bachelorproef in het Nederlands schrijft, kan dit negeren, de inhoud
% wordt niet in het document ingevoegd.

\IfLanguageName{english}{%
\selectlanguage{dutch}
\chapter*{Samenvatting}
\lipsum[1-4]
\selectlanguage{english}
}{}

%%---------- Samenvatting -----------------------------------------------------
% De samenvatting in de hoofdtaal van het document

\chapter*{\IfLanguageName{dutch}{Samenvatting}{Abstract}}

De zeer snelle opmars van digitalisatie zorgt ervoor dat kleine softwarebedrijven zoals Quivvy Solutions zorgvuldig 
moeten bepalen welke platformen men gebruikt om software zo snel mogelijk te ontwikkelen.
 De platformen die hier van toepassing zijn, maken het mogelijk om mobile en web toepassingen 
 te creëren aan de hand van een drag-and-drop systeem. Maar welke platform is nu het meest geschikt voor Quivvy Solutions waarbij het 
 platform noodzakelijk moet kunnen integreren met MAKE.com en Airtable? Om dit te kunnen bepalen, werd er een analyse gedaan van no-code en low-code platforms die 
 vergelijkbaar zijn aan Softr en Stacker. Daarnaast worden de drie platformen uitgebreid geanalyseerd door middel van een vergelijkende analyse en Proof of Concept. 
 Uit deze methodologie bleek dat het alternatief platform genaamd Bubble op verschillende criteria superieur was zoals; platformflexibiliteit, integratie mogelijkheden 
 en kostprijs. Hieruit kunnen we concluderen dat Bubble het meest geschikt zou kunnen zijn voor een softwarebedrijf als Quivvy Solutions. 
 Ook belangrijk om te vermelden is dat de uitgevoerde Proof of Concept niet voldoende is om een definitieve beslissing te 
 nemen over welk platform het beste is voor Quivvy Solutions, door het kleine aantal testpersonen.


%---------- Inhoud, lijst figuren, ... -----------------------------------------

\tableofcontents

% In a list of figures, the complete caption will be included. To prevent this,
% ALWAYS add a short description in the caption!
%
%  \caption[short description]{elaborate description}
%
% If you do, only the short description will be used in the list of figures

\listoffigures

% If you included tables and/or source code listings, uncomment the appropriate
% lines.
%\listoftables
%\listoflistings

% Als je een lijst van afkortingen of termen wil toevoegen, dan hoort die
% hier thuis. Gebruik bijvoorbeeld de ``glossaries'' package.
% https://www.overleaf.com/learn/latex/Glossaries

%---------- Kern ---------------------------------------------------------------

\mainmatter{}

% De eerste hoofdstukken van een bachelorproef zijn meestal een inleiding op
% het onderwerp, literatuurstudie en verantwoording methodologie.
% Aarzel niet om een meer beschrijvende titel aan deze hoofdstukken te geven of
% om bijvoorbeeld de inleiding en/of stand van zaken over meerdere hoofdstukken
% te verspreiden!

%%=============================================================================
%% Inleiding
%%=============================================================================

\chapter{\IfLanguageName{dutch}{Inleiding}{Introduction}}%
\label{ch:inleiding}

%De inleiding moet de lezer net genoeg informatie verschaffen om het onderwerp te begrijpen en in te zien waarom de onderzoeksvraag de moeite waard is om te onderzoeken. In de inleiding ga je literatuurverwijzingen beperken, zodat de tekst vlot leesbaar blijft. Je kan de inleiding verder onderverdelen in secties als dit de tekst verduidelijkt. Zaken die aan bod kunnen komen in de inleiding~\autocite{Pollefliet2011}:

%\begin{itemize}
 % \item context, achtergrond
 % \item afbakenen van het onderwerp
 % \item verantwoording van het onderwerp, methodologie
 % \item probleemstelling
 % \item onderzoeksdoelstelling
 % \item onderzoeksvraag
 % \item \ldots
%\end{itemize}

Softwarebedrijven merken op dat projecten telkens over het budget gaan. Daarbij doet het project meestal niet wat de eindklant verwacht en vervolgens doet het opgeleverde product niet wat het zou moeten doen ~\autocite{Moskal_2021}.
Maar volgens ~\textcite{Moskal_2021} zijn deze problemen niet enkel op te merken in de softwarewereld maar ook in andere categorieën binnen de IT-sector. 
Daarnaast moeten softwarebedrijven ook rekening houden dat er in deze tijdsperiode meer opslag en verwerking van data is.
Volgens ~\textcite{Moskal_2021} en ~\textcite{Parviainen_2022} brengt het verwerken en opslaan van data veranderingen in de business mee, 
doordat bedrijven zich meer digitaliseren. Maar digitalisering betekent dat men producten of diensten moet omzetten naar een digitaal product 
of dienst. Het kan ook zijn dat men software producten moet kopen om de business processen te automatiseren ~\autocite{Moskal_2021}. 
Het verkrijgen van een competitief voordeel kan men bereiken door een specifiek ontworpen en ontwikkelde softwareoplossing die voldoet aan de behoeften van de eindklant.
Maar volgens ~\textcite{Moskal_2021} is een toegewijde IT-oplossing zo duur dat bedrijven dit niet kunnen veroorloven. Dit zet No-Code en Low-Code platformen in de spotlight
door de functies zoals het integreren van data via tools en een drag-and-drop systeem om de ontwikkeling te versnellen \autocite{Kulkarni_2021}. Met deze platformen kan je software ontwikkelen door gebruik van een minimale code
of zonder code door middel van het drag-and-drop systeem.

\section{\IfLanguageName{dutch}{Probleemstelling}{Problem Statement}}%
\label{sec:probleemstelling}
Als jong en klein bedrijf is Quivvy Solutions constant opzoek naar de beste en snelste platformen om hun software te ontwikkelen. 
Doordat er constant nieuwe technologieën bijkomen en veranderen heeft Quivvy Solutions zich beperkt tot het ontwikkelen van software 
via low-code en/of no-code platformen. Hieruit volgt dat het bedrijf nog niet heeft kunnen bepalen welke platform nu het beste past bij hun.
%Uit je probleemstelling moet duidelijk zijn dat je onderzoek een meerwaarde heeft voor een concrete doelgroep. De doelgroep moet goed gedefinieerd en afgelijnd zijn. Doelgroepen als ``bedrijven,'' ``KMO's'', systeembeheerders, enz.~zijn nog te vaag. Als je een lijstje kan maken van de personen/organisaties die een meerwaarde zullen vinden in deze bachelorproef (dit is eigenlijk je steekproefkader), dan is dat een indicatie dat de doelgroep goed gedefinieerd is. Dit kan een enkel bedrijf zijn of zelfs één persoon (je co-promotor/opdrachtgever).

\section{\IfLanguageName{dutch}{Onderzoeksvraag}{Research question}}%
\label{sec:onderzoeksvraag}
Met welke Low-Code en/of No-Code platform kan Quivvy Solutions zowel mobile als web toepassingen creëren, die ook kunnen integreren met
zowel MAKE.com als Airtable?
%Wees zo concreet mogelijk bij het formuleren van je onderzoeksvraag. Een onderzoeksvraag is trouwens iets waar nog niemand op dit moment een antwoord heeft (voor zover je kan nagaan). Het opzoeken van bestaande informatie (bv. ``welke tools bestaan er voor deze toepassing?'') is dus geen onderzoeksvraag. Je kan de onderzoeksvraag verder specifiëren in deelvragen. Bv.~als je onderzoek gaat over performantiemetingen, dan 

\section{\IfLanguageName{dutch}{Onderzoeksdoelstelling}{Research objective}}%
\label{sec:onderzoeksdoelstelling}
Voor de onderzoeksvraag zo goed mogelijk te kunnen beantwoorden zal er een uitgebreide onderzoek plaatsnemen waarbij het begint met het opstellen 
van de vereisten waaraan een platform moet voldaan. Vervolgens zal er een analyse gedaan worden tussen verschillende alternatieven om daaruit nog een platform 
te analyseren. Daarna wordt er voor Stacker, Softr en het alternatief platform een vergelijkende analyse gedaan op basis van de vereisten; snelheid van het platform, 
herstelbeheer, veiligheid, snelheid van applicatieontwikkeling, integratie mogelijkheden, platformflexibiliteit, schaalbaarheid, gebruiksvriendelijkheid, integratie met 
Airtable en MAKE.com, kostprijs, en updatebeleid. Om bepaalde criteria zorgvuldig te kunnen testen zal er een Proof of Concept plaatsnemen waarbij drie 
niet-programmeurs en één programmeur de drie platformen uittesten. Men neemt aan dat een platform het beste past bij het softwarebedrijf Quivvy Solutions wanneer het superieur is in alle criteria, en 
waarbij het met zowel Airtable als MAKE.com kan integreren.
%Wat is het beoogde resultaat van je bachelorproef? Wat zijn de criteria voor succes? Beschrijf die zo concreet mogelijk. Gaat het bv.\ om een proof-of-concept, een prototype, een verslag met aanbevelingen, een vergelijkende studie, enz.

\section{\IfLanguageName{dutch}{Opzet van deze bachelorproef}{Structure of this bachelor thesis}}%
\label{sec:opzet-bachelorproef}

% Het is gebruikelijk aan het einde van de inleiding een overzicht te
% geven van de opbouw van de rest van de tekst. Deze sectie bevat al een aanzet
% die je kan aanvullen/aanpassen in functie van je eigen tekst.

De rest van deze bachelorproef is als volgt opgebouwd:

In Hoofdstuk~\ref{ch:stand-van-zaken} wordt een overzicht gegeven van de stand van zaken binnen het onderzoeksdomein, op basis van een literatuurstudie.

In Hoofdstuk~\ref{ch:methodologie} wordt de methodologie toegelicht en worden de gebruikte onderzoekstechnieken besproken om een antwoord te kunnen formuleren op de onderzoeksvragen.

% TODO: Vul hier aan voor je eigen hoofstukken, één of twee zinnen per hoofdstuk
In Hoofdstuk~\ref{ch:vereisten} wordt een lijst van criteria opgesteld waarop het platform getest wordt.

In Hoofdstuk~\ref{ch:evaluatie-en-selectie-van-alternatieven}, wordt een analyse gedaan tussen verschillende alternatieven platformen die mogelijk geschikt zijn voor Quivvy Solutions.

In Hoofdstuk~\ref{ch:vergelijkende-analyse} wordt er een vergelijkende analyse gedaan tussen Stacker, Softr en het alternatief platform, op basis van de vereisten.

In Hoofdstuk~\ref{ch:proof-of-concept} neemt er een proof-of-concept plaats waarbij drie niet-programmeurs en één programmeur de drie platformen op bepaalde criteria uittesten.

In Hoofdstuk~\ref{ch:conclusie}, tenslotte, wordt de conclusie gegeven en een antwoord geformuleerd op de onderzoeksvragen. Daarbij wordt ook een aanzet gegeven voor toekomstig onderzoek binnen dit domein.
\chapter{\IfLanguageName{dutch}{Stand van zaken}{State of the art}}%
\label{ch:stand-van-zaken}

\section{Inleiding}%
\label{sec:inleiding}
Projecten in softwarebedrijven gaan meestal over het budget. Daarbij doet het project meestal niet volledig wat de eindklant verwacht en vervolgens
doet het opgeleverde product niet wat het zou moeten doen ~\autocite{Moskal_2021}. Maar volgens ~\textcite{Moskal_2021} zijn deze problemen niet alleen op te merken 
in de softwarewereld maar ook in andere categorieën binnen de IT-sector, of bij het implementeren en ontwerpen van software systemen. In deze tijdsperiode worden er steeds meer
data verwerkt en ook opgeslaan ~\autocite{Moskal_2021}. Volgens ~\textcite{Moskal_2021} en ~\textcite{Parviainen_2022} veroorzaakt dit proces van het verwerken van data en opslaan
verandingen in de business, door de adoptie van digtale technologieën als gevolg tot digitalisering. Maar digitalisering betekent dat men bestaande producten of diensten moeten omzetten
naar een digitaal product of dienst, het kan ook zo zijn dat men software producten moeten kopen om de business processen te automatiseren ~\autocite{Moskal_2021}. Het verkrijgen van een competitief voordeel
kan bereikt worden door een specifiek ontworpen en ontwikkeld softwareoplossing dat voldoet aan de unieke behoeften van de eindklant ~\autocite{Moskal_2021}. Maar volgens ~\textcite{Moskal_2021} is een toegewijde IT-oplossing
zeer duur waardoor heel wat bedrijven dit niet kan veroorloven. Dit zet No-Code en Low-Code platformen in de spotlight  ~\autocite{Moskal_2021}.

\section{Reden tot gebruik}%
\label{sec:reden-tot-gebruik}
In dit hoofdstuk van de literatuurstudie gaan we ons verdiepen in de reden waarom Low-Code en No-Code platformen toch stijgen ze in gebruik. Dit zal helpen
om een beter beeld te krijgen waarom dit soort platformen, op dit moment, toch waard zijn om eens een kijkje in te nemen, voor zowel het bedrijf als de eindklant van het bedrijf.

In vergelijking met andere technologie trends zoals AI, Blockchain, Edge Computing en RPA groeit Low-Code en No-Code platformen zeer matig ~\autocite{Kulkarni_2021}.
Dit komt omdat het idee van Low-Code en No-Code development niet nieuw is, maar toch is er een stijging in het gebruik van deze platformen ~\autocite{Elshan2023}.
Volgens ~\textcite{Elshan2023} en ~\textcite{Kulkarni_2021} zijn er verschillende redenen waarom deze platformen toch stijgen in gebruik, vooral in kleine en middelgrote bedrijven.
\begin{itemize}
    \item \textbf{Beperkt aantal programmeurs}: 
    Low-Code platformen werden geïntroduceerd als een oplossing voor de dilemma tussen het te kort aan programmeurs en de hoge vraag naar softwareontwikkeling ~\autocite{ALSAADI_2021}. Volgens 
    ~\textcite{Moskal_2021} is de reden tot te kort aan programmeurs te wijten aan dat heel wat studenten zich laat afschrikken door de complexiteit van het programmeren. Dit geeft als gevolg dat weinig 
    studenten een diepe kennis hebben op het vlak van programmeren. Maar volgens ~\textcite{Moskal_2021} brengt een zeer gekwalificeerde programmeur het project niet altijd tot een goed einde.
    \item \textbf{Technologische turbulentie}:
    Doordat programmeertalen steeds veranderen en er steeds nieuwe technologieën worden geïntroduceerd is de kennis van de programmeurs niet altijd up-to-date ~\autocite{Moskal_2021}.
    \item \textbf{Hoge kosten}:
    De tradionele softwareontwikkeling eist een grote tol op de financiën van een bedrijf. Daarbij beseffen Software engineers dat het bouwen van een applicatie
    niet gemakkelijk is binnen het gegeven budget  ~\autocite{Moskal_2021}. Volgens ~\textcite{Elshan2023} zijn concepten zoals DevOps en BizOps, die de operations, developers en bussiness teams samenbrengen,
    overspoeld met moeilijkheden. Als gevolg van kosten dat over het budget gaat en requirement conflicten tussen de teams ~\autocite{Elshan2023}. Low-Code en No-Code platformen kunnen hier een
    oplossing bieden omdat het minder tijd en geld kost om een applicatie te bouwen ~\autocite{Elshan2023} ~\autocite{Bock_2021} ~\autocite{Rokis_2023}.
    \item \textbf{Tijdrovend}:
    Het ontwikkelen van software is een tijdrovend proces. De geschatte tijd dat nodig is om een applicatie te bouwen, op één operating systeem, is vaak
     zes maanden of langer ~\autocite{Moskal_2021}. Het probleem hiervan is dat snelheid binnen de business een belangrijke factor is ~\autocite{Sanchis_2019}.
     Want volgens ~\textcite{Sanchis_2019} betekent een veranderende markt dat bedrijven snel en flexibel moeten kunnen reageren op verandering om aan de requirements van de omgeving te kunnen voldoen.

    \item \textbf{Klantentevredenheid}:
    Low-Code en No-Code platformen zorgen voor een hogere klantentevredenheid ~\autocite{Elshan2023}.
    Volgens ~\textcite{Elshan2023} ontstaat er een soort van rolomkering plaats in het ontwikkelingsproject. De ontwikkelaars van het bedrijf worden niet langer meer gezien als de enige
    opdrachtgevers van een applicatie, maar ook de eindklanten van het bedrijf. Hierdoor kan de eindklant zelf de applicatie aanpassen naar zijn eigen wensen en behoeften ~\autocite{Elshan2023}.
    Volgens ~\textcite{Elshan2023} reduceert dit ook de misverstanden tussen de ontwikkelaars en de eindklant.
    \item \textbf{Digitale Transformatie}: 
    Niet alleen in de IT-sector maar ook in de bussines sector is er een digitale transformatie aan de gang. 
    Deze transformatie binnen de business omgeving zorgt voor een nood aan automatiseren in verschillende aspecten van de business ~\autocite{ALSAADI_2021}.
    Wat als gevolg heeft dat men papier documenten vervangen door digitale documenten om de fysieke processen te vervangen door digitale processen ~\autocite{ALSAADI_2021}.
    Volgens ~\textcite{ALSAADI_2021} merkt men op dat er een daling is van de nood aan menselijke tussenkomst in de processen omdat de automatisering gebruik maakt van vertrouwlijke software 
    dat minder fouten maakt en ook nog eens minder kosten met zich meebrengt ~\autocite{ALSAADI_2021}.
    \item \textbf{Complexe softwareontwikkeling}
  \end{itemize} 

\section{Voordelen en Nadelen}
\label{sec:voordelen-nadelen}
Op gevolg van het vorige hoofdstuk "Reden tot gebruik" zal er nu gekeken worden naar wat de voordelen en nadelen zijn van Low-Code en No-Code platformen. Dit zal
een overzicht geven van wat er allemaal mogelijk is en wat de limieten zijn van deze platformen.
%Hier komt de "voordelen en nadelen" van low-code en no-code platformen Beperkt aantal programmeurs, .... maar met meer tekst (gebruik voorstel)
\subsection{Voordelen}%
\label{subsec:voordelen}
\subsubsection{Snelheid}
\label{subsec:snelheid}
Zoals eerder besproken in het vorige hoofdstuk neemt de tradionele softwareontwikkeling veel tijd in beslag ~\autocite{Moskal_2021}. Dit is niet het geval 
bij LCNC platformen, wat volgens ~\textcite{Adrian_2020} een groot voordeel is. Vervolgens kan dit ook nog eens versterkt worden door ~\textcite{Yan2021}
die verteld dat heel wat bedrijven die gebruikmaken van Low-Code platformen vaststellen dat hun release van een applicatie bij 5 van de 10 keer sneller was dan voorheen.
In 2019 werd er dan ook een enquête gehouden van OutSystems, waaruit bleek dat gebruikers van Low-Code platformen 68\% van hun webapplicaties en 64\% van hun apps elk konden
bouwen in vier maanden ~\autocite{Yan2021}. Tegenover de traditionele ontwikkeling is dit een positief resultaat, want volgens ~\textcite{Yan2021}
was er maar 57\% van de webapplicaties en 49\% van de apps gebouwd binnen dezelfde tijdspanne.
\\
\\
Niet alleen het bouwen van applicaties gaat sneller, maar volgens ~\textcite{da_Cruz_2021} is het grootste voordeel
van Low-Code en No-Code platformen te danken aan het eenvoudig bouwen van complexe software waardoor bedrijven
sneller kunnen reageren op de veranderende markt. Daarnaast is het leren van LCNC platformen gemakkelijker en sneller dan het leren van 
een programmeertaal. Al deze eigenschappen zorgen ervoor dat developers sneller een prototype kunnen maken, feedback krijgen en een betere
klantenervaring kunnen bieden ~\autocite{da_Cruz_2021}.

\subsubsection{Veiligheid}
\label{subsec:veiligheid}
Het gebrek aan werknemers binnen de IT-sector, die heel wat massa aan software moeten ontwikkelen, zorgt ervoor dat de werknemers buiten de IT-sector
er vaak alleen voor staan waardoor ze gebruik moeten maken van third-party software ~\autocite{Yan2021}. Volgens ~\textcite{Yan2021} is dit een groot probleem,
want dit kan schade brengen op het bedrijf ~\autocite{Yan2021}. De oorzaak hiervan zou kunnen zijn dat ze niet op de hoogte zijn van de licentie en veiligheid, dit 
wordt ook wel "Shadow IT" genoemd ~\autocite{Yan2021} ~\autocite{Rokis_2022}. Daarom zorgt LCNC platformen, die geautoriseerd zijn door de IT-sector,
op vermindering van "Shadow IT" ~\autocite{Yan2021}. Daarnaast kan de werknemers buiten de IT-sector makkelijk een oplossing ontwikkelen met Low-Code en No-Code platformen  ~\autocite{Yan2021}.
Als gevolg dat de IT medewerkers niet telkens verstoord worden door andere werknemers ~\autocite{Yan2021}.  Dit biedt verschillende voordelen aan de IT-sector zoals het verminderen van de werkdruk en het verhogen van de veiligheid, want
LCNC platformen bevat de internationale standaarden voor veiligheid (ISO/IEC 27001, PCIDSS) ~\autocite{Sufi_2023}.
\\
\\
Volgens ~\textcite{Sufi_2023} wordt er ook hedendaags een principe genaamd "Security by Design" toegepast. Dit principe neemt heel wat zorgen af op het vlak van veligheid
in de IT voor de Citizen Developers, ook wel de werknemers buiten de IT-sector genoemd ~\autocite{Sufi_2023}. Vervolgens verteld ~\textcite{Elshan2023} dat deze digitalisering en het automatiseren van werkprocessen
een positieve impact hebben op de kwaliteit van het bedrijf, wat als gevolg de veiligheid van de bedrijfsprocessen verhoogt. De reden dat het ook de veiligheid verhoogt is omdat de standaardisatie van de werkprocessen zowel
bronnen als menselijke fouten vermindert ~\autocite{Elshan2023}.

\subsubsection{Universele toegangkelijkheid}
\label{subsec:universele-toegangkelijkheid}

Eerder in de literatuurstudie werd er al gesproken over dat er een te kort is aan IT-personeel die over de kwaliteiten beschikt. 
Dit kan door ~\textcite{Sufi_2023} nog eens bevestigd worden waarbij 
verteld wordt dat bedrijven vaak falen bij het rekruteren van IT-personeel, door het gebrek aan IT'ers met de nodige kennis en ervaring.
Volgens ~\textcite{Sufi_2023} is hier een oplossing voor, namelijk Low-Code en No-Code platformen. Deze platformen laten niet-programmeurs, met een zeer basis kennis,
toe om te werken aan IT-oplossingen zoals: dashboards, applicaties, en databanken zonder problemen ~\autocite{Sufi_2023}. Dit lost het probleem op van het te kort aan IT-personeel binnen het bedrijf ~\autocite{Sufi_2023}.
Maar volgens ~\textcite{Sufi_2023} is dit niet de enige voordeel van universele toegankelijkheid want door LCNC platformen kunnen de niet-programmeurs ook
snel het benodigde IT-oplossing ontwikkelen, zonder het verspillen van tijd en geld ~\autocite{Sufi_2023}.
\\
\\
De reden waarom we kunnen spreken van universele toegankelijkheid is omdat Low-Code en No-Code platformen makkelijk te leren zijn ~\autocite{Sufi_2023} ~\autocite{ALSAADI_2021}.
Zoals vermeld door ~\textcite{ALSAADI_2021} zullen niet alleen professionele ontwikkelaars LCNC platformen benutten, maar ook de niet-programmeurs.
Beginners ofwel de niet-programmeurs hebben nu ook de mogelijjkheid om applicaties te bouwen zonder enige kennis van programmeertalen ~\autocite{ALSAADI_2021}.
Dit is allemaal mogelijk doordat LCNC-platformen een drag-en-drop systeem hebben, hierbij kunnen de gebruikeres de compenenten van de applicatie slepen en neerzetten waarbij vervolgens in de achtergrond de code wordt 
gegenereerd in de achtergrond ~\autocite{ALSAADI_2021}.

\subsubsection{Cloud-Based LCNC: Dataherstel}
\label{subsec:cloud-based-lcnc}
Hedendaags stappen heel wat bedrijven over naar cloud-based technologieën omdat het heel 
wat voordelen biedt zoals: kostenbesparing, schaalbaarheid, flexibiliteit, herstel van data, en nog heel wat andere voordelen ~\autocite{Sufi_2023}.
Volgens ~\textcites{Sufi_2023} zijn grotendeels alle LCNC platformen cloud-based wat als gevolg heeft dat
de LCNC snelle strategieën kan toepassen voor het migreren naar de cloud.
\\
\\
Doordat het cloud-based is, is het ook mogelijk om data te herstellen in het geval van een ramp ~\autocite{Sufi_2023}.
Deze platformen verzekeren dat het systeem regelmatig is geback-upt en dat de data hersteld kan worden in het geval van een ramp ~\autocite{Sufi_2023}.
Dit is een zeer belangrijk voordeel voor bedrijven want volgens ~\textcite{Sufi_2023} 
kan 20\% van cloud gebruikers hun data herstellen in vier uur of minder herstellen, terwijl 9\%
van de niet cloud gebruikers hun data herstellen in vier uur of minder ~\autocite{Sufi_2023}. Daarnaast riskeren developers, die niet gebruikmaken van de cloud,
dat ze hun data verliezen op hun computer in het geval van een ramp ~\autocite{Sufi_2023}. Dit is niet zo bij cloud-hosted services want volgens ~\textcite{Sufi_2023}
verzekeren deze services dat de data altijd beschikbaar is.

\subsection{Nadelen}%
\label{subsec:nadelen}
%Rood gemarkeerd in paper = nadelen
\subsubsection{Beperkte flexibiliteit}
\label{subsec:beperkte-flexibiliteit}
Niet-programmeurs kunnen snel expert worden in hun gekozen LCNC platform, maar dit betekent ook dat ze vastzitten aan de beperkingen en de framework van het platform ~\autocite{Sufi_2023} ~\autocite{Talesra_2021}.
Deze limitaties kunnen voor problemen zorgen wanneer de niet-programmeurs een applicatie moeten bouwen dat zeer aanpasbaar moet zijn ~\autocite{Talesra_2021}. Vervolgens hebben de meeste LCNC platformen ook een gelimiteerde
opties voor integratie met andere systemen, waardoor het zowel een uitdaging is voor de niet-programmeurs als het bedrijf ~\autocite{Talesra_2021}. Daarnaast bevat LCNC platformen ook third-party afhankelijkheden, wat als gevolg heeft dat
men afhankelijk is van de verkoper van de third-party bij het oplossen van veiligheid en prestatie problemen ~\autocite{Talesra_2021}.
Dit is niet het geval bij tradionele programmeertalen en platformen zoals Java, C\#, en Python ~\autocite{Sufi_2023}, waarbij de developers de vrijheid hebben om de applicatie te bouwen zoals ze zelf willen.
Jammer genoeg is er geen enkele bestaande LCNC-platformen die in 2023 deze flexibiliteit biedt ~\autocite{Sufi_2023}. Als gevolg dat niet-programmeurs
beperkt zijn op het vlak van beschikbare opties bij het ontwikkelen van hun oplossing ~\autocite{Sufi_2023}.
\\
\\
De beperkingen van LCNC platformen hebben te maken met dat de platformen bestaat uit gevisualiseerde componenten die de gebruiker kan slepen en neerzetten ~\autocite{Yan2021}.
Deze componenten zijn vooraf gedefinieerd en staan ook vast, wat als gevolg heeft dat het niet zo aanpasbaar is als een applicatie dat gebouwd is met een programmeertaal ~\autocite{Yan2021}.
Volgens ~\textcite{Yan2021} is het hierdoor moeilijk en tijd spenderend om ingewikkelde of aanpasbare features of functionaliteiten, dat niet door de platformen wordt aangeboden, te ontwikkelen.
\subsubsection*{Gelimiteerde schaalbaarheid}
\label{subsec:gelimiteerde-schaalbaarheid}
Volgens ~\textcite{Elshan2023} en ~\textcite{Sufi_2023} is het bouwen van applicaties met LCNC platformen op dit moment zeer gelimiteerd in schaalbaarheid. Daarom worden deze platformen
in hedendaags enkel gebruikt voor het ontwikkelen van kleine schaalbare applicaties ~\autocite{Sufi_2023}. Vervolgens kunnen de meeste platformen door limitatie niet gebruikt worden voor applicaties waarbij het in 
de toekomst zou moeten uitgebreid worden ~\autocite{Elshan2023}. Volgens ~\textcite{Yan2021} lag, in 2015, de gemiddelde runtime schaal van applicaties, die gemaakt zijn door LCNC-platformen, tussen de 200 en 2000 gelijktijdige gebruikers.

\subsection*{Veiligheidszorgen}
\label{subsec:veiligheidszorgen}
In de literatuur werd eerder verteld dat veiligheid een voordeel is van LCNC platformen, maar volgens ~\textcite{Yan2021} is dit niet altijd het geval. Ze kunnen namelijk moeilijk of zelfs niet aangepast worden, waardoor bedrijven, dat er gebruik van maken,
volledig de diensten van de verkoper moeten vertouwen op het niet genereren van veiligheidsproblemen ~\autocite{Yan2021}. Doordat het bedrijf volledig afhankelijk is van de verkoper, kan de data van het bedrijf in gevaar komen door een data lek bij de verkoper omdat namelijk de data security
en de source code niet beheerst wordt door het bedrijf ~\autocite{Yan2021}.

\subsection*{Vendor Lock-in}
\label{subsec:technische-schulden}
Volgens ~\textcite{Yan2021} betekent vendor lock-in dat je als bedrijf afhankelijk bent van de verkoper voor hun diensten en producten, wat als gevolg heeft dat het moeilijk is om als klant te veranderen naar een andere verkoper.
Het gevaar is dat dit ook zo kan zijn bij LCNC platformen, waarbij het bedrijf in de toekomst meer zal investeren in het platform van de verkoper ~\autocite{Yan2021}. Volgens ~\textcite{Yan2021} zal dit dan kunnen leiden tot stijgende kosten van de diensten en producten van de verkoper, en zal het bedrijf
moeilijker kunnen veranderen. Als het bedrijf toch van plan zou zijn om te veranderen, zal de applicatie helemaal opnieuw moeten gebouwd worden ~\autocite{Sufi_2023}. Volgens ~\textcite{Sufi_2023} is dit omdat de bestaande LCNC platformen beschikken niet over integreren van gebouwden applicaties tussen verschillende LCNC platformen, maar
ook omdat elke verkoper hun eigen ecosysteem hebben voor applicatie ontwikkeling ~\autocite{Sufi_2023}.


\section{LCNC uitdagingen}
\label{sec:lcnc-uitdagingen}
Hieronder lichten we toe wat de obstakels en problemen zijn die overwonnen moeten worden. Deze obstakels en problemen bieden een kans voor groei en
verbetering van de LCNC platformen.
%Oranje gemarkeerd in paper = nadelen
\subsubsection*{Weerstand binnen het bedrijf}
\label{subsec:weerstand-binnen-het-bedrijf}
%(1) in Elshan paper

\subsubsection*{Shadow IT}
\label{subsec:shadow-it}
%(2) in Elshan paper

\subsubsection*{User Interface}
\label{subsec:user-interface}
%(3) in Elshan paper


\subsubsection*{Gebrek aan eigen controle}
\label{subsec:gebrek-aan-eigen-controle}
%(3) in Sufi paper Lack of On-Premise Support

\section{LCNC binnen bedrijven}
\label{sec:lcnc-bedrijven}
Dit hoofdstuk zal meer inzicht geven in waarom deze platformen worden gebruikt binnen bedrijven. Dit zal een betere duidelijkheid geven
waarom deze bachelorproef relevant is voor software bedrijven, meer specifiek Quivvy Solutions BV.

\section{Gebruik van LCNC door eindklanten}
\label{sec:lcnc-eindklanten}
Dit is een opvolging van het vorige hoofdstuk "LCNC binnen bedrijven". Omdat het belangrijk is dat de eindklant ook met deze platformen moet kunnen werken
zal er in dit hoofdstuk gekeken worden naar vorige onderzoeken waarbij men spreekt over de eindklant. 
Meer specifiek naar de ervaringen van personen met weinig tot geen ervaring in het programmeren die deze platformen gebruiken.


% Tip: Begin elk hoofdstuk met een paragraaf inleiding die beschrijft hoe
% dit hoofdstuk past binnen het geheel van de bachelorproef. Geef in het
% bijzonder aan wat de link is met het vorige en volgende hoofdstuk.

% Pas na deze inleidende paragraaf komt de eerste sectiehoofding.

Dit hoofdstuk bevat je literatuurstudie. De inhoud gaat verder op de inleiding, maar zal het onderwerp van de bachelorproef *diepgaand* uitspitten. De bedoeling is dat de lezer na lezing van dit hoofdstuk helemaal op de hoogte is van de huidige stand van zaken (state-of-the-art) in het onderzoeksdomein. Iemand die niet vertrouwd is met het onderwerp, weet nu voldoende om de rest van het verhaal te kunnen volgen, zonder dat die er nog andere informatie moet over opzoeken \autocite{Pollefliet2011}.

Je verwijst bij elke bewering die je doet, vakterm die je introduceert, enz.\ naar je bronnen. In \LaTeX{} kan dat met het commando \texttt{$\backslash${textcite\{\}}} of \texttt{$\backslash${autocite\{\}}}. Als argument van het commando geef je de ``sleutel'' van een ``record'' in een bibliografische databank in het Bib\LaTeX{}-formaat (een tekstbestand). Als je expliciet naar de auteur verwijst in de zin (narratieve referentie), gebruik je \texttt{$\backslash${}textcite\{\}}. Soms is de auteursnaam niet expliciet een onderdeel van de zin, dan gebruik je \texttt{$\backslash${}autocite\{\}} (referentie tussen haakjes). Dit gebruik je bv.~bij een citaat, of om in het bijschrift van een overgenomen afbeelding, broncode, tabel, enz. te verwijzen naar de bron. In de volgende paragraaf een voorbeeld van elk.

\textcite{Knuth1998} schreef een van de standaardwerken over sorteer- en zoekalgoritmen. Experten zijn het erover eens dat cloud computing een interessante opportuniteit vormen, zowel voor gebruikers als voor dienstverleners op vlak van informatietechnologie~\autocite{Creeger2009}.

Let er ook op: het \texttt{cite}-commando voor de punt, dus binnen de zin. Je verwijst meteen naar een bron in de eerste zin die erop gebaseerd is, dus niet pas op het einde van een paragraaf.

\lipsum[7-20]

%%=============================================================================
%% Methodologie
%%=============================================================================

\chapter{\IfLanguageName{dutch}{Methodologie}{Methodology}}%
\label{ch:methodologie}

%% TODO: In dit hoofstuk geef je een korte toelichting over hoe je te werk bent
%% gegaan. Verdeel je onderzoek in grote fasen, en licht in elke fase toe wat
%% de doelstelling was, welke deliverables daar uit gekomen zijn, en welke
%% onderzoeksmethoden je daarbij toegepast hebt. Verantwoord waarom je
%% op deze manier te werk gegaan bent.
%% 
%% Voorbeelden van zulke fasen zijn: literatuurstudie, opstellen van een
%% requirements-analyse, opstellen long-list (bij vergelijkende studie),
%% selectie van geschikte tools (bij vergelijkende studie, "short-list"),
%% opzetten testopstelling/PoC, uitvoeren testen en verzamelen
%% van resultaten, analyse van resultaten, ...
%%
%% !!!!! LET OP !!!!!
%%
%% Het is uitdrukkelijk NIET de bedoeling dat je het grootste deel van de corpus
%% van je bachelorproef in dit hoofstuk verwerkt! Dit hoofdstuk is eerder een
%% kort overzicht van je plan van aanpak.
%%
%% Maak voor elke fase (behalve het literatuuronderzoek) een NIEUW HOOFDSTUK aan
%% en geef het een gepaste titel.

\section*{Requirements analyse}
\label{sec:requirements-analyse}
Voor de onderzoeksvraag te beantwoorden is het belangrijk om de requirements te analyseren. De opgestelde requirements zullen de basis vormen voor 
de evaluatie, de vergelijkende analyse en de proof of concept. Deze vereisten zullen aan de hand van verschillende bronnen worden opgesteld. Als eerste zal 
er in de voorgaande literatuurstudie worden gezocht naar de belangrijkste vereisten. Vervolgens zal er een vraag worden gesteld die zal worden 
beantwoord door mijn co-promotor. De vraag bestaat uit, namelijk welke vereisten zijn volgens u belangrijk bij het kiezen van een Low-Code platform en rangshik deze vereisten van belangrijk naar minder belangrijk. Op deze
manier kan er een duidelijke beeld worden gevormd van waar er rekening mee moet worden gehouden bij het evalueren van de Low-Code en/of No-Code platformen.
\\
\\
In de literatuurstudie kwam er een aantal criteria's naar boven die impact kunnen hebben op de keuze van een Low-Code en/of No-Code platform.
Om hiervoor een eenvoudig overzicht te krijgen, zal er hieronder een tabel worden opgesteld met de volgende velden; Criteria, Reden/Relevantie, Bron. 
Ter info, de criteria's die in de tabel zullen worden opgenomen staan niet vast en zijn meer bedoelt als een leidraad voor de requirements analyse.
\\
\\


\begin{longtable}{lp{4.4cm}p{3.4cm}}
    \caption{Criteria lijst a.d.h.v. voorgaande literatuurstudie} \label{crouch} \\
    \toprule
    \textbf{Criteria} & \textbf{Reden/Relevantie} & \textbf{Bron} \\
    \midrule
    \endfirsthead

    % Repeat the headers on the next page
    \multicolumn{3}{c}{{\bfseries \tablename\ \thetable{} -- vervolg van de vorige pagina}} \\
    \toprule
    \textbf{Criteria} & \textbf{Reden/Relevantie} & \textbf{Bron} \\
    \midrule
    \endhead

    % Footer at the end of each page, except the last page
    \midrule
    \multicolumn{3}{r}{{Vervolg op volgende pagina}} \\
    \endfoot

    % Footer at the end of the table
    \bottomrule
    \endlastfoot

    % Table content
    Snelheid van het platform & Om snel en eenvoudig ingewikkelde applicaties te ontwikkelen. & Voordelen: snelheid \\
    Snelheid van applicatieontwikkeling & Zorgt dat er meerdere applicaties in een korte tijd kunnen worden ontwikkeld. Vervolgens moeten bedrijven ook steeds snel en flexibel kunnen inspelen op de veranderende markt. &  \hyperref[subsec:snelheid]{Voordelen: snelheid} \& \hyperref[sec:reden-tot-gebruik]{Reden tot gebruik: Tijdrovend} \\
    Leercurve & Heel wat mensen laten zich afschrikken tot het programmeren door de complexiteit van de programmeertalen. Hierdoor is het belangrijk dat het snel en eenvoudig is om te leren. & \hyperref[sec:reden-tot-gebruik]{Reden tot gebruik: Beperkt aantal programmeurs} \\
    Updatebeleid \& Moderniteit & Omdat technologie vaak verandert is de impact van een updatebeleid en moderniteit van het platform belangrijk op de lange termijn. & \hyperref[sec:reden-tot-gebruik]{Reden tot gebruik: Technologische turbulentie} \\
    Kostprijs & Omdat Quivvy Solutions BV een start-up is, zijn er beperkte financiële middelen. Waardoor het belangrijk is dat de kostprijs niet over het budget gaat. & \hyperref[sec:reden-tot-gebruik]{Reden tot gebruik: Hoge kosten} \& \hyperref[sec:lcnc-bedrijven]{LCNC binnen bedrijven} \\
    Klantentevredenheid & Als klein bedrijf is het belangrijk dat de klanten tevreden zijn over de applicaties die worden ontwikkeld. Daarbij moet de eindklant van het bedrijf ook zelf kunnen werken met het LCNC platform, waardoor de klantentevredenheid ook een belangrijk criterium is. &  \hyperref[sec:reden-tot-gebruik]{Reden tot gebruik: Klantentevredenheid} \\
    Veiligheid & Om ervoor te zorgen dat zowel het bedrijf Quivvy Solutions BV als hun eindklant zo weinig mogelijk schade kan krijgen door het gebruikte platform. &  \hyperref[subsec:veiligheid]{Voordelen: Veiligheid} \\
    Herstelbeheer \& Back-up & Wanneer er zich een ramp afspeelt moet de gegevens zo snel mogelijk kunnen hersteld worden. Hierdoor is het regelmatig back-ups relevant. & \hyperref[subsec:cloud-based-lcnc]{Cloud-based LCNC : Dataherstel} \\
    Integratie mogelijkheden & Hoe meer Integratie mogelijkheden dat het LCNC platform heeft, hoe beter. Flexibiliteit in een platform heeft een impact op de keuze van het platform. & \hyperref[subsec:beperkte-flexibiliteit]{Nadelen: Beperkte Flexibiliteit} \\
    Platformflexibiliteit \& Aanpasbaarheid & Mogelijkheid om high code te implementeren wanneer nodig en de mogelijke functionaliteiten. & \hyperref[subsec:beperkte-flexibiliteit]{Nadelen: Beperkte Flexibiliteit} \\
    Schaalbaarheid & Een applicatie in LCNC platformen zou gemakkelijk uitbreidbaar moeten zijn en wanneer mogelijk meer tegelijk gebruikers toelaten. & \hyperref[subsec:gelimiteerde-schaalbaarheid]{Nadelen: Gelimiteerde schaalbaarheid} \\
    Documentatiekwaliteit &  Documentatie kan leiden tot een oplossing voor het beter verstaan van bijvoorbeeld het opslaan van data. & \hyperref[subsec:lcnc-binnen-agile]{LCNC in Software Development Life Cycle: Application Design} \\
    Test- en debugmogelijkheden  & LCNC platformen verwaarlozen vaak testen, maar dit blijft nogsteeds een belangrijke fase in de Software Development Life Cycle.  &  \hyperref[subsec:lcnc-binnen-agile]{LCNC in Software Development Life Cycle: Testing}\\
    \\\endline
\end{longtable}
Bij het antwoord van de copromoter kwam er ook andere criteria's naar boven. Deze bestaan uit de volgende criteria's;
gebruiksvriendelijkheid, integratie met AirTable, en integratie met MAKE.com . Vervolgens heb ik samen met mijn copromoter overlegd welke criteria's de belangrijkste zijn en welke minder belangrijk zijn.
Hieruit volgt een lijst van requirements met een score op 10, waarbij 10 het belangrijkste is en 1 het minst belangrijk.

\begin{table}[H]
    \centering
    \caption{Requirements score}
    \begin{tabular}{llc}
    \toprule
    Criteria & Score \\
    \midrule
    Snelheid van het platform & 10 \\
    Herstelbeheer & 10 \\
    Veiligheid & 9 \\
    Snelheid van applicatieontwikkeling & 8 \\
    Integratie mogelijkheden & 8 \\
    Platformflexibiliteit (bv. high-code kunnen implementeren) & 8 \\
    Schaalbaarheid & 8 \\
    Gebruiksvriendelijkheid & 8 \\
    Integratie met AirTable en/of MAKE.com & 8 \\
    Kostprijs & 7.5 \\
    Updatebeleid & 7.5 \\
    Klantentevredenheid & 7.5 \\
    Test- en debugmogelijkheden & 7.5 \\
    Leercurve & 6 \\
    Documentatiekwaliteit & 6 \\
   
    \bottomrule
 \end{tabular}
\end{table}

\subsubsection*{Officiële requirements}
De officiële requirements bevat de criteria's die een score hebben van 8 of hoger, met uitzondering van kostprijs en updatebeleid. De reden hiervoor is omdat de kostprijs toch wel belangrijk
is voor een start-up en het updatebeleid is belangrijk voor de lange termijn. Hieruit volgt dat de klantentevredenheid, test- en debugmogelijkheden, documentatiekwaliteit en leercurve niet in de officiële requirements zullen worden opgenomen.
Een mogelijke reden waarom leercurve en documentatiekwaliteit niet belangrijk is omdat er hedendaags veel bronnen zijn die kunnen helpen bij het leren van een platform en de documentatiekwaliteit is vaak goed bij de meeste platformen, zo niet dan zijn er 
toturials die meestal kunnen helpen bij een bepaalde probleem.


%% TODO: In dit hoofstuk geef je een korte toelichting over hoe je te werk bent
%% gegaan. Verdeel je onderzoek in grote fasen, en licht in elke fase toe wat
%% de doelstelling was, welke deliverables daar uit gekomen zijn, en welke
%% onderzoeksmethoden je daarbij toegepast hebt. Verantwoord waarom je
%% op deze manier te werk gegaan bent.

\section*{Evaluatie en Selectie van alternatieven}
\label{sec:evaluatie-en-selectie-van-alternatieven}
Bij deze fase zal er onderzoek gedaan worden naar verschillende Low-Code en/of No-Code platformen. Hierbij zal 
er een online onderzoek gebeuren waarbij men bronnen zal gebruiken zoals officiële websites, reviews, en technologieblogs.
Vervolgens zal er per platform de voordelen, nadelen en beschrijvende informatie worden opgenomen. Daarna komt de belangrijkste
eigenschappen van een platform terecht in een tabel waarbij de criteria's van de requirements analyse zullen worden opgenomen. Hieruit volgt een
selectie van een alternatief die verder zal worden geanalyseerd.
\subsection*{Alternatieve platformen}
\label{subsec:alternatieve-platformen}

\subsubsection*{Zoho Creator}
Zoho Creator is een Low-Code platform, wat ook een gemakkelijke platform is om te gebruiken, dat werknemers van bedrijven maar ook mensen zonder programmeerervaring toelaat
om eenvoudig krachtige bedrijf applicaties te ontwikkelen \autocite{Computer2022}. Dit platform is gemaakt door Zoho Corporation. \textcite{ZohoCorporation2024a} gelooft er in dat software de ultieme tool is voor zowel de handen als de hersenen.
Zoho Corporation is een bedrijf dat vooral focust op het verder ontwikkelen van hun product, zoals Zoho Creator, en hun klanten support \autocite{ZohoCorporation2024a}. Dit toont aan dat Zoho Corporation
hun updatebeleid volgens hun website goed is. Volgens \textcite{ZohoCorporation2024a} biedt het bedrijf meer dan alleen een product, maar zoals het bedrijf het noemt 'the operating system for business' \autocite{ZohoCorporation2024a}.
Dit bevat maar liefst 55 integreerde applicaties voor zowel mobile als web voor elke bedrijfsnood \autocite{ZohoCorporation2024a}.

\paragraph{Functies van Zoho Creator}
In 2022 was er nog geen enkele Low-Code platform die toeliet om business gebruikers en IT een end to end business oplossingen te maken \autocite{Computer2022}.
De Zoho Creator platform zorgt ervoor dat je applicatie development, business intelligence en analytics, integraties, en automatiseringen kan doen in één platform, wat men ook een end to end business oplossing noemen \autocite{Computer2022}.
Dit geeft verschillende voordelen voor het bedrijf dat dit platform gebruikt zoals het verhogen van veiligheid maar ook omdat het platform een end to end business oplossing is, is het ook mogelijk om uniforme oplossingen te maken via de low-code platform,
waardoor elke werknemer de mogelijkheid krijgt om te innoveren \autocite{Computer2022}. Volgens \textcite{Computer2022} zorgt Zoho Creator er ook voor dat de business gebruikers snel een schaalbare low-code oplossingen kunnen maken zoals het maken van een app of automatisering.
Het ontwikkelen van een low-code oplossing is zelfs 10 keer sneller dan eender welke andere oplossing op de markt \autocite{Computer2022}. Het platform heeft ook hun eigen AI, genaamd Zia, die de gebruiker helpt bij het maken van applicaties \autocite{Computer2022}. Deze AI
kan ontwikkelaars helpen bij het importeren van data door deze te analyseren en te structureren \autocite{Computer2022}. Daarnaast kan het ook relaties tussen data detecteren \autocite{Computer2022}. 
Zoho Creator heeft de mogelijkheid om zelf code te schrijven, wat een groot voordeel is voor bedrijven die toch nog high-code willen implementeren \autocite{Computer2022}. 
De code kan geschreven worden in Deluge, Java of Node.js en zijn ook herbruikbaar om duplicatie te voorkomen \autocite{Computer2022}.
De gebruikers van het platform Zoho Creator kunnen ook zien hoe goed integraties aan het werken zijn aan de hand van de 'Integration Status Dashboard', wat helpt om te anlayseren waar er problemen zijn \autocite{Computer2022}.
\\ %TODO (nu enkel info van Express Computer, moet nog andere bronnen zoeken)
\\
Volgens het bedrijf \autocite{ZohoCorporation2024b} zorgt het platform ervoor dat 90\% van de complexiteit van het ontwikkelen van applicaties wordt weggenomen. Dit is zo dat de gebruikers van het platform
kunnen focussen op de functies, businesswaarde en de eindklant \autocite{ZohoCorporation2024b}.  Met Zoho Creator kan je heel wat verschillende apps maken zoals volledig jouw eigen app, mobile apps, en een online portal \autocite{ZohoCorporation2024b}.
Maar Zoho Creator kan je voor meer dan app development gebruiken, je kan businessprocessen automatiseren, BI \& Analytics, en integraties maken met andere systemen \autocite{ZohoCorporation2024b}. Daarnaast zorgt het ook voor een veilige omgeving door 
jouw geen zorgen te maken over middleware, authenticatie, en business transacties want Zoho Creator hebben hier namelijk services voor \autocite{ZohoCorporation2024b}. Vervolgens hebben Zoho Creators verschillende functies zoals het opslaan van data,
betalingstransacties, het updaten van je CRM, en emails en rapporten sturen \autocite{ZohoCorporation2024b}.

\paragraph{Integratie mogelijkheden}
Zoho Creator heeft een lijst met maar liefst 600+ Apps waarmee het kan integreren \autocite{ZohoCorporation2024b}. Ze hebben integraties voor Sales automation, IT Security en Team Collaboration \autocite{ZohoCorporation2024b}.
Helaas heeft Zoho Creator nog geen integratie met AirTable, maar heeft wel integratie met MAKE.com \autocite{MAKE.com2024}.

\paragraph{Beoordelingen}
Op \textcite{Gartner2024} heeft Zoho Creator een beoordeling van 4.6/5.0, gebaseerd op 205 beoordelingen. Hieruit volgt dat 38\% van bedrijven met minder dan 50 miljoen dollar waarde Zoho Creator aanbevelen, 
en 37\% van bedrijven met meer dan 50 miljoen dollar waarde tot 1 miljard Zoho Creator aanbevelen \autocite{Gartner2024}. Vervolgens waren de beoordelingen, met 4.3/5, op business logica en workflow, integratie met API's, en platformflexibiliteit \autocite{Gartner2024}.
\paragraph{Kostprijs}
%per abonnement; herstelbeheer, integratie, automatiseringen (ai)
Zoho Corporation heeft verschillende soorten abonnementen voor hun platform, deze zijn Standard, Professional, Enteprise, en Flex \autocite{ZohoCorporation2024}.
De standaard abonnementen bedraagt 12 euro, per gebruiker, per maand. Hiermee kan je maar maximaal 1 business applicatie maken \autocite{ZohoCorporation2024}.
Daarbij wordt ook je data opgeslagen en geback-upt in de cloud en 20 AI models maken om te trainen voor specifieke taken \autocite{ZohoCorporation2024}. Op vlak van integratie
kan je maar maximum 5 data bronnen integreren en 5 eigen connectie maken met systemen dat niet ondersteund worden door Zoho Creator \autocite{ZohoCorporation2024}. Vervolgens heb je dan het
abonnement Professional, wat 30 euro per gebruiker, per maand bedraagt. Hiermee kan je unlimited business applicaties maken en 100 AI models trainen \autocite{ZohoCorporation2024}. 
Daarbij kan je 15 data bronnen integreren en 10 eigen connecties maken \autocite{ZohoCorporation2024}. Daarnaast heb je dan Enteprise abonnement, wat 37 euro per gebruiker, per maand bedraagt.
Het voordeel van dit abonnement is dat je integratie mogelijkheid hebt met 650+ business apps, 30 data bronnen integreren, en 20 eigen connecties maken \autocite{ZohoCorporation2024}. Vervolgens heb je ook
Business Intelligence and Analytics bij Enteprise om zeer eenvoudig en verschillende anlaysis te maken \autocite{ZohoCorporation2024}. Het handige is dat je ook de mogelijkheid hebt om hun te contacteren in het geval 
van gepersonaliseerde requirements via het Flex abonnement \autocite{ZohoCorporation2024}.

\subsubsection*{OutSystems}
OutSystems is een Low-Code platform dat developers pixel-perfect applicaties laat maken aan de hand van een drag-and-drop functionaliteiten \autocite{Ranosys2023} \autocite{Payne2023}.
OutSystems maakt gebruik van AI, cloud technologie, DevOps, en visuele ontwikkeling om citizen developers deel te laten nemen applicatie ontwikkeling \autocite{Ranosys2023}.

\paragraph{Functies}
Deze Low-code platform heeft een unieke visuele omgeving dat het ontwinkkelingsproces versnelt en de complexiteit van het ontwikkelen van applicaties vermindert \autocite{Payne2023}.
Hierdoor gaat de productiviteit naar omhoog, vermindert de leercurve, en vermindert het aantal errors and bugs \autocite{Payne2023}. OutSystems heeft ook heel wat built-in templates 
en componenten die het ontwikkelen van applicaties versnelt en vermindert de eigen gemaakt code \autocite{Payne2023}.
Vervolgens laat het ook toe om eigen gebouwde componenten te gebruiken voor consistentie en schaalbaarheid \autocite{Ranosys2023}.
OutSystems heeft ook een aantal Agile Software Development functies zoals version control, programmer collaboration en geautomatiseerde testing tools \autocite{Ranosys2023}.
Daarnaast geeft het uw volledig controle over veilige hosting door het genereren van op standaarden gebaseerde, niet-eigen code, hoogwaardige digitale oplossingen voor bedrijven \autocite{Ranosys2023}.
Het Low-Code platform heeft ook een AI om bijvoorbeeld errors te vermideren, om pre-built componenten te genereren of zoeken,en om code te analyseren en detecteren van bugs \autocite{Ranosys2023}.
Het bedrijf OutSystems laat ook bedrijven toe om de prestaties en groei van uw bedrijf visueel te monitoren en te beheren \autocite{Ranosys2023}.

\paragraph{Integratie mogelijkheden}
OutSystems bevat heel wat integratie mogelijkheden zoals het connecteren met externe data bronnen, enterprise systemen, en API's \autocite{Payne2023}.
Hiervoor biedt het Low-Code platform connecties en tools om het integratieproces te vergemakkelijken en tijd te besparen \autocite{Payne2023}. Helaas
is er geen informatie gevonden over integratie met AirTable en MAKE.com.

\paragraph{Beoordelingen}
Volgens \textcite{Gartner2024} heeft OutSystems een beoordeling van 4.5/5.0, gebaseerd op 885 beoordelingen. 
Hieruit volgt dat 22\% van bedrijven met minder dan 50 miljoen dollar waarde OutSystems aanbevelen en dat 44\% van bedrijven met meer dan 50 miljoen dollar waarde tot 1 miljard OutSystems aanbevelen \autocite{Gartner2024}.
Vervolgens is de beoordeling op schaalbaarheid (4.3/5), aanpasbaarheid (4.3/5), en een goede platform voor overheid (4.2/5) de laagste beoordelingen \autocite{Gartner2024}.

\paragraph{Kostprijs}
OutSystems heeft drie verschillende abonnementen, namelijk Free, Multiple apps, en Large app portfolios \autocite{OutSystems}. Het Free abonnement is gratis en is bedoelt voor het maken van één app \autocite{OutSystems}.
Bij Free plan kan enkel je app gerund worden in development en niet in production \autocite{OutSystems}. Vervolgens kunnen er ook maar 100 eindgebruikers op de app zitten \autocite{OutSystems}.
Het Multiple apps abonnement heeft heel wat meer voordelen, maar wel tegen een prijs van 1.250 euro per maand. Deze voordelen zijn dat het zowel in development als in production modus kan gerund worden \autocite{OutSystems}.
Daarnaast heb je een uptime van 99.5\% en is er geen maximum aantal eindgebruikers \autocite{OutSystems}. Het laatste abonnement, Large app portfolio, is bedoelt voor bedrijven met meer nodige functies dan gegeven bij Multiple apps \autocite{OutSystems}. 
\subsubsection*{Microsoft PowerApps}
\paragraph{Functies}
\paragraph{Integratie mogelijkheden}
\paragraph{Beoordelingen}
\paragraph{Kostprijs}
\subsubsection*{Appian}
\paragraph{Functies}
\paragraph{Integratie mogelijkheden}
\paragraph{Beoordelingen}
\paragraph{Kostprijs}
\subsubsection*{Wix}
\paragraph{Functies}
\paragraph{Integratie mogelijkheden}
\paragraph{Beoordelingen}
\paragraph{Kostprijs}
\subsubsection*{Bubble}
\paragraph{Functies}
\paragraph{Integratie mogelijkheden}
\paragraph{Beoordelingen}
\paragraph{Kostprijs}


\section*{Vergelijkende analyse}
\label{sec:vergelijkende-analyse}
In deze fase vergelijken we de alternatieve platform met Softr en Stacker aan de hand van verschillende testen. 
Deze soort testen hangen af van de requirements analyse.


\section*{Proof of Concept}
\label{sec:proof-of-concept}
\subsection*{Ontwikkelingsfase}
\subsection*{Uitvoer van de Proof of Concept}
\subsection*{Resultaten van de Proof of Concept}

\section*{Conclusie}



% Voeg hier je eigen hoofdstukken toe die de ``corpus'' van je bachelorproef
% vormen. De structuur en titels hangen af van je eigen onderzoek. Je kan bv.
% elke fase in je onderzoek in een apart hoofdstuk bespreken.

%\input{...}
%\input{...}
%...

%%=============================================================================
%% Conclusie
%%=============================================================================

\chapter{Conclusie}%
\label{ch:conclusie}

% TODO: Trek een duidelijke conclusie, in de vorm van een antwoord op de
% onderzoeksvra(a)g(en). Wat was jouw bijdrage aan het onderzoeksdomein en
% hoe biedt dit meerwaarde aan het vakgebied/doelgroep? 
% Reflecteer kritisch over het resultaat. In Engelse teksten wordt deze sectie
% ``Discussion'' genoemd. Had je deze uitkomst verwacht? Zijn er zaken die nog
% niet duidelijk zijn?
% Heeft het onderzoek geleid tot nieuwe vragen die uitnodigen tot verder 
%onderzoek?
Als men terug blikt op de vraag die gesteld werd in de inleiding: welk platform het meest geschikt is voor Quivvy Solutions, had men verwacht dat het Softr zou zijn. Maar men kan concluderen dat Bubble beter zou passen bij het bedrijf. 
Deze conclusie heeft men kunnen trekken door de zeer uitgebreide methodologie; alternatieven, vergelijkende analyse, en Proof of Concept. De reden dat Bubble aangeraden wordt, is omdat het voor Quivvy Solutions noodzakelijk is dat het platform
kan integreren met zowel AirTable als MAKE.com. Bubble heeft deze mogelijkheden en heeft ook een groot aantal plugins. Daarnaast is Bubble ook superieur bij het gebruik van diverse databases en platformflexibiliteit.
Vervolgens is Bubble stabieler op lange termijn en interessanter voor grote bedrijven. De prijs van Bubble is tegenover de twee andere platformen minder duur. Het platform is ook veel transparanter naar hun 
gebruikers toe op het vlak van updatebeleid. De enigste minpunten van Bubble is dat Softr en Stacker beter scoorde op snelheid van applicatieontwikkeling en gebruiksvriendelijkheid. Helaas weten we niet hoe goed elke platform is bij het maken
van complexe apps. Dit omdat we enkel een Proof of Concept hebben gemaakt waarin simpele applicaties werden gebouwd. We kunnen concluderen dat volgens de programmeur van de Proof of Concept Softr en Stacker minder geschikt zou zijn voor complexe apps. Hieruit volgt dat men verder kan onderzoeken 
hoe deze platformen presteren bij het maken van complexe apps. Ten slotte is het belangrijk om te vermelden dat de uitgevoerde Proof of Concept niet voldoende is om een definitieve beslissing te nemen.




%---------- Bijlagen -----------------------------------------------------------

\appendix

\chapter{Onderzoeksvoorstel}

Het onderwerp van deze bachelorproef is gebaseerd op een onderzoeksvoorstel dat vooraf werd beoordeeld door de promotor. Dat voorstel is opgenomen in deze bijlage.

%% TODO: 
%\section*{Samenvatting}

% Kopieer en plak hier de samenvatting (abstract) van je onderzoeksvoorstel.

% Verwijzing naar het bestand met de inhoud van het onderzoeksvoorstel
%---------- Inleiding ---------------------------------------------------------

\section{Introductie}%
\label{sec:introductie}

Bedrijven aarsen op effeciëntie, kosten  verminderen, en veiligheid. Dit is niets anders dan ook bij het softwarebedrijf genaamd Quivvy waarbij de trend No-Code en Low-Code
platforms voor het bouwen van applicaties de laatste jaren rond dwaalt. Momenteel gebruikt Quivvy FlutterFlow voor het bouwen van Mobile \& Web Portals. 
Hierbij interageert FlutterFlow met AirTable, een database platform. Maar bedrijven zijn voortdurend op zoek naar de beste softwareplatformen om hun bedrijf te runnen.
Daarom zoekt Quivvy naar een alternatief voor het bouwen van Mobile \& Web Portals dat zowel goed werkt voor bedrijf als eindklant, maar ook interageert met AirtTable of Podio.
De markt van No-Code en Low-Code platforms worden overspoeld met verschillende platformen waaronder Microsoft PowerApps, Bubble, Webflow, Softr en Stacker.
Als gevolg dat drie op de vier van de grootste bedrijven tegen 2024 gebruik zullen maken van Low-Code platforms ~\autocite{Moskal_2021} ~\autocite{Kulkarni_2021}.
Deze No-Code en Low-Code platforms hebben ook een reden tot de populariteit, namelijk dat
deze platforms het mogelijk maken om applicaties te bouwen zonder enige tot weinig kennis van programmeren. Dit is niet het enige waarom deze platforms een trend zijn,
want het zorgt ook voor een snelle ontwikkeling met lage kosten door de effeciënte gebruik van de ontwikkelaars.
Maar welke bestaande softwareplatform is geschikt voor Mobile \& Web portals te creëren, dat zowel goed werkt voor bedrijf als eindklant?
In deze bachelorproef hanteren we het meest geschikte Mobile \& Web Portals voor een softwarebedrijf en eindklant.
Waarbij we een vergelijking maken tussen Softr en Stacker, beide No-Code en/of Low-Code platforms. Daarbij wordt er rekening gehouden met verschillende 
factoren zoals snelheid, gebruiksvriendelijkheid, enzovoort.


%Waarover zal je bachelorproef gaan? Introduceer het thema en zorg dat volgende zaken zeker duidelijk aanwezig zijn:

% \begin{itemize}
%   \item kaderen thema
%   \item de doelgroep
%   \item de probleemstelling en (centrale) onderzoeksvraag
%   \item de onderzoeksdoelstelling
% \end{itemize}

% Denk er aan: een typische bachelorproef is \textit{toegepast onderzoek}, wat betekent dat je start vanuit een concrete probleemsituatie in bedrijfscontext, een \textbf{casus}. Het is belangrijk om je onderwerp goed af te bakenen: je gaat voor die \textit{ene specifieke probleemsituatie} op zoek naar een goede oplossing, op basis van de huidige kennis in het vakgebied.

% De doelgroep moet ook concreet en duidelijk zijn, dus geen algemene of vaag gedefinieerde groepen zoals \emph{bedrijven}, \emph{developers}, \emph{Vlamingen}, enz. Je richt je in elk geval op it-professionals, een bachelorproef is geen populariserende tekst. Eén specifiek bedrijf (die te maken hebben met een concrete probleemsituatie) is dus beter dan \emph{bedrijven} in het algemeen.

% Formuleer duidelijk de onderzoeksvraag! De begeleiders lezen nog steeds te veel voorstellen waarin we geen onderzoeksvraag terugvinden.

% Schrijf ook iets over de doelstelling. Wat zie je als het concrete eindresultaat van je onderzoek, naast de uitgeschreven scriptie? Is het een proof-of-concept, een rapport met aanbevelingen, \ldots Met welk eindresultaat kan je je bachelorproef als een succes beschouwen?

%---------- Stand van zaken ---------------------------------------------------

\section{State-of-the-art}%
\label{sec:state-of-the-art}

 In de software wereld bevindt men in projecten dat er telkens over het budget wordt gegaan. Daarnaast doet het project meestal niet wat de klant verwacht 
 en vervolgens  doet het opgeleverde product niet wat het moet doen ~\autocite{Moskal_2021}. Volgens ~\textcite{Moskal_2021} zijn deze problemen niet alleen maar te zien in de software wereld, 
 maar ook in andere categorieën binnen de IT-sector, of bij het implementeren en ontwerpen van software systemen. 
 Dit brengt het onderwerp Low-Code en No-Code naar boven ~\autocite{Kulkarni_2021}. 
 Deze platforms zorgen ervoor dat u geen tot weinig kennis nodig heeft over programmeren waardoor deze wordt beschouwt als een trend waar heel wat mensen geïnteresseerd in is ~\autocite{Kulkarni_2021}.
 Maar in vergelijking met andere technologie trends zoals AI, Blockchain, Edge Computing, en RPA groeit LCNC zeer matig in vergelijking met de andere trends ~\autocite{Kulkarni_2021}.
 Toch stijgt het gebruik van No-Code platforms, maar waarom? Dit kan men onderverdelen in vier categorieën:
\begin{itemize}
  \item \textbf{Beperkt aantal programmeurs}: 
  Heel wat studenten laten zich afschrikken door het programmeren. 
  Hierdoor is er kort aantal studenten die werkelijk diepe kennis hebben op vlak van programmeren. 
  Maar het vinden van een zeer gekwalificeerde programmeur brengt projecten niet altijd tot een goed einde ~\autocite{Moskal_2021}.
  \item \textbf{Technologische turbulentie}: De constante evolutie van programmeertalen zorgt ervoor dat de kennis van programmeurs niet altijd de meest recente is  ~\autocite{Moskal_2021}.
  \item \textbf{Hoge kosten}: De traditionele softwareontwikkeling heeft een grote tol op de kosten van een bedrijf. Daarbij beseffen Software engineers dat het bouwen van een applicatie niet gemakklijk is binnen een budget ~\autocite{Moskal_2021}.
  \item \textbf{Tijdrovend}: De traditionele softwareontwikkeling is een tijdrovend proces. De geschatte tijd om Software te maken op één operating systeem is zes maanden of zelfs lange ~\autocite{Moskal_2021}.
  \item \textbf{Complexe softwareontwikkeling}
\end{itemize} 
Het gebruik van het woord No-Code development wordt gezien als een synoniem voor Low-Code development,
daarom kan me verwijzen naar low-code als no-code maar ook omgekeerd ~\autocite{Rokis_2022}. 

\subsection*{LCNC Uitdagingen}
\label{sub:lcnc-uitdagingen}
Om grondig door alle LCNC uitdagingen te gaan, zullen we ze op delen in bepaalde fasen
die gebeuren tijdens het ontwikkelen van een applicatie. 
Deze zijn; requirements analyse, planning, application design, development, testing, deployment, en maintenance ~\autocite{Rokis_2022}.
\subsubsection*{Requirements Analyse}
\label{sub:requirements-analyse}
Doordat specificaties verschilt per platform binnen software, kan dit als een uitdaging worden gezien ~\autocite{Rokis_2022}.
Hierdoor wordt een tool voor eisenbeheer binnen LCNC beschouwd als een waardevolle toevoeging ~\autocite{Rokis_2022}. 
In deze fase wordt veranderingen in de eisen ook gezien als een struikelblok. Maar door het gebruik van LCNC kan dit worden opgelost 
door het gebruik van een prototype, waarbij men steeds de volgende dag snel een oplossing kunnen leveren en onderhouden ~\autocite{Rokis_2022}.
\subsubsection*{Planning}
\label{sub:planning}
In de planning fase komen we terecht bij het kiezen van de meest compatibele LCNC platform die de eisen goed vastneemt ~\autocite{Rokis_2022}.
Volgens ~\textcite{Rokis_2022} maakt de overvloed van de beschikbare platforms op de markt het zeer moeilijk om de meest compatibele te vinden. De 
kosten, leercurve, ondersteuning van functies en functionaliteiten in ontwikkeling spelen allemaal een belangrijke rol ~\autocite{Rokis_2022}. Daarnaast zijn vele
bedrijven bezorgd over "vendor-lock-in". De klant is daarbij sterk afhankelijk van een bepaalde levarancier, waardoor het moeilijk is om over te stappen ~\autocite{Rokis_2022} ~\autocite{Yan2021}.
\subsubsection*{Application Design}
\label{sub:application-design}
De applicatie ontwerp is redelijk gelimmiteerd binnen LCNC.
Ten eerste hebben we te maken met het probleem dat de meeste LCNC platform niet uitbreidbaar zijn ~\autocite{Rokis_2022}.
Als tweede zijn er overwegingen over de betrekking tot schaalbaarheid ~\autocite{Rokis_2022}. 
Vervolgens kan er bepereking zijn op het vlak gegevens opslaan en het ontwerp van de gebruikersinterface ~\autocite{Rokis_2022}.
\subsubsection*{Development}
\label{sub:development}
Bij het ontwikkelen van de software kunnen verschillende uitdagingen voorkomen.
Volgens ~\textcite{Rokis_2022} kunnen in sommige gevallen kan de gelimmiteerde functionaliteit van LCNC platformen een probleem vormen.
Hierdoor moet men dan zelf code schrijven waardoor men meer tijd moeten spenderen en complexiteit toevoegen aan het project ~\autocite{Rokis_2022}.
Daarnaast zouden we ook rekening moeten houden met de moeilijk op het vlak debuggen in een grafische representatie ~\autocite{Rokis_2022}.
\subsubsection*{Maintenance}
\label{sub:maintenance}
Een voordeel van LCNC is dat het onderhoud van de applicatie zeer weinig vraagt ~\autocite{Rokis_2022}.
Volgens ~\textcite{Rokis_2022} komen we hier weer terecht met de uitdaging van debuggen in een grafische representatie.

\subsection*{LCNC binnen bedrijven}
\label{sub:lcnc-binnen-bedrijven}
Verschillende soorten organisaties hangen af van software zodat de organisaties kunnen functioneren ~\autocite{Hintsch2021}.
Volgens ~\textcite{Rafiq_2022} is dit ook het geval bij software start-ups en midden tot grote bedrijven. 
Software start-ups zijn jonge bedrijven met een gelimmiteerd aantal middelen ~\autocite{Rafiq_2022}. 
Deze hebben dan ook enkele uitdagingen dat ze moeten weerstaan waaronder tijdsdruk, team formatie en snel groeiende markten ~\autocite{Rafiq_2022}.
Hiervoor maken ze gebruik van Low-Code en No-Code applicatie development platformen ~\autocite{Rafiq_2022}. 
Maar waarom maken ze gebruik van LCNC? en Wat is het verschil tegenover midden tot grote bedrijven?
~\textcite{Rafiq_2022} heeft deze vragen beantwoord door een onderzoek te voeren bij twee bedrijven, een software start-up en een groot bedrijf met meerde kantoren.
Hierbij kwam naar voor dat de software start-up zowel gebruik maakt van Low-Code en No-Code platformen, terwijl het grote bedrijf enkel gebruik maakt van Low-Code platformen ~\textcite{Rafiq_2022}.
De combinatie van LCNC development zorgt voor verschillende doelen dat bereikt kan worden ~\autocite{Rafiq_2022}.
Ten eerste voor het maken van een prototype, daarnaast bij het ontwerpen, en als laatste bij het uitvoeren van de dienst ~\autocite{Rafiq_2022}.
Maar in de software start-up wordt dit niet gebruikt voor het hoofd product ~\autocite{Rafiq_2022}. In het grote bedrijf wordt Low-Code
gebruikt door de snelle ontwikkeling van applicatie, snelle feedback, en de mindere werklast ~\autocite{Rafiq_2022}.
\subsection*{Gebruik van LCNC door eindklanten}
\label{sub:gebruik-van-lcnc-door-eindklanten}
In de voorgaande literatuur werd aangehaald dat LCNC gebruikt zou kunnen worden door eindklanten. Dit kan ook nog eens bevestigd
worden door ~\textcite{Yan2021} waarbij men vertelt dat LCNC kan gebruikt worden door gebruikers door visuele maken van applicaties zonder enige tot weinig kennis van programmeren. 
Volgens ~\textcite{Hintsch2021} is er in 2020 een studie gedaan over hoe eindgebruikers tegenover ervaren ontwikkelaars performeren bij het gebruik van Low-Code Development.
Hier uit bleek dat ervaren ontwikkelaars moeilijkheden hadden bij het identificren van belangrijke concepten binnen software engineering ~\autocite{Hintsch2021}. Al 
hoewel ontwikkelaars problemen ervaarden was dit ook het geval bij eindgebruikers ~\autocite{Hintsch2021}. Deze hadden een probleem met meer envoudige taken zoals het maken van een scherm,
de connectie met de database, en "paramer passing" ~\autocite{Hintsch2021}. Dit leidt tot de vraag of eindgebruikers wel in staat zijn om applicaties te maken met Softr of Stacker.
\subsection*{LCNC Voordelen}
\label{sub:lcnc-voordelen}
\subsubsection*{Snelheid}
\label{sub:snelheid}
De mindere werklast door de snelle otnwikkeling van applicaties is een groot voordeel van LCNC ~\autocite{Adrian_2020}.
Heel wat bedrijven dat gebruikmaken van Low-Code platformen stelde vast dat hun release van de applicatie sneller was dan voorheen, bij 5 van de 10 keer ~\autocite{Yan2021}.
Volgens ~\textcite{Yan2021} is er ook een enquête van OutSystems in 2019 bleek dat gebruikers van Low-Code plaftormen 68\% van hun webapplicaties en 64\% van hun apps elk konden bouwen in vier maand.
Dit was niet het geval bij traditionele ontwikkeling waarbij 57\% van de webapplicaties en 49\% van de apps elk werden gebouwd in vier maand ~\autocite{Yan2021}.

\subsubsection*{Veiligheid}
\label{sub:veiligheid}
Door het gebrek aan mensen in de IT-sector, die een massa aan software moeten ontwikkelen, moeten de mensen buiten 
de IT-sector gebruikmaken van third-party software ~\autocite{Yan2021}. Dit kan schade brengen op het bedrijf omdat ze niet op de hoogte is 
van de licentie en veiligheid, dit noemen we ook wel "Shadow IT" ~\autocite{Yan2021}. Daarom zorgt LCNC, die geautoriseerd is door de IT-sector, enigszins tot veiligheid omdat het
de risico op "Shadow IT" vermindert ~\autocite{Yan2021}. Daarnaast zorgt LCNC ook ervoor dat de IT-personeel niet telkens wordt verstoord door de mensen buiten de IT-sector ~\autocite{Yan2021}.
Met deze Low-Code en No-Code platformen kunnen de mensen makkelijk oplossing ontwikkelen ~\autocite{Yan2021}. Voor de IT-sector is dit ook handig want Low-Code en No-Code platformen bevat 
de internationale standaarden voor veiligheid (ISO/IEC 27001, PCIDSS) ~\autocite{Sufi_2023}. Hedendaags wordt er ook volgens ~\textcite{Sufi_2023} een principe genaamd "Security by Design" toegepast.
Deze principe neemt heel wat zorgen af op het vlak van de veiligheid in de IT voor de Citizen Developers,  ook wel mensen buiten de IT-sector genoemd ~\autocite{Sufi_2023}.

\subsubsection*{Verstaanbaar}
\label{sub:Verstaanbaar}
Doordat moderne Low-Code en No-Code platformen gebruik maken van visuele representatie dat ondersteund wordt door drag-and-drop, is het makkelijk te begrijpen voor de gebruikers ~\autocite{Sufi_2023}.
Hierdoor kunnen Citizen Developers en eindklanten makkelijk zeer complexe applicaties maken zonder enige tot weinig kennis van programmeren ~\autocite{Sufi_2023}.

\subsubsection*{Cloud Forward Approach}
\label{sub:cloud-forward-approach}
Hedendaags beginnen heel wat bedrijven te migreren naar de cloud technologie ~\autocite{Sufi_2023}.
Dit komt omdat de cloud technologie heel wat voordelen biedt zoals schaalbaarheid, flexibiliteit, enzovoort ~\autocite{Sufi_2023}.
Gelukkig zijn de meeste LCNC platformen cloud gebaseerd ~\autocite{Sufi_2023}. 
Hierdoor biedt LCNC snelle strategieën voor het migreren naar de cloud, aan moderne bedrijven ~\autocite{Sufi_2023}.




% Hier beschrijf je de \emph{state-of-the-art} rondom je gekozen onderzoeksdomein, d.w.z.\ een inleidende, doorlopende tekst over het onderzoeksdomein van je bachelorproef. Je steunt daarbij heel sterk op de professionele \emph{vakliteratuur}, en niet zozeer op populariserende teksten voor een breed publiek. Wat is de huidige stand van zaken in dit domein, en wat zijn nog eventuele open vragen (die misschien de aanleiding waren tot je onderzoeksvraag!)?

% Je mag de titel van deze sectie ook aanpassen (literatuurstudie, stand van zaken, enz.). Zijn er al gelijkaardige onderzoeken gevoerd? Wat concluderen ze? Wat is het verschil met jouw onderzoek?

% Verwijs bij elke introductie van een term of bewering over het domein naar de vakliteratuur, bijvoorbeeld~\autocite{Hykes2013}! Denk zeker goed na welke werken je refereert en waarom.

% Draag zorg voor correcte literatuurverwijzingen! Een bronvermelding hoort thuis \emph{binnen} de zin waar je je op die bron baseert, dus niet er buiten! Maak meteen een verwijzing als je gebruik maakt van een bron. Doe dit dus \emph{niet} aan het einde van een lange paragraaf. Baseer nooit teveel aansluitende tekst op eenzelfde bron.

% Als je informatie over bronnen verzamelt in JabRef, zorg er dan voor dat alle nodige info aanwezig is om de bron terug te vinden (zoals uitvoerig besproken in de lessen Research Methods).

% % Voor literatuurverwijzingen zijn er twee belangrijke commando's:
% % \autocite{KEY} => (Auteur, jaartal) Gebruik dit als de naam van de auteur
% %   geen onderdeel is van de zin.
% % \textcite{KEY} => Auteur (jaartal)  Gebruik dit als de auteursnaam wel een
% %   functie heeft in de zin (bv. ``Uit onderzoek door Doll & Hill (1954) bleek
% %   ...'')

% Je mag deze sectie nog verder onderverdelen in subsecties als dit de structuur van de tekst kan verduidelijken.

%---------- Methodologie ------------------------------------------------------
\section{Methodologie}%
\label{sec:methodologie}
Er zal een vergelijkende studie gebeuren tussen twee No-Code en/of Low-Code platforms, namelijk Softr en Stacker.
Deze twee platformen bieden het mogelijk om zowel als bedrijf of als een eindklant een Web \& Mobile Portal te maken.
Hierbij moet er ook rekening gehouden worden met de integratie van AirTable of Podio, een database platform. 
Vervolgens zal deze vergelijkende studie ons raadplegen over welk platform, als alternatief voor Quivvy, het meest geschikt is voor het bouwen van Web \& Mobile Portals,
dat zowel goed werkt voor het bedrijf Quivvy als eindklant. Voor de vergelijkende studie zal er opgesplitst worden in verschillende fasen namelijk,
requirements analyse, alternatieven, interessante alternatieven, proof-of-concept, en conclusies.
\subsection*{Requirements Analyse}
\label{sub:requirements-analyse}
Aan de hand van de co-promotor wordt de criteria opgelijst waaran de No-Code en/of Low-Code platformen moeten voldoen.
Daarnaast zal ook de co-promotor een lijst geven op welke categorieën de No-Code en/of Low-Code platformen zullen vergeleken worden. 
Vervolgens zal er ook nog een grondige literatuurstudie gebeuren over andere LCNC platformen waarbij onderzoek al gedaan is op verschillende
criteria's. Deze fase zal ongeveer 2 weken duren. Als resultaat zal er een lijst met requirements zijn geordend op prioriteit.
De volgende criteria's zullen onderzocht worden:
\begin{itemize}
  \item Snelheid van het platform
  \item Gebruiksvriendelijkheid van het platform
  \item Integratie met AirTable of Podio
  \item Capaciteit van het platform
  \item Veiligheid
  \item functionaliteiten binnen het platform
\end{itemize}

\subsection*{Alternatieven}
\label{sub:alternatieven}
In deze fase zullen we een grondige onderzoek doen naar andere No-Code en/of Low-Code platformen die voldoen aan de requirements.
Hierbij zullen we ook rekening houden met de criteria's die opgesteld zijn door de co-promotor. Deze fase zal ongeveer 1 week duren. Tenslotte 
zal dit een lijst opleveren met verschillende alternatieven.

\subsection*{Interessante Alternatieven}
\label{sub:interessante-alternatieven}
Hier zal er gekeken worden naar de verschillende alternatieven en zal er een selectie gemaakt worden van de meest interessante.
Uit deze selectie zal er dan 1 No-Code en/of Low-Code platform gekozen worden. Om verder onderzoek over te doen
Deze fase zal ongeveer 2 weken duren.

\subsection*{Proof-Of-Concept}
\label{sub:proof-of-concept}
Om een goed beeld te krijgen van het No-Code en/of Low-Code platform Softr en Stacker zal er een proof-of-concept gemaakt worden.
Hiervoor moet men Softr en Stacker installeren. 
Deze LCNC platformen vereisen deze minimum-requirements voor de installatie:
\begin{itemize}
  \item Besturingssysteem: Windows, macOS, of Linux wordt ondersteund.
  \item Processor (CPU): Een dual-core processor of hoger is vereist.
  \item Geheugen (RAM): Minimaal 4 GB RAM of meer wordt aanbevolen.
  \item Opslagruimte: Er moet minimaal 10 GB beschikbare opslagruimte zijn.
  \item Grafische kaart: Een standaard geïntegreerde grafische kaart is meestal voldoende.
  \item Internetverbinding: Een internetverbinding is vereist voor toegang tot cloudservices of updates. 
\end{itemize}
Vervolgens moet men ook rekening houden met een database platform, namelijk AirTable of Podio.
Hiervoor moet men een geldige account hebben voor AirTable of Podio. Deze vereisten zullen vervolgens in een 
document worden opgelijst, met meer details. Dit zal ongeveer 2 dagen duren.
Na de vereisingen zal de daadwerkelijke vergelijking gebeuren. Dit zal bestaan uit twee soorten vergelijkingen. Ten eerste
een vergelijking tussen Softr en Stacker waarbij we een simpel Web \& Mobile Portal maken. Hier zal dan gekeken naar feiten op vlak van snelheid en andere beschreven criteria's.
Ten tweede een vergelijking tussen Softr en Stacker waarbij we ook portals zullen maken maar uitgevoerd door drie gebruikers met weinig tot geen kennis van programmeren, ook wel eindklanten genoemd.
Naast de eindklanten zal dit ook uitgevoerd worden door drie programmeurs. Hierbij zal er gekeken worden naar de gebruiksvriendelijkheid van het platform en andere benodigde analyse dat niet door feiten kan worden aangetoond.
De eerste vergelijking zal ongeveer 3 weken duren terwijl de tweede 4 weken zal duren, inclusief data verwerking.

\subsection*{Conclusies}
\label{sub:conclusies}
Uiteindelijk zal er met de data van de proof-of-concept een conclusie worden gemaakt, in de 2 laatste weken, over welk No-Code en/of Low-Code platform het meest geschikt is voor het bedrijf Quivvy en eindklant. 
Waarbij men rekening houden met de integratie van AirTable of Podio. Dit onderzoek zal niet alles omvatten, maar zal wel een goed beeld geven over de No-Code en/of Low-Code platformen Softr en Stacker.
De aspecten die niet mogelijk zijn om te onderzoeken zijn; een uitgebreide onderzoek naar de gebruiksvriendelijkheid van het platform en de capaciteit van het platform.


% Hier beschrijf je hoe je van plan bent het onderzoek te voeren. Welke onderzoekstechniek ga je toepassen om elk van je onderzoeksvragen te beantwoorden? Gebruik je hiervoor literatuurstudie, interviews met belanghebbenden (bv.~voor requirements-analyse), experimenten, simulaties, vergelijkende studie, risico-analyse, PoC, \ldots?

% Valt je onderwerp onder één van de typische soorten bachelorproeven die besproken zijn in de lessen Research Methods (bv.\ vergelijkende studie of risico-analyse)? Zorg er dan ook voor dat we duidelijk de verschillende stappen terug vinden die we verwachten in dit soort onderzoek!

% Vermijd onderzoekstechnieken die geen objectieve, meetbare resultaten kunnen opleveren. Enquêtes, bijvoorbeeld, zijn voor een bachelorproef informatica meestal \textbf{niet geschikt}. De antwoorden zijn eerder meningen dan feiten en in de praktijk blijkt het ook bijzonder moeilijk om voldoende respondenten te vinden. Studenten die een enquête willen voeren, hebben meestal ook geen goede definitie van de populatie, waardoor ook niet kan aangetoond worden dat eventuele resultaten representatief zijn.

% Uit dit onderdeel moet duidelijk naar voor komen dat je bachelorproef ook technisch voldoen\-de diepgang zal bevatten. Het zou niet kloppen als een bachelorproef informatica ook door bv.\ een student marketing zou kunnen uitgevoerd worden.

% Je beschrijft ook al welke tools (hardware, software, diensten, \ldots) je denkt hiervoor te gebruiken of te ontwikkelen.

% Probeer ook een tijdschatting te maken. Hoe lang zal je met elke fase van je onderzoek bezig zijn en wat zijn de concrete \emph{deliverables} in elke fase?

%---------- Verwachte resultaten ----------------------------------------------
\section{Verwacht resultaat, conclusie}%
\label{sec:verwachte_resultaten}

Hier beschrijf je welke resultaten je verwacht. Als je metingen en simulaties uitvoert, kan je hier al mock-ups maken van de grafieken samen met de verwachte conclusies. Benoem zeker al je assen en de onderdelen van de grafiek die je gaat gebruiken. Dit zorgt ervoor dat je concreet weet welk soort data je moet verzamelen en hoe je die moet meten.

Wat heeft de doelgroep van je onderzoek aan het resultaat? Op welke manier zorgt jouw bachelorproef voor een meerwaarde?

Hier beschrijf je wat je verwacht uit je onderzoek, met de motivatie waarom. Het is \textbf{niet} erg indien uit je onderzoek andere resultaten en conclusies vloeien dan dat je hier beschrijft: het is dan juist interessant om te onderzoeken waarom jouw hypothesen niet overeenkomen met de resultaten.



%%---------- Andere bijlagen --------------------------------------------------
% TODO: Voeg hier eventuele andere bijlagen toe. Bv. als je deze BP voor de
% tweede keer indient, een overzicht van de verbeteringen t.o.v. het origineel.
%\input{...}

%%---------- Backmatter, referentielijst ---------------------------------------

\backmatter{}

\setlength\bibitemsep{2pt} %% Add Some space between the bibliograpy entries
\printbibliography[heading=bibintoc]

\end{document}
