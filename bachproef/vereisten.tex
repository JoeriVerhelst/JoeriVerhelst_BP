\chapter{\IfLanguageName{dutch}{Vereisten}{Requirements}}%
\label{ch:vereisten}
\section{Vereisten uit literatuurstudie}%
\label{sec:vereisten-literatuurstudie}
In de literatuurstudie kwamen er enkele criteriums naar boven die invloed kunnen hebben op 
de keuze van een Low-Code en/of No-Code platform. Voor een duidelijke overzicht zal hieronder een tabel opgesteld zijn met de volgende velden; 
Criteria, Reden/Relevantie, Bron. Ter info; de criteriums die in de tabel zijn opgenomen 
staan niet vast en zijn bedoeld als een leidraad voor de requirements analyse.
\begin{longtable}{lp{4.4cm}p{3.4cm}}
    \caption{Criteria lijst a.d.h.v. voorgaande literatuurstudie} \label{criteriums} \\
    \toprule
    \textbf{Criteria} & \textbf{Reden/Relevantie} & \textbf{Bron} \\
    \midrule
    \endfirsthead

    % Repeat the headers on the next page
    \multicolumn{3}{c}{{\bfseries \tablename\ \thetable{} -- vervolg van de vorige pagina}} \\
    \toprule
    \textbf{Criteria} & \textbf{Reden/Relevantie} & \textbf{Bron} \\
    \midrule
    \endhead

    % Footer at the end of each page, except the last page
    \midrule
    \multicolumn{3}{r}{{Vervolg op volgende pagina}} \\
    \endfoot

    % Footer at the end of the table
    \bottomrule
    \endlastfoot

    % Table content
    Snelheid van het platform & Om snel en eenvoudig ingewikkelde applicaties te ontwikkelen. &  \hyperref[subsec:snelheid]{Voordelen: snelheid} \\
    Snelheid van applicatieontwikkeling & Zorgt dat er meerdere applicaties in een korte tijd ontwikkeld kunnen worden. Vervolgens moeten bedrijven ook steeds snel en flexibel kunnen inspelen op de veranderende markt. &  \hyperref[subsec:snelheid]{Voordelen: snelheid} \& \hyperref[sec:reden-tot-gebruik]{Reden tot gebruik: Tijdrovend} \\
    Leercurve & Heel wat mensen laten zich afschrikken om te programmeren, door de complexiteit van de programmeertalen. Hierdoor is het belangrijk dat het snel en eenvoudig is om te leren. & \hyperref[sec:reden-tot-gebruik]{Reden tot gebruik: Beperkt aantal programmeurs} \\
    Updatebeleid \& Moderniteit & Omdat technologie vaak verandert, is de impact van een updatebeleid en moderniteit van het platform belangrijk op lange termijn. & \hyperref[sec:reden-tot-gebruik]{Reden tot gebruik: Technologische turbulentie} \\
    Kostprijs & Omdat Quivvy Solutions BV een start-up is, zijn er beperkte financiële middelen. Daardoor is het belangrijk dat de kostprijs niet over het budget gaat. & \hyperref[sec:reden-tot-gebruik]{Reden tot gebruik: Hoge kosten} \& \hyperref[sec:lcnc-bedrijven]{LCNC binnen bedrijven} \\
    Klantentevredenheid & Als klein bedrijf is het belangrijk dat de klanten tevreden zijn over de applicaties die worden ontwikkeld. &  \hyperref[sec:reden-tot-gebruik]{Reden tot gebruik: Klantentevredenheid} \\
    Veiligheid & Om ervoor te zorgen dat zowel het bedrijf Quivvy Solutions BV als hun eindklant zo weinig mogelijk schade kan ondervinden door het gebruikte platform. &  \hyperref[subsec:veiligheid]{Voordelen: Veiligheid} \\
    Herstelbeheer \& Back-up & Wanneer er zich een ramp afspeelt moeten de gegevens zo snel mogelijk kunnen hersteld worden. Hierdoor is regelmatig back-ups relevant. & \hyperref[subsec:cloud-based-lcnc]{Cloud-based LCNC : Dataherstel} \\
    Integratie mogelijkheden & Hoe meer Integratie mogelijkheden dat het LCNC platform heeft, hoe beter. & \hyperref[subsec:beperkte-flexibiliteit]{Nadelen: Beperkte Flexibiliteit} \\
    Platformflexibiliteit \& Aanpasbaarheid & Mogelijkheid om high-code en de mogelijke functionaliteiten te implementeren wanneer nodig. & \hyperref[subsec:beperkte-flexibiliteit]{Nadelen: Beperkte Flexibiliteit} \\
    Schaalbaarheid & Een applicatie in LCNC platformen zou gemakkelijk uitbreidbaar moeten zijn en de mogelijkheid moeten hebben om meer gelijktijdig gebruikers toe te laten. & \hyperref[subsec:gelimiteerde-schaalbaarheid]{Nadelen: Gelimiteerde schaalbaarheid} \\
    Documentatiekwaliteit &  Documentatie kan leiden tot een oplossing voor het beter verstaan van bijvoorbeeld het opslaan van data. & \hyperref[subsec:lcnc-binnen-agile]{LCNC in Software Development Life Cycle: Application Design} \\
    Test- en debugmogelijkheden  & LCNC platformen verwaarlozen vaak testen, maar dit blijft nogsteeds een belangrijke fase in de Software Development Life Cycle.  &  \hyperref[subsec:lcnc-binnen-agile]{LCNC in Software Development Life Cycle: Testing}\\
    \\\endline
\end{longtable}
\section{Bepaling van belangrijke vereisten}%
\label{sec:bepaling-van-belangrijke-vereisten}
Bij het antwoord van de copromotor kwam er ook andere criteriums naar boven. 
Deze bestaan uit; gebruiksvriendelijkheid, integratie met AirTable, 
en integratie met MAKE.com. Vervolgens heb ik samen met mijn copromotor overlegt welke
 criteriums de belangrijkste zijn en welke minder belangrijk zijn. Hieruit volgt een lijst van 
 requirements met een score op 10, waarbij 10 het belangrijkste is en 1 het minst belangrijk.

\begin{table}[H]
    \centering
    \caption{Requirements score}
    \begin{tabular}{llc}
    \toprule
    Criteria & Score \\
    \midrule
    Snelheid van het platform & 10 \\
    Herstelbeheer & 10 \\
    Veiligheid & 9 \\
    Snelheid van applicatieontwikkeling & 8 \\
    Integratie mogelijkheden & 8 \\
    Platformflexibiliteit (bv. high-code kunnen implementeren) & 8 \\
    Schaalbaarheid & 8 \\
    Gebruiksvriendelijkheid & 8 \\
    Integratie met AirTable en/of MAKE.com & 8 \\
    Kostprijs & 7.5 \\
    Updatebeleid & 7.5 \\
    Klantentevredenheid & 7.5 \\
    Test- en debugmogelijkheden & 7.5 \\
    Leercurve & 6 \\
    Documentatiekwaliteit & 6 \\
   
    \bottomrule
 \end{tabular}
\end{table}

\section{Officiële Vereisten}%
\label{sec:officiële-vereisten}
De officiële requirements bevatten de criteriums die een score hebben van 8 of hoger, 
met uitzondering van kostprijs en updatebeleid. De oorzaak hiervoor is dat de 
kostprijs toch wel belangrijk is voor een start-up en het updatebeleid voor 
de lange termijn. Hieruit volgt dat de klantentevredenheid, test- en debug mogelijkheden, 
documentatiekwaliteit en leercurve niet in de officiële requirements zullen opgenomen worden. 
Een reden waarom leercurve en documentatiekwaliteit niet belangrijk is, is omdat er hedendaags bronnen 
zijn die kunnen helpen bij het leren van een platform, daarbij beschikt een platform vaak over goede documentatiekwaliteit. Zo niet, 
dan zijn er tutorials die kunnen helpen bij een bepaalt probleem.
