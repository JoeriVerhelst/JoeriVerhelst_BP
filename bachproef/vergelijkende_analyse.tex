\chapter{\IfLanguageName{dutch}{Vergelijkende Analyse}{Comparative Analysis}}%
\label{ch:vergelijkende-analyse}
\subsection{Snelheid van het platform}%
\label{subsec:snelheid-van-het-platform}
\paragraph{Softr}
Volgens \textcite{Code2023} 
beschikt Softr over een score van 4.5 op 5 op het vlak van snelheid. 
Dit komt omdat Softr het toelaat om in real-time data op te halen en aan te passen. \autocite{Youssef2023}.
\paragraph{Stacker}
Met Stacker heb je niet het voordeel van real-time data aan te passen, wat Softr wel heeft.
 Het heeft namelijk 5 tot 10 minuten vertraging bij het updaten van data, als men de data bijvoorbeeld opslaat in AirTable \autocite{Youssef2023}.

\subsection{Herstelbeheer}%
\label{subsec:herstelbeheer}
Het voordeel aan Low-Code en No-Code platformen is dat het cloud-based is, waardoor de drie platformen automatisch beschikken over het herstellen van data in het geval van nood. 
Voor meer details kan men terecht bij \hyperref[subsec:cloud-based-lcnc]{Cloud-Based LCNC: Dataherstel}.
\subsection{Veiligheid}%
\label{subsec:veiligheid}
\paragraph{Bubble}
Bubble.io neemt het veilig opstellen en hosten van de applicatie in eigen handen, wat betekent dat de gebruikers van het platform niks kan verkeerd doen op het vlak van veiligheid \hyperref[subsec:bubble]{Alternatieven: Bubble}.
In Bubble.io is het veilig om een business applicatie te ontwikkelen. Hiervoor neemt Bubble verschillende maatregelen om 100\% veiligheid te creëren. 
Als eerste test en monitort Bubble.io op kwetsbaarheden \autocite{Agency2023}. Vervolgens heb je de optie om point-in-time dataherstel te gebruiken, dit betekent dat je terug kan keren naar verschillende versies om belangrijke informatie op te halen. Daarnaast is Bubble ook transparant met hun gebruikers waarbij je uitgebreide logs van je app kan bekijken. Voor het beschermen van je data gebruikt Bubble een data encryptie genaamd AWS RDS’s AES-256. Daarbij worden ook wachtwoorden van gebruikers en eigenaar van de app niet opgeslagen waardoor de kans dat je een wachtwoordlek hebt door Bubble zeer klein is. Ten slotte voor betalingen in je app maakt Bubble gebruik van Stripe.
\paragraph{Softr}
Softr gebruikt zoals Bubble Amazon Web Services (AWS) om data te beschermen \autocite{Softr}. Omdat Softr een Europees bedrijf is probeert het bedrijf dan ook alle opgeslagen data binnen EU te houden, in dit geval Duitsland. Hierbij bestaat de datacenter uit Service Organization Control (SOC) 1, SOC 2 en ISO 27001 gecertificeerd. Daarnaast wordt het ook nog eens 24/7 beveiligd. Het versturen van data tussen uw apparaat en de servers van 
Softr is beschermd door de encryptie 256-bit Transport Layer Security (TLS) encryptie. Als laatste worden betalingen van je app in Softr afgehandeld via Stripe.
\paragraph{Stacker}
Stacker is een platform die goed en grondig helpt bij het beveiligen van de data dat je toepast in je app \autocite{JDN2023}.
 Helaas is er weinig informatie weergegeven over de veiligheid van Stacker. Maar dit is wel beschikbaar.
 Het is mogelijk dat jouw gegevens worden overgezet naar een andere datacenter in de EU \autocite{Stacker2023}. 
 Deze gegevens worden opgeslagen op computerservers, in een gecontroleerde omgeving. Voor betalingen maakt Stacker gebruik van third-party betalingsprocessen.
\subsection{Snelheid van applicatieontwikkeling}%
\label{subsec:snelheid-van-applicatieontwikkeling}
\paragraph{Bubble}
In dit platform heb je de mogelijkheid om samen in real-time te werken aan een project in Bubble.io, wat zeer nuttig is voor bedrijven \autocite{Bubble2024b}. 
Met dit voordeel kan je de snelheid van het ontwikkelen van een project vermenigvuldigen.
\paragraph{Softr}
Bij Softr zijn ze helemaal mee met de tijd doordat ze AI features bieden bij het ontwerpen van je applicatie. 
De AI van het platform kan namelijk je eigen app genereren maar ook verschillende onderdelen binnen je app 
die je later nog kan aanpassen \autocite{Frater2024}. Dit is niet het enige wat de Softr’s 
AI kan doen, het heeft ook een tekst generator om de snelheid van een developer te versnellen. 
Dit versnelt de applicatieontwikkeling want dit zorgt ervoor dat de developer zelf geen 
tijd moet verspillen bij het bedenken van een gepaste titel of tekst in het algemeen \autocite{Frater2024}. 
Daarnaast zorgt het visuele drag-and-drop interface van het platform tot het snel bouwen van een applicatie \autocite{Code2023}. 
\paragraph{Stacker}
Stacker is een platform dat kan gebruikt worden als een front-end app voor AirTable \autocite{Advice}. 
De combinatie van AirTable en Stacker zorgt voor een vloeiende applicatieontwikkeling door de perfecte samenwerking tussen deze twee platformen \autocite{Advice}. 
Zoals Bubblie.io brengt Stacker ook real-time samenwerking op tafel \autocite{Allen2020}.
\subsection{Integratie mogelijkheden}%
\label{subsec:integratie-mogelijkheden}
\paragraph{Bubble}
Voor het platform Bubble.io is er geen exacte nummer op het vlak van integraties maar er kan wel geïntegreerd worden met zowel AirTable als MAKE.com \hyperref[subsec:bubble]{Alternatieven: Bubble}. 
Daarnaast biedt het ook nog andere integraties zoals Google Maps, Figma, Zapier en Google Drive.
\paragraph{Softr}
Met het platform Softr kan je enkel met 4 databases integreren, namelijk; Google Sheets, AirTable, HubSpot Beta, en Smart Suite Beta \autocite{Frater2024}. 
Het kan ook met andere soorten platformen integreren zoals; Stripe, PayPal, Buy Me a Coffee voor betalingen en MAKE.com voor het creëren van complexe automatiseringen \autocite{Code2023} \autocite{Youssef2023}. 
Voor het e-mailen kan je ook connecteren met platformen als MailChimp en MailerLite. Indien je een tracking website moet maken kan je het connecteren met Google Analytics en Google Tag Manager.
\paragraph{Stacker}
Stacker heeft de optie Google Sheets, SalesForce en AirTable voor het opslaan van data \autocite{Englert2021} \autocite{JDN2023} \autocite{Youssef2023}. 
Helaas kan je met Stacker geen connectie leggen met MAKE.com, maar wel met Zapier. 
Zapier is evenwaardig platform zoals MAKE.com waarbij je Stacker met duizenden apps kan connecteren \autocite{Zapier}.
\subsection{Platformflexibiliteit}%
\label{subsec:platformflexibiliteit}

\paragraph{Bubble}
Dit platform geeft de mogelijkheid om zowel apps te maken als websites \autocite{Youssef2023}. 
Deze apps en website kunnen verschillende dingen zijn zoals een klant portaal, e-commerce, en CRM \autocite{Sharma2022}. 
Volgens \textcite{Youssef2023} is Bubble.io meer aanpasbaar dan het No-Code platform Softr. Helaas biedt Bubble.io nog niet het maken van native mobiele applicaties aan, dus spreekt men van Progressive Web Apps (PWA’s). 
Dit betekent dat het niet mogelijk is om je app dat gemaakt is in Bubble.io te publiceren op bijvoorbeeld Google Play Store. Volgens \textcite{Sharma2022} bestaan hier wel omzeilingen voor via third-party services. Dit platform biedt zoals geen andere no-code platform de flexibiliteit op het vlak je eigen database, services en eigen code te integreren in je Bubble.io app \autocite{Bas2024}. 
Vervolgens heeft het platform een marketplace waar je third-party templates en plugins kan kopen \autocite{Sharma2022}. 
\paragraph{Softr}
Met dit platform kan je verschillende websites maken zoals; portals voor klanten, interne tools, een webshop, en een website voor een online community op te bouwen\autocite{Code2023} \autocite{Youssef2023} .
Softr bezit over meer dan 30 templates en 100 vóór opgebouwde componenten die je kan gebruiken om je app te 
ontwikkelen \autocite{Frater2024} \autocite{Youssef2023}. 
Deze templates bestaan uit vóór opgebouwde componenten die bedoeld zijn voor te helpen bij het ontwerpen, features en integreren met andere platformen. 
Ook hier zit je met het probleem dat je enkel PWA’s kan maken, dus enkel beschikbaar via een browser. \autocite{Frater2024}. 
Op het vlak van ontwerpen in Softr ben je wel gelimiteerd. Het platform beschikt ook over een ingebouwde formulier ontwerper/maker \autocite{Youssef2023}. 

\paragraph{Stacker}
Het ontwikkelen van web- en mobiele applicaties, het creëren van formulieren, het versturen van meldingen, het beheren van documenten, 
het beheren van automatiseringen, en zelfs het delen en overdragen van bestanden is allemaal mogelijk met het platform 
Stacker \autocite{JDN2023}. 
Het nadeel dat Stacker bevat is dat een scherm van je app soms een template moet gebruiken voor bepaalde functies 
of een bepaalde versie van het platform moet hebben om het goed te laten werken \autocite{Advice}.
\subsection{Schaalbaarheid}%
\label{subsec:schaalbaarheid}
\paragraph{Bubble}
Zoals eerder vermeld in \hyperref[subsec:bubble]{Alternatieven: Bubble} is er geen limiet op het aantal gebruikers van je app of website. Volgens \textcite{Youssef2023} is 
Bubble.io meer schaalbaar dan het platform Softr, waardoor Bubble beter is voor zowel in de toekomst te gebruiken als voor grote bedrijven. 
Dit komt omdat Bubble.io automatisch zal schalen wanneer de app meer nodig heeft, 
waardoor de app steeds snel en online blijft \autocite{Bas2024}.
\paragraph{Softr}
De schaalbaarheid van het platform hangt af van het abonnement \autocite{Frater2024}. 
Als je bezit over de standaardversie heb je toegang tot 1000 eindgebruikers en 10 interne gebruikers. 
Voor een klein bedrijf zal je meer gaan voor het professional abonnement dat beschikt over 5000 eindgebruikers en 50 interne gebruikers. Is dit nog niet genoeg, 
dan heb je de optie voor een businessplan met maar liefst 10000 eindgebruikers en 100 interne gebruikers. Daarnaast is Softr redelijk ontwerpgericht waardoor je steeds 
meer opties hebt om je applicaties aantrekkelijk te maken voor je eindgebruikers \autocite{Noloco2023}.
\paragraph{Stacker}
Stacker zorgt ervoor dat je geïntegreerde databronnen en je workflow efficiënt worden beheerd, op een centraal punt, zodat je als bedrijf makkelijker kan aanpassen aan veranderingen en 
bepaalde behoeften \autocite{Noloco2023}. Dus wanneer je bedrijf groeit zal jouw Stacker applicatie mee kunnen groeien om meer databronnen, gebruikers en samenwerking te ondersteunen.
\subsection{Gebruiksvriendelijkheid}%
\label{subsec:gebruiksvriendelijkheid}
\paragraph{Bubble}
Volgens \textcite{Czerny2024} 
heeft Bubble een gebruiksvriendelijke interface door de gemakkelijke drag-and-drop editor die gebruikers toelaat om eenvoudig en snel hun ideeën tot een echte app te verwerken. 
Op \textcite{Capterra} bezit Bubble over een score van 4.3/5 op vlak van gebruiksvriendelijkheid.
\paragraph{Softr}
Volgens \textcite{Code2023} heeft dit platform een score van maar liefst 4.5/5 op support en community. 
Op \textcite{Capterra} bezit het een score van 4.8/5 op gebruiksvriendelijkheid.
Dit komt omdat Softr een 24/7 support systeem heeft \autocite{Youssef2023}. 
Helaas kan het limiteren van databases een gebruiker ontevreden stellen doordat de gebruiker niet gewoon is om 1 van de 4 databases te gebruiken 
waardoor het mogelijk is dat hun favoriete database niet kan gekozen worden \autocite{Frater2024}. 
Maar dit brengt ook een positief puntje mee want door de gemakkelijke integratie met AirTable of Google Sheets is het eenvoudig om als beginner op het platform data op te slaan \autocite{Code2023}. Softr heeft namelijk ook een zeer eenvoudige en visuele drag-and-drop interface zodat het makkelijk is om je applicatie op te bouwen. Hieruit volgt dat volgens \textcite{Youssef2023} Softr gebruiksvriendelijker is dan Bubble.io .
\paragraph{Stacker}
Stacker heeft een gebruiksvriendelijkheid score van 4.5/5 op \textcite{Capterra}
en \textcite{Advice}. Het heeft ook op klantenservice een score van 4.7/5 op Capterra en 4.5/5 op Software Advice.

\subsection{Kostprijs}%
\label{subsec:kostprijs}

\paragraph{Softr}
Softr begint met het abonnement Standaard waarbij je 59 dollar per maand neerlegt \autocite{Frater2024}
\autocite{Youssef2023}. Vervolgens heb je de Professional Version dat een bedrag bevat van 167 dollar per maand, 
dit abonnement is ook meer bedoeld voor kleine bedrijven. Daarnaast kom je terecht bij het Business Plan dat 323 dollar per maand is. 
Indien dit niet genoeg is kan je ook altijd Softr contacteren voor een gepersonaliseerde abonnement, wat ongeveer rond de 1000 dollar per maand ligt. 
Daarbij kan de prijs van een abonnement ook veel afhangen van hoeveel interne en externe gebruikers gebruik zullen maken van een app \autocite{Frater2024}.
\paragraph{Stacker}
Stacker biedt 4 verschillende pakketten aan, afhankelijk van de wensen van het bedrijf \autocite{JDN2023}
\autocite{Advice}. 
Bij een Starter-pakket begint het bedrag bij 79 dollar per maand (\$59 per maand met jaarlijkse abonnement).  
Daarna heb je Plus-pakket van 179 dollar per maand (\$149/maand jaarlijks). Vervolgens kan je kiezen uit het Pro-pakket 
dat begint bij een prijs van 349 dollar per maand (\$290/maand jaarlijks). Als laatste heb je dan nog het Business-pakket waarbij je een offerte zal krijgen.

\paragraph{Bubble}
De kostprijs van Bubble.io werd eerder vermeld in het deel \hyperref[subsec:bubble]{Alternatieven: Bubble}. 
Samengevat is het eerste en tweede abonnement minder duur dan dat van Stacker, maar bij volgende abonementen evenaart de prijs tussen Bubble en Stacker.
\subsection{Updatebeleid}%
\label{subsec:updatebeleid}
Stacker brengt ongeveer elke maand 3 updates uit die vaak nieuwe componenten of integraties voorstellen \autocite{Stacker}.  
Softr brengt ook ongeveer elke maand 3 updates uit waar er duidelijk wordt weergegeven wat nieuw en verbeterd is \autocite{Softra}. 
Vervolgens heb je Bubble.io die meerdere releases publiceert op een dag \autocite{Bubble}. Deze updates zijn ook veel transparanter met hun gebruikers. 
Dit brengt Bubble.io tot een superieur bij het updatebeleid.