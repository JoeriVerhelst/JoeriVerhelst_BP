%%=============================================================================
%% Methodologie
%%=============================================================================

\chapter{\IfLanguageName{dutch}{Methodologie}{Methodology}}%
\label{ch:methodologie}

%% TODO: In dit hoofstuk geef je een korte toelichting over hoe je te werk bent
%% gegaan. Verdeel je onderzoek in grote fasen, en licht in elke fase toe wat
%% de doelstelling was, welke deliverables daar uit gekomen zijn, en welke
%% onderzoeksmethoden je daarbij toegepast hebt. Verantwoord waarom je
%% op deze manier te werk gegaan bent.
%% 
%% Voorbeelden van zulke fasen zijn: literatuurstudie, opstellen van een
%% requirements-analyse, opstellen long-list (bij vergelijkende studie),
%% selectie van geschikte tools (bij vergelijkende studie, "short-list"),
%% opzetten testopstelling/PoC, uitvoeren testen en verzamelen
%% van resultaten, analyse van resultaten, ...
%%
%% !!!!! LET OP !!!!!
%%
%% Het is uitdrukkelijk NIET de bedoeling dat je het grootste deel van de corpus
%% van je bachelorproef in dit hoofstuk verwerkt! Dit hoofdstuk is eerder een
%% kort overzicht van je plan van aanpak.
%%
%% Maak voor elke fase (behalve het literatuuronderzoek) een NIEUW HOOFDSTUK aan
%% en geef het een gepaste titel.

\section*{Requirements analyse}
\label{sec:requirements-analyse}
Voor de onderzoeksvraag te beantwoorden is het belangrijk om de requirements te analyseren. De opgestelde requirements zullen de basis vormen voor 
de evaluatie, de vergelijkende analyse en de proof of concept. Deze vereisten zullen aan de hand van verschillende bronnen worden opgesteld. Als eerste zal 
er in de voorgaande literatuurstudie worden gezocht naar de belangrijkste vereisten. Vervolgens zal er een vraag worden gesteld die zal worden 
beantwoord door mijn co-promotor. De vraag bestaat uit, namelijk welke vereisten zijn volgens u belangrijk bij het kiezen van een Low-Code platform en rangshik deze vereisten van belangrijk naar minder belangrijk. Op deze
manier kan er een duidelijke beeld worden gevormd van waar er rekening mee moet worden gehouden bij het evalueren van de Low-Code en/of No-Code platformen.
\\
\\
In de literatuurstudie kwam er een aantal criteria's naar boven die impact kunnen hebben op de keuze van een Low-Code en/of No-Code platform.
Om hiervoor een eenvoudig overzicht te krijgen, zal er hieronder een tabel worden opgesteld met de volgende velden; Criteria, Reden/Relevantie, Bron. 
Ter info, de criteria's die in de tabel zullen worden opgenomen staan niet vast en zijn meer bedoelt als een leidraad voor de requirements analyse.
\\
\\


\begin{longtable}{lp{4.4cm}p{3.4cm}}
    \caption{Criteria lijst a.d.h.v. voorgaande literatuurstudie} \label{crouch} \\
    \toprule
    \textbf{Criteria} & \textbf{Reden/Relevantie} & \textbf{Bron} \\
    \midrule
    \endfirsthead

    % Repeat the headers on the next page
    \multicolumn{3}{c}{{\bfseries \tablename\ \thetable{} -- vervolg van de vorige pagina}} \\
    \toprule
    \textbf{Criteria} & \textbf{Reden/Relevantie} & \textbf{Bron} \\
    \midrule
    \endhead

    % Footer at the end of each page, except the last page
    \midrule
    \multicolumn{3}{r}{{Vervolg op volgende pagina}} \\
    \endfoot

    % Footer at the end of the table
    \bottomrule
    \endlastfoot

    % Table content
    Snelheid van het platform & Om snel en eenvoudig ingewikkelde applicaties te ontwikkelen. & Voordelen: snelheid \\
    Snelheid van applicatieontwikkeling & Zorgt dat er meerdere applicaties in een korte tijd kunnen worden ontwikkeld. Vervolgens moeten bedrijven ook steeds snel en flexibel kunnen inspelen op de veranderende markt. &  \hyperref[subsec:snelheid]{Voordelen: snelheid} \& \hyperref[sec:reden-tot-gebruik]{Reden tot gebruik: Tijdrovend} \\
    Leercurve & Heel wat mensen laten zich afschrikken tot het programmeren door de complexiteit van de programmeertalen. Hierdoor is het belangrijk dat het snel en eenvoudig is om te leren. & \hyperref[sec:reden-tot-gebruik]{Reden tot gebruik: Beperkt aantal programmeurs} \\
    Updatebeleid \& Moderniteit & Omdat technologie vaak verandert is de impact van een updatebeleid en moderniteit van het platform belangrijk op de lange termijn. & \hyperref[sec:reden-tot-gebruik]{Reden tot gebruik: Technologische turbulentie} \\
    Kostprijs & Omdat Quivvy Solutions BV een start-up is, zijn er beperkte financiële middelen. Waardoor het belangrijk is dat de kostprijs niet over het budget gaat. & \hyperref[sec:reden-tot-gebruik]{Reden tot gebruik: Hoge kosten} \& \hyperref[sec:lcnc-bedrijven]{LCNC binnen bedrijven} \\
    Klantentevredenheid & Als klein bedrijf is het belangrijk dat de klanten tevreden zijn over de applicaties die worden ontwikkeld. Daarbij moet de eindklant van het bedrijf ook zelf kunnen werken met het LCNC platform, waardoor de klantentevredenheid ook een belangrijk criterium is. &  \hyperref[sec:reden-tot-gebruik]{Reden tot gebruik: Klantentevredenheid} \\
    Veiligheid & Om ervoor te zorgen dat zowel het bedrijf Quivvy Solutions BV als hun eindklant zo weinig mogelijk schade kan krijgen door het gebruikte platform. &  \hyperref[subsec:veiligheid]{Voordelen: Veiligheid} \\
    Herstelbeheer \& Back-up & Wanneer er zich een ramp afspeelt moet de gegevens zo snel mogelijk kunnen hersteld worden. Hierdoor is het regelmatig back-ups relevant. & \hyperref[subsec:cloud-based-lcnc]{Cloud-based LCNC : Dataherstel} \\
    Integratie mogelijkheden & Hoe meer Integratie mogelijkheden dat het LCNC platform heeft, hoe beter. Flexibiliteit in een platform heeft een impact op de keuze van het platform. & \hyperref[subsec:beperkte-flexibiliteit]{Nadelen: Beperkte Flexibiliteit} \\
    Platformflexibiliteit \& Aanpasbaarheid & Mogelijkheid om high code te implementeren wanneer nodig en de mogelijke functionaliteiten. & \hyperref[subsec:beperkte-flexibiliteit]{Nadelen: Beperkte Flexibiliteit} \\
    Schaalbaarheid & Een applicatie in LCNC platformen zou gemakkelijk uitbreidbaar moeten zijn en wanneer mogelijk meer tegelijk gebruikers toelaten. & \hyperref[subsec:gelimiteerde-schaalbaarheid]{Nadelen: Gelimiteerde schaalbaarheid} \\
    Documentatiekwaliteit &  Documentatie kan leiden tot een oplossing voor het beter verstaan van bijvoorbeeld het opslaan van data. & \hyperref[subsec:lcnc-binnen-agile]{LCNC in Software Development Life Cycle: Application Design} \\
    Test- en debugmogelijkheden  & LCNC platformen verwaarlozen vaak testen, maar dit blijft nogsteeds een belangrijke fase in de Software Development Life Cycle.  &  \hyperref[subsec:lcnc-binnen-agile]{LCNC in Software Development Life Cycle: Testing}\\
    \\\endline
\end{longtable}
Bij het antwoord van de copromoter kwam er ook andere criteria's naar boven. Deze bestaan uit de volgende criteria's;
gebruiksvriendelijkheid, integratie met AirTable, en integratie met MAKE.com . Vervolgens heb ik samen met mijn copromoter overlegd welke criteria's de belangrijkste zijn en welke minder belangrijk zijn.
Hieruit volgt een lijst van requirements met een score op 10, waarbij 10 het belangrijkste is en 1 het minst belangrijk.

\begin{table}[H]
    \centering
    \caption{Requirements score}
    \begin{tabular}{llc}
    \toprule
    Criteria & Score \\
    \midrule
    Snelheid van het platform & 10 \\
    Herstelbeheer & 10 \\
    Veiligheid & 9 \\
    Snelheid van applicatieontwikkeling & 8 \\
    Integratie mogelijkheden & 8 \\
    Platformflexibiliteit (bv. high-code kunnen implementeren) & 8 \\
    Schaalbaarheid & 8 \\
    Gebruiksvriendelijkheid & 8 \\
    Integratie met AirTable en/of MAKE.com & 8 \\
    Kostprijs & 7.5 \\
    Updatebeleid & 7.5 \\
    Klantentevredenheid & 7.5 \\
    Test- en debugmogelijkheden & 7.5 \\
    Leercurve & 6 \\
    Documentatiekwaliteit & 6 \\
   
    \bottomrule
 \end{tabular}
\end{table}

\subsubsection*{Officiële requirements}
De officiële requirements bevat de criteria's die een score hebben van 8 of hoger, met uitzondering van kostprijs en updatebeleid. De reden hiervoor is omdat de kostprijs toch wel belangrijk
is voor een start-up en het updatebeleid is belangrijk voor de lange termijn. Hieruit volgt dat de klantentevredenheid, test- en debugmogelijkheden, documentatiekwaliteit en leercurve niet in de officiële requirements zullen worden opgenomen.
Een mogelijke reden waarom leercurve en documentatiekwaliteit niet belangrijk is omdat er hedendaags veel bronnen zijn die kunnen helpen bij het leren van een platform en de documentatiekwaliteit is vaak goed bij de meeste platformen, zo niet dan zijn er 
toturials die meestal kunnen helpen bij een bepaalde probleem.


%% TODO: In dit hoofstuk geef je een korte toelichting over hoe je te werk bent
%% gegaan. Verdeel je onderzoek in grote fasen, en licht in elke fase toe wat
%% de doelstelling was, welke deliverables daar uit gekomen zijn, en welke
%% onderzoeksmethoden je daarbij toegepast hebt. Verantwoord waarom je
%% op deze manier te werk gegaan bent.

\section*{Evaluatie en Selectie van alternatieven}
\label{sec:evaluatie-en-selectie-van-alternatieven}
Bij deze fase zal er onderzoek gedaan worden naar verschillende Low-Code en/of No-Code platformen. Hierbij zal 
er een online onderzoek gebeuren waarbij men bronnen zal gebruiken zoals officiële websites, reviews, en technologieblogs.
Vervolgens zal er per platform de voordelen, nadelen en beschrijvende informatie worden opgenomen. Daarna komt de belangrijkste
eigenschappen van een platform terecht in een tabel waarbij de criteria's van de requirements analyse zullen worden opgenomen. Hieruit volgt een
selectie van een alternatief die verder zal worden geanalyseerd.
\subsection*{Alternatieve platformen}
\label{subsec:alternatieve-platformen}

\subsubsection*{Zoho Creator}
Zoho Creator is een Low-Code platform, wat ook een gemakkelijke platform is om te gebruiken, dat werknemers van bedrijven maar ook mensen zonder programmeerervaring toelaat
om eenvoudig krachtige bedrijf applicaties te ontwikkelen \autocite{Computer2022}. Dit platform is gemaakt door Zoho Corporation. \textcite{ZohoCorporation2024a} gelooft er in dat software de ultieme tool is voor zowel de handen als de hersenen.
Zoho Corporation is een bedrijf dat vooral focust op het verder ontwikkelen van hun product, zoals Zoho Creator, en hun klanten support \autocite{ZohoCorporation2024a}. Dit toont aan dat Zoho Corporation
hun updatebeleid volgens hun website goed is. Volgens \textcite{ZohoCorporation2024a} biedt het bedrijf meer dan alleen een product, maar zoals het bedrijf het noemt 'the operating system for business' \autocite{ZohoCorporation2024a}.
Dit bevat maar liefst 55 integreerde applicaties voor zowel mobile als web voor elke bedrijfsnood \autocite{ZohoCorporation2024a}.

\paragraph{Functies}
In 2022 was er nog geen enkele Low-Code platform die toeliet om business gebruikers en IT een end to end business oplossingen te maken \autocite{Computer2022}.
De Zoho Creator platform zorgt ervoor dat je applicatie development, business intelligence en analytics, integraties, en automatiseringen kan doen in één platform, wat men ook een end to end business oplossing noemen \autocite{Computer2022}.
Dit geeft verschillende voordelen voor het bedrijf dat dit platform gebruikt zoals het verhogen van veiligheid maar ook omdat het platform een end to end business oplossing is, is het ook mogelijk om uniforme oplossingen te maken via de low-code platform,
waardoor elke werknemer de mogelijkheid krijgt om te innoveren \autocite{Computer2022}. Volgens \textcite{Computer2022} zorgt Zoho Creator er ook voor dat de business gebruikers snel een schaalbare low-code oplossingen kunnen maken zoals het maken van een app of automatisering.
Het ontwikkelen van een low-code oplossing is zelfs 10 keer sneller dan eender welke andere oplossing op de markt \autocite{Computer2022}. Het platform heeft ook hun eigen AI, genaamd Zia, die de gebruiker helpt bij het maken van applicaties \autocite{Computer2022}. Deze AI
kan ontwikkelaars helpen bij het importeren van data door deze te analyseren en te structureren \autocite{Computer2022}. Daarnaast kan het ook relaties tussen data detecteren \autocite{Computer2022}. 
Zoho Creator heeft de mogelijkheid om zelf code te schrijven, wat een groot voordeel is voor bedrijven die toch nog high-code willen implementeren \autocite{Computer2022}. 
De code kan geschreven worden in Deluge, Java of Node.js en zijn ook herbruikbaar om duplicatie te voorkomen \autocite{Computer2022}.
De gebruikers van het platform Zoho Creator kunnen ook zien hoe goed integraties aan het werken zijn aan de hand van de 'Integration Status Dashboard', wat helpt om te anlayseren waar er problemen zijn \autocite{Computer2022}.
\\ %TODO (nu enkel info van Express Computer, moet nog andere bronnen zoeken)
\\
Volgens het bedrijf \autocite{ZohoCorporation2024b} zorgt het platform ervoor dat 90\% van de complexiteit van het ontwikkelen van applicaties wordt weggenomen. Dit is zo dat de gebruikers van het platform
kunnen focussen op de functies, businesswaarde en de eindklant \autocite{ZohoCorporation2024b}.  Met Zoho Creator kan je heel wat verschillende apps maken zoals volledig jouw eigen app, mobile apps, en een online portal \autocite{ZohoCorporation2024b}.
Maar Zoho Creator kan je voor meer dan app development gebruiken, je kan businessprocessen automatiseren, BI \& Analytics, en integraties maken met andere systemen \autocite{ZohoCorporation2024b}. Daarnaast zorgt het ook voor een veilige omgeving door 
jouw geen zorgen te maken over middleware, authenticatie, en business transacties want Zoho Creator hebben hier namelijk services voor \autocite{ZohoCorporation2024b}. Vervolgens hebben Zoho Creators verschillende functies zoals het opslaan van data,
betalingstransacties, het updaten van je CRM, en emails en rapporten sturen \autocite{ZohoCorporation2024b}.

\paragraph*{Voordelen}
\begin{enumerate}
    \item Het heeft een zeer simpele en straightforward interface \autocite{Marvin2017Zoho}.
    \item Zoho Creator is betaalbaar  \autocite{Marvin2017Zoho}.
    \item Het heeft een grote selectie aan pre-built app templates en fields \autocite{Marvin2017Zoho}.
    \item Je kan je eigen workflows creëren \autocite{Marvin2017Zoho}.
    \item Het heeft ingebouwde auto translation \autocite{Marvin2017Zoho}.
    \item ondersteund barcode scanning \autocite{Marvin2017Zoho}.
\end{enumerate}


\paragraph*{Nadelen}
\begin{enumerate}
    \item Het heeft een gelimiteerde aantal features \autocite{G22024}.
    \item Het heeft database limitaties \autocite{G22024}.
    \item Zoho Creator heeft een zwakke Customer Support \autocite{G22024}.
\end{enumerate}

\paragraph{Integratie mogelijkheden}
Zoho Creator heeft een lijst met maar liefst 600+ Apps waarmee het kan integreren \autocite{ZohoCorporation2024b}. Ze hebben integraties voor Sales automation, IT Security en Team Collaboration \autocite{ZohoCorporation2024b}.
Helaas heeft Zoho Creator nog geen integratie met AirTable, maar heeft wel integratie met MAKE.com \autocite{MAKE.com2024}.

\paragraph{Beoordelingen}
Op \textcite{Gartner2024} heeft Zoho Creator een beoordeling van 4.6/5.0, gebaseerd op 205 beoordelingen. Hieruit volgt dat 38\% van bedrijven met minder dan 50 miljoen dollar waarde Zoho Creator aanbevelen, 
en 37\% van bedrijven met meer dan 50 miljoen dollar waarde tot 1 miljard Zoho Creator aanbevelen \autocite{Gartner2024}. Vervolgens waren de beoordelingen, met 4.3/5, op business logica en workflow, integratie met API's, en platformflexibiliteit \autocite{Gartner2024}.
\paragraph{Kostprijs}
%per abonnement; herstelbeheer, integratie, automatiseringen (ai)
Zoho Corporation heeft verschillende soorten abonnementen voor hun platform, deze zijn Standard, Professional, Enteprise, en Flex \autocite{ZohoCorporation2024}.
De standaard abonnementen bedraagt 12 euro, per gebruiker, per maand. Hiermee kan je maar maximaal 1 business applicatie maken \autocite{ZohoCorporation2024}.
Daarbij wordt ook je data opgeslagen en geback-upt in de cloud en 20 AI models maken om te trainen voor specifieke taken \autocite{ZohoCorporation2024}. Op vlak van integratie
kan je maar maximum 5 data bronnen integreren en 5 eigen connectie maken met systemen dat niet ondersteund worden door Zoho Creator \autocite{ZohoCorporation2024}. Vervolgens heb je dan het
abonnement Professional, wat 30 euro per gebruiker, per maand bedraagt. Hiermee kan je unlimited business applicaties maken en 100 AI models trainen \autocite{ZohoCorporation2024}. 
Daarbij kan je 15 data bronnen integreren en 10 eigen connecties maken \autocite{ZohoCorporation2024}. Daarnaast heb je dan Enteprise abonnement, wat 37 euro per gebruiker, per maand bedraagt.
Het voordeel van dit abonnement is dat je integratie mogelijkheid hebt met 650+ business apps, 30 data bronnen integreren, en 20 eigen connecties maken \autocite{ZohoCorporation2024}. Vervolgens heb je ook
Business Intelligence and Analytics bij Enteprise om zeer eenvoudig en verschillende anlaysis te maken \autocite{ZohoCorporation2024}. Het handige is dat je ook de mogelijkheid hebt om hun te contacteren in het geval 
van gepersonaliseerde requirements via het Flex abonnement \autocite{ZohoCorporation2024}.

\subsubsection*{OutSystems}
OutSystems is een Low-Code platform dat developers pixel-perfect applicaties laat maken aan de hand van een drag-and-drop functionaliteiten \autocite{Ranosys2023} \autocite{Payne2023}.
OutSystems maakt gebruik van AI, cloud technologie, DevOps, en visuele ontwikkeling om citizen developers deel te laten nemen applicatie ontwikkeling \autocite{Ranosys2023}.

\paragraph{Functies}
Deze Low-code platform heeft een unieke visuele omgeving dat het ontwinkkelingsproces versnelt en de complexiteit van het ontwikkelen van applicaties vermindert \autocite{Payne2023}.
Hierdoor gaat de productiviteit naar omhoog, vermindert de leercurve, en vermindert het aantal errors and bugs \autocite{Payne2023}. OutSystems heeft ook heel wat built-in templates 
en componenten die het ontwikkelen van applicaties versnelt en vermindert de eigen gemaakt code \autocite{Payne2023}.
Vervolgens laat het ook toe om eigen gebouwde componenten te gebruiken voor consistentie en schaalbaarheid \autocite{Ranosys2023}.
OutSystems heeft ook een aantal Agile Software Development functies zoals version control, programmer collaboration en geautomatiseerde testing tools \autocite{Ranosys2023}.
Daarnaast geeft het uw volledig controle over veilige hosting door het genereren van op standaarden gebaseerde, niet-eigen code, hoogwaardige digitale oplossingen voor bedrijven \autocite{Ranosys2023}.
Het Low-Code platform heeft ook een AI om bijvoorbeeld errors te vermideren, om pre-built componenten te genereren of zoeken,en om code te analyseren en detecteren van bugs \autocite{Ranosys2023}.
Het bedrijf OutSystems laat ook bedrijven toe om de prestaties en groei van uw bedrijf visueel te monitoren en te beheren \autocite{Ranosys2023}.

\paragraph*{Voordelen}
\begin{enumerate}
    \item Het reduceert de kost van software development omdat men zeer snel applicaties kan maken \autocite{Payne2023}.
    \item Het platform ondersteund Agile Software Development \autocite{Payne2023}.
    \item Zeer goed platform voor samenwerking tussen developers \autocite{Payne2023}.
    \item Het platform is makkelijk te gebruiken \autocite{G22024OutSystems}.
\end{enumerate}


\paragraph*{Nadelen}
\begin{enumerate}
    \item OutSystems is een duur platform \autocite{G22024OutSystems}.
    \item Het heeft een gelimiteerde aantal features \autocite{G22024OutSystems}.
    \item Het platform bedraagt volgens \textcite{G22024OutSystems} een hoge leercurve.
\end{enumerate}

\paragraph{Integratie mogelijkheden}
OutSystems bevat heel wat integratie mogelijkheden zoals het connecteren met externe data bronnen, enterprise systemen, en API's \autocite{Payne2023}.
Hiervoor biedt het Low-Code platform connecties en tools om het integratieproces te vergemakkelijken en tijd te besparen \autocite{Payne2023}. Helaas
is er geen informatie gevonden over integratie met AirTable en MAKE.com.

\paragraph{Beoordelingen}
Volgens \textcite{Gartner2024} heeft OutSystems een beoordeling van 4.5/5.0, gebaseerd op 885 beoordelingen. 
Hieruit volgt dat 22\% van bedrijven met minder dan 50 miljoen dollar waarde OutSystems aanbevelen en dat 44\% van bedrijven met meer dan 50 miljoen dollar waarde tot 1 miljard OutSystems aanbevelen \autocite{Gartner2024}.
Vervolgens is de beoordeling op schaalbaarheid (4.3/5), aanpasbaarheid (4.3/5), en een goede platform voor overheid (4.2/5) de laagste beoordelingen \autocite{Gartner2024}.

\paragraph{Kostprijs}
OutSystems heeft drie verschillende abonnementen, namelijk Free, Multiple apps, en Large app portfolios \autocite{OutSystems}. Het Free abonnement is gratis en is bedoelt voor het maken van één app \autocite{OutSystems}.
Bij Free plan kan enkel je app gerund worden in development en niet in production \autocite{OutSystems}. Vervolgens kunnen er ook maar 100 eindgebruikers op de app zitten \autocite{OutSystems}.
Het Multiple apps abonnement heeft heel wat meer voordelen, maar wel tegen een prijs van 1.250 euro per maand. Deze voordelen zijn dat het zowel in development als in production modus kan gerund worden \autocite{OutSystems}.
Daarnaast heb je een uptime van 99.5\% en is er geen maximum aantal eindgebruikers \autocite{OutSystems}. Het laatste abonnement, Large app portfolio, is bedoelt voor bedrijven met meer nodige functies dan gegeven bij Multiple apps \autocite{OutSystems}. 
\subsubsection*{Microsoft PowerApps}
Microsoft PowerApps wordt gezien als één van de meeste bekenste Low-Code platformen \autocite{Nguyen} \autocite{Gupta2023}.
Microsoft PowerApps is meer dan alleen een Low-Code platform, het heeft verschillende app features, templates voor ontwerp, pre-built connectors, en tools om ontwikkelaars te helpen bij het maken van een Low-Code applicatie \autocite{Nguyen}.

\paragraph{Functies}
Microsoft PowerApps heeft een robuuste veiligheid om zowel de applicatie als de data op elke laag te beschermen \autocite{Nguyen}.
Gebruiks van de gemaakt applicatie worden beveiligd door Office 365 of Azure \autocite{Nguyen}. Daarnaast is er ook access control, app-level security, form-level security, record-level security, en field-level security  \autocite{Nguyen}.
Het Low-Code platform support ook een wijdt bereik aan verschillende apparaten van zowel Android, iOS, en Windows voor mobiele apps \autocite{Nguyen}.
Voor de web versie van de app kan het op eender welke moderne web browser draaien \autocite{Nguyen}. Dat terzijde beschikt volgens \textcite{Gupta2023} Microsoft PowerApps over unieke functies. Als eerste heb je 
streamline operations, dit is voor het traceren van werknemers kosten tot automatiseren van communicaties, analyse van data en het introduceren van AI in de bedrijfsprocessen \autocite{Gupta2023}.
Als tweede heb je snellere ontwikkeling want met Microsoft PowerApps kan je een applicatie maken in maar enkele dagen \autocite{Gupta2023}. Vervolgens is het integreren met Office 365 suite zeer eenvoudig met Microsoft PowerApps,
 dus samengevat alles van Microsoft kan geïntegreerd worden \autocite{Gupta2023}.

 \paragraph*{Voordelen}
\begin{enumerate}
    \item Een robuuste reeks aan functies \autocite{Marvin2018}.
    \item Het platform biedt heel veel integratie mogelijkheden aan \autocite{Marvin2018}.
    \item Het is een zeer makkelijk platform om te gebruiken \autocite{Marvin2018}.
\end{enumerate}


\paragraph*{Nadelen}
\begin{enumerate}
    \item Het platform kan soms traag zijn \autocite{Marvin2018}.
    \item De gebruikersinterface van PowerApps kan in het begin overweldigend zijn \autocite{Marvin2018}.
    \item Het platform heeft een redelijk steile leercurve \autocite{Marvin2018}.
\end{enumerate}

\paragraph{Integratie mogelijkheden}
Volgens \textcite{Nguyen} kan PowerApps integreren met over 400 connectors (applicaties/services). De meest bekende
ervan zijn Office 365, SQL Server en Azure,  en OpenAI \autocite{Nguyen}. Volgens het bedrijf \textcite{Microsoft2024} kan PowerApps integreren met 
AirTable, maar helaas met \textcite{MAKE.com2024a} is er geen integratie mogelijkheid.
\paragraph*{Limitaties en nadelen}
Microsoft PowerApps heeft 3 belangrijke limitaties waar bedrijven rekening mee moeten houden. 
Volgens \textcite{Gupta2023} is de eerste limitatie dat het platform gelimiteerde is op vlak van platformflexibiliteit.
Daarbij is het ook zeer lastig om het low-code platform te integreren met externe systemen want het heeft maar een paar 
third-party applicaties of services waarmee het kan integreren \autocite{Gupta2023}. Ten slotte kunnen de applicaties die gemaakt zijn met
PowerApps niet gepubliceerd worden op de Apple App Store, Google Play Store en Windows Store \autocite{Gupta2023}.
Volgens \textcite{Nguyen} is het moeilijk om complexe applicaties te maken met PowerApps, maar het is wel mogelijk om simpele applicaties te maken.
De reden hiervoor is omdat je dan waarschijnlijk high-code moet implementeren, wat betekent dat je de programmeertaal Power FX zal moeten leren \autocite{Nguyen}.
Daarnaast zijn er volgens \textcite{Nguyen} een aantal reviews die zeggen dat het pricing plan alleen voordelig is voor kleinere specifeke apps.
\paragraph{Beoordelingen}
Microsoft PowerApps heeft op \textcite{Gartner2024} een beoordeling van 4.5/5.0, gebaseerd op 295 beoordelingen.
Uit deze beoordeling zijn er 12\% van bedrijven met minder dan 50 miljoen dollar waardie die Microsoft PowerApps gebruiken en aanbevelen \autocite{Gartner2024}.
Dit is anders voor bedrijven met een waarden tussen 50 miljoen en 1 miljard waarde want daar is het 41\% \autocite{Gartner2024}.
Microsoft PowerApps scoort het slechtse voor de overheid (4.0/5) en DevOps Practices (4.2/5) \autocite{Gartner2024}.

\paragraph{Kostprijs}
Volgens \textcite{Gupta2023} heeft het Low-Code platform 3 abonnementen, namelijk de eerste is ongeveer 4,60 euro per maand per gebruiker om één applicatie te runnen.
Als tweede heb je een abonnement Premium van 18,70 euro per maand per gebruiker om meerdere applicaties te runnen en als laatste heb je een pay-as-you-go abonnement
waarbij het ongeveer 9,20 euro per actieve gebruiker per app per maand is, maar hiervoor heb je een Azure abonnement nodig \autocite{Gupta2023}.
\subsubsection*{Appian}
Volgens \textcite{Shala} is Appian Corporation een cloud-based business software bedrijf. Het bedrijf biedt een Platform as a Service (PaaS) aan voor het maken van beedrijfsapplicaties \autocite{Shala}.
Appian Corporation focust vooral op Low-Code development, Business Process, en Case Management Markets \autocite{Shala}. Het Low-Code platform Appian wordt gebruikt om bedrijven te helpen
bij het maken van applicaties en automatiseren van bedrijfsprocessen, en case management \autocite{Shala}.
\paragraph{Functies}
De hoofd functie van Appian is om heel snel een enterprise-ready applicatie te maken met een schitterende user interface \autocite{Shala}. Appian maakt het ook makkelijk 
om je middelen en mensen, technologie, data, en systemen combineren in een uniform proces voor de productiviteit te verhogen \autocite{Shala}.
Het Low-Code platform Appian doet een goede onderscheiding tussen de niet-programmeer delen en de zware processen \autocite{Marvin2017}.
Volgens \textcite{Marvin2017} is de interface en de mogelijkheden van Appian Quick App, dat een onderdeel is van het Low-Code platform, niet zo goed in vergelijking met dat van Zoho Creator of Microsoft PowerApps maar Appian is wel de enige die een echte no-code ervaring biedt.

\paragraph*{Voordelen}
\begin{enumerate}
    \item Met Appain Quick Apps heb je een echte no-code ervaring \autocite{Marvin2017}.
    \item Kan gebruikt worden om native mobiele applicaties te maken \autocite{Marvin2017}.
    \item Bevat een drag-and-drop procesontwerper \autocite{Marvin2017}.
    \item Ingebouwde team samenwerking, taakbeheer systeem, en bedrijfsnetwerkplatform \autocite{Marvin2017}.
\end{enumerate}


\paragraph*{Nadelen}
\begin{enumerate}
    \item Tegenover andere Low-Code platformen is het zeer duur \autocite{Marvin2017}.
    \item Gebruikersinterface is aan de veroudere kant \autocite{Marvin2017}.
\end{enumerate}

\paragraph{Integratie mogelijkheden}
Volgens \textcite{Marvin2017} heb je ervaring nodig in het programmeren om Appian te integreren met andere systemen. 
Helaas kan het ook niet integreren met MAKE.com \autocite{MAKE.com2024a}.

\paragraph{Beoordelingen}
Appian scoort op \textcite{Gartner2024} een beoordeling van 4.5/5.0, gebaseerd op 525 beoordelingen. Hieruit volgt dat 85\% van de reviews het aanbevelen \autocite{Gartner2024}.
Enkel 12\% van de reviews zijn kleine bedrijven dat het aanbevelen \autocite{Gartner2024}. Appian wordt volgens de recommendaties vooral gebruikt in zeer grote bedrijven met een waarde van meer dan 10 miljard dollar, dit is namelijk 28\% \autocite{Gartner2024}.
\paragraph{Kostprijs}
Helaas is er zeer weinig info gevonden over de kostprijs van Appian. Ze hebben enkel info over de verschillende abonnementen, namelijke de eerste is 
Standard, de tweede is Advandced en de laatste is Premium \autocite{Appian2024}. De prijs van deze abonnementen is niet bekend omdat dit kan verschillen per bedrijf want het is namelijk 
per gebruiker per maand per app. Het groostste verschil tussen de abonnementen is dat je bij Premium niet gelimiteerd meer bent op het aantal portals en bots \autocite{Appian2024}. Maar volgens
\textcite{Shala} kost het Standard abonnement ongeveer 7 euro per gebruiker per maand.


\subsubsection*{Wix}
Volgens \textcite{Ryan2024} is Wix de beste website builder voor 2024, dat beter was dan 16 andere platformen dat ze hebben getest. Vervolgens raden ze enkel 
Wix aan voor simpele websites en kleine bedrijven om hun merk online op te bouwen, maar niet voor zeer grote online winkels door het gebrek aan schaalbaarheid \autocite{Ryan2024}.
Wix geeft de gebruikers een goede verhouding tussen prijs en kwaliteit door de vele functies die het platform aanbiedt tegenover een eerlijke prijs  \autocite{Singleton2024}.
Bij een abonnement van Wix krijg je een website editor, verkoop tools, video hosting, een manier om afspraken te maken, en nog meer \autocite{Singleton2024}.
\paragraph{Functies}
Zoals eerder vermeld heeft Wix heel wat functies. Eén van zijn beste functies is zoekmachineoptimalisatie (SEO) tools, dit bedraagt het personaliseren van de URL en meta data \autocite{Ryan2024}.
Op Wix zijn er ook heel wat integraties die kunnen helpen met zoekmachineoptimalisatie zoals Google My Business \autocite{Ryan2024}. Vervolgens heb je een andere functie genaamd email marketing
om website eigenaar een connectie te kunnen laten leggen met zijn klanten \autocite{Ryan2024}. Deze functie bevat ook templates die je volledig kan personaliseren naar uw eigen stijl aan de hand van de drag-and-drop editor \autocite{Ryan2024}.
Met Wix kan je ook je website synschroniseren met je social media accounts zoals Facebook, LinkedIn, en X hiermee kan je dan via uw Wix Dashboard nieuwe social media posts maken en publiceren \autocite{Ryan2024}.
Meestal moet een website meertalig zijn, met Wix kan je dit heel eenvoudig realiseren want het platform ondersteund 180 talen en vertaald automatisch uw website \autocite{Ryan2024}. Het Low-Code platform biedt ook een afspraak maker aan, waardoor je geen 
gebruik moet maken van een externe afspraak maker zoals Google Calendar \autocite{Singleton2024}. 
\paragraph*{Voordelen}
\begin{enumerate}
    \item Wix bevat een drag-and-drop editor \autocite{Ryan2024}.
    \item Maar liefts 900+ professionele ontworpen templates voor uw website \autocite{Ryan2024} \autocite{Singleton2024}.
    \item Beschikt over AI tools om u te helpen bij het maken van uw website \autocite{Ryan2024}.
    \item Beschikt over een ingebouwde email marketing tool \autocite{Singleton2024}.
\end{enumerate}


\paragraph*{Nadelen}
\begin{enumerate}
    \item Ongelimiteerde opslag is enkel verkrijgbaar bij het duurste abonnement \autocite{Ryan2024}.
    \item De verschillende abonnementen van Wix zijn wat duurder dan andere website builders \autocite{Ryan2024}.
    \item Door de vele functies kan het platform overweldigend zijn voor beginners \autocite{Ryan2024}.
    \item Templates zijn niet volledig responsive \autocite{Singleton2024}.
\end{enumerate}

\paragraph{Integratie mogelijkheden}
Volgens \textcite{Singleton2024} bevat Wix een soort van app store genaamd Wix App Market. 
Hierin kan je maar liefst 821 apps vinden, van het bedrijf zelf maar ook van andere bedrijven, die je kan integreren met uw website \autocite{Singleton2024}.
Volgens \textcite{MAKE.com2024a} kan je Wix integreren met MAKE.com.

\paragraph{Kostprijs}
De prijzen variëren van €11 tot €149 per maand (jaarlijks gefactureerd) \autocite{Wix2024} {Ryan2024}. 
Er bestaan 4 verschillende abonnementen, namelijk Light, Core, Business, en Business Elite \autocite{Wix2024}.
Het Light abonnement is het goedkoopste abonnement en kost €11 per maand \autocite{Wix2024}. Light wordt vooral gebruikt voor enkele informele websites \autocite{Ryan2024}.
Het Core abonnement kost €22 per maand \autocite{Wix2024}. Dit abonnement is vooral voor als jouw website betalingen moet accepteren en online verkoopt \autocite{Ryan2024}.
Het Business abonnement kost €34 per maand \autocite{Wix2024}. Dit abonnement is vooral voor groeiende kleine bedrijven \autocite{Ryan2024}.
Het Business Elite abonnement kost €149 per maand \autocite{Wix2024}. Dit abonnement is voor succesvolle online stores \autocite{Ryan2024}.
\subsubsection*{Bubble}
Bubble is een No-Code development platform dat je toelaat om web applicaties te ontwerpen, maken en te hosten zonder enige lijntje code te schrijven \autocite{Sharma2022}.
Je hebt er rechtstreeks toegang op je browser via Bubble.io, zoals de meeste LCNC platformen \autocite{Minor2022}. Het maken van een applicatie in Bubble kan zeer ruim zijn, van een simpele blog tot een CRM systeem \autocite{Sharma2022}.
Het platform kan gebruikt worden als zeer interactieve website builder met SEO tools en analytics, maar zou ook kunnen gebruikt worden om een 2D game te maken zoals Wordle \autocite{Minor2022}.

\paragraph{Functies}
Dit No-Code platform maakt het mogelijk om elke soort web applicatie te maken met geen enkele code te schrijven \autocite{Bubble2024b}.
Dit kan gaan tot een interactieve en multi-user apps voor desktop en mobile web browsers, daarnaast bevat het alles om een site zoals Facebook of Airbnb te maken \autocite{Bubble2024b}.
Bubble maakt het zeer makelijk om je applicatie te ontwerpen en te maken met hun drag-and-drop editor \autocite{Bubble2024b}.
Vervolgens zorgt het platform zelf voor veilig opstellen en hosten van de applicatie, waarbij er geen limiet is op het aantaal gebruikers, volume van verkeer of data opslag \autocite{Bubble2024b}.
Daarnaast kan je ook in Bubble samenwerken in real-time, wat betekent dat je de wijzigingen van je collega's direct kan zien \autocite{Bubble2024b}.
\paragraph*{Voordelen}
\begin{enumerate}
    \item Het is een visuele programming language \autocite{Minor2022}.
    \item Geen enkele ervaring van programmeren nodig \autocite{Minor2022}.
    \item Het heeft toturials over verschillende onderwerpen binnen het platfrom \autocite{Minor2022}.
    \item Bevat een community marketplace voor templates en plugins \autocite{Minor2022}.
\end{enumerate}


\paragraph*{Nadelen}
\begin{enumerate}
    \item Kan snel duur worden op langere termijn \autocite{Minor2022}.
    \item Door het gebrek aan coderen kan het platform beperkt zijn \autocite{Minor2022}.
    \item Met Bubble.io kan je geen native mobiele applicaties maken, dus rechtstreeks publiceren op bijvoorbeeld Google Play Store is niet mogelijk
    maar hiervoor zijn workarounds voor zoals gebruikmaken van een third-party service \autocite{Sharma2022}.
    \item Je kan geen python of andere scripts uitvoeren \autocite{Sharma2022}.
\end{enumerate}

\paragraph{Integratie mogelijkheden}
Er is geen exacte nummer van hoeveel integraties mogelijk is met Bubble.io maar op de officiële website \textcite{Bubble2024a} 
is het mogelijk om zowel met MAKE.com en AirTable te integreren. Daarnaast kan het ook integreren met Google Maps, Figma, Zapier, Discord, Google Drive en nog veel meer \autocite{Bubble2024a}.

\paragraph{Beoordelingen}
Op \textcite{Gartner2024} heeft Bubble.io een beoordeling van 4.4/5.0, gebaseerd op 33 beoordelingen. Volgens \textcite{Gartner2024}
wordt het platform vooral gebruikt in bedrijven met een waarde tussen 50 miljoen en 1 miljard dollar en minder dan 50 miljoen.
Bubble scoort het slechtste op een goede platform voor overheid (4.0/5), DevOps Practices (4.1/5), en security \& QoS  (4.1/5) \autocite{Gartner2024}. Maar het scoort wel
het beste op User Experience Design (4.8/5)  \autocite{Gartner2024}.

\paragraph{Kostprijs}
Bubble.io heeft 5 verschillende abonnementen, namelijk Free, Starter, Growth, Team, en Enterprise \autocite{Bubble2024}.
Het Free abonnement is gratis en is bedoelt voor projecten te maken die nog in aanbouw zijn \autocite{Bubble2024}.
Het Starter abonnement kost ongeveer 27 euro per maand (\$ 29) en is bedoelt voor Minimum Viable Products (MVPs) en simpele tools \autocite{Bubble2024}.
Het Growth abonnement kost ongeveer 111 euro per maand (\$ 119). Het is goed voor consumer projects met complexe functionaliteiten \autocite{Bubble2024}.
Het Team abonnement kost ongeveer 327 euro per maand (\$ 349) en is goed voor schaalbare projecten met hoge gebruiksfrequentie \autocite{Bubble2024}. Als laatste heb je dan nog
het Enteprise abonnement, waarvan de prijs niet bekend is, en is bedoelt voor bedrijven met specifieke requirements \autocite{Bubble2024}.


\section*{Vergelijkende analyse}
\label{sec:vergelijkende-analyse}
In deze fase vergelijken we de alternatieve platform met Softr en Stacker aan de hand van verschillende testen. 
Deze soort testen hangen af van de requirements analyse.


\section*{Proof of Concept}
\label{sec:proof-of-concept}
\subsection*{Ontwikkelingsfase}
\subsection*{Uitvoer van de Proof of Concept}
\subsection*{Resultaten van de Proof of Concept}

\section*{Conclusie}

