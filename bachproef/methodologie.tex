%%=============================================================================
%% Methodologie
%%=============================================================================

\chapter{\IfLanguageName{dutch}{Methodologie}{Methodology}}%
\label{ch:methodologie}

%% TODO: In dit hoofstuk geef je een korte toelichting over hoe je te werk bent
%% gegaan. Verdeel je onderzoek in grote fasen, en licht in elke fase toe wat
%% de doelstelling was, welke deliverables daar uit gekomen zijn, en welke
%% onderzoeksmethoden je daarbij toegepast hebt. Verantwoord waarom je
%% op deze manier te werk gegaan bent.
%% 
%% Voorbeelden van zulke fasen zijn: literatuurstudie, opstellen van een
%% requirements-analyse, opstellen long-list (bij vergelijkende studie),
%% selectie van geschikte tools (bij vergelijkende studie, "short-list"),
%% opzetten testopstelling/PoC, uitvoeren testen en verzamelen
%% van resultaten, analyse van resultaten, ...
%%
%% !!!!! LET OP !!!!!
%%
%% Het is uitdrukkelijk NIET de bedoeling dat je het grootste deel van de corpus
%% van je bachelorproef in dit hoofstuk verwerkt! Dit hoofdstuk is eerder een
%% kort overzicht van je plan van aanpak.
%%
%% Maak voor elke fase (behalve het literatuuronderzoek) een NIEUW HOOFDSTUK aan
%% en geef het een gepaste titel.

\section*{Requirements analyse}
\label{sec:requirements-analyse}
Voor de onderzoeksvraag te beantwoorden is het belangrijk om de requirements te analyseren. De opgestelde requirements zullen de basis vormen voor 
de evaluatie, de vergelijkende analyse en de proof of concept. Deze vereisten zullen aan de hand van verschillende bronnen worden opgesteld. Als eerste zal 
er in de voorgaande literatuurstudie worden gezocht naar de belangrijkste vereisten. Vervolgens zal er een korte vragenlijst worden opgesteld die zal worden 
beantwoord door mijn co-promotor en een collega die al twee jaar ervaring heeft met een Low-Code platform, genaamd FlutterFlow. De vragenlijst zal bestaan uit
twee vragen, namelijk welke vereisten zijn volgens u belangrijk bij het kiezen van een Low-Code platform en rangshik deze vereisten van belangrijk naar minder belangrijk. Op deze
manier kan er een duidelijke beeld worden gevormd van waar er rekening mee moet worden gehouden bij het evalueren van de Low-Code en/of No-Code platformen.
\\
\\
In de literatuurstudie kwam er een aantal criteria's naar boven die impact kunnen hebben op de keuze van een Low-Code en/of No-Code platform.
Om hiervoor een eenvoudig overzicht te krijgen, zal er hieronder een tabel worden opgesteld met de volgende velden; Criteria, Reden/Relevantie, Bron. 
Ter info, de criteria's die in de tabel zullen worden opgenomen staan niet vast en zijn meer bedoelt als een leidraad voor de requirements analyse.
\\
\\


\begin{longtable}{lp{4.4cm}p{3.4cm}}
    \caption{Criteria lijst a.d.h.v. voorgaande literatuurstudie} \label{crouch} \\
    \toprule
    \textbf{Criteria} & \textbf{Reden/Relevantie} & \textbf{Bron} \\
    \midrule
    \endfirsthead

    % Repeat the headers on the next page
    \multicolumn{3}{c}{{\bfseries \tablename\ \thetable{} -- vervolg van de vorige pagina}} \\
    \toprule
    \textbf{Criteria} & \textbf{Reden/Relevantie} & \textbf{Bron} \\
    \midrule
    \endhead

    % Footer at the end of each page, except the last page
    \midrule
    \multicolumn{3}{r}{{Vervolg op volgende pagina}} \\
    \endfoot

    % Footer at the end of the table
    \bottomrule
    \endlastfoot

    % Table content
    Snelheid van het platform & Om snel en eenvoudig ingewikkelde applicaties te ontwikkelen. & Voordelen: snelheid \\
    Snelheid van applicatieontwikkeling & Zorgt dat er meerdere applicaties in een korte tijd kunnen worden ontwikkeld. Vervolgens moeten bedrijven ook steeds snel en flexibel kunnen inspelen op de veranderende markt. &  \hyperref[subsec:snelheid]{Voordelen: snelheid} \& \hyperref[sec:reden-tot-gebruik]{Reden tot gebruik: Tijdrovend} \\
    Leercurve & Heel wat mensen laten zich afschrikken tot het programmeren door de complexiteit van de programmeertalen. Hierdoor is het belangrijk dat het snel en eenvoudig is om te leren. & \hyperref[sec:reden-tot-gebruik]{Reden tot gebruik: Beperkt aantal programmeurs} \\
    Updatebeleid \& Moderniteit & Omdat technologie vaak verandert is de impact van een updatebeleid en moderniteit van het platform belangrijk op de lange termijn. & \hyperref[sec:reden-tot-gebruik]{Reden tot gebruik: Technologische turbulentie} \\
    Kostprijs & Omdat Quivvy Solutions BV een start-up is, zijn er beperkte financiële middelen. Waardoor het belangrijk is dat de kostprijs niet over het budget gaat. & \hyperref[sec:reden-tot-gebruik]{Reden tot gebruik: Hoge kosten} \& \hyperref[sec:lcnc-bedrijven]{LCNC binnen bedrijven} \\
    Klantentevredenheid & Als klein bedrijf is het belangrijk dat de klanten tevreden zijn over de applicaties die worden ontwikkeld. Daarbij moet de eindklant van het bedrijf ook zelf kunnen werken met het LCNC platform, waardoor de klantentevredenheid ook een belangrijk criterium is. &  \hyperref[sec:reden-tot-gebruik]{Reden tot gebruik: Klantentevredenheid} \\
    Veiligheid & Om ervoor te zorgen dat zowel het bedrijf Quivvy Solutions BV als hun eindklant zo weinig mogelijk schade kan krijgen door het gebruikte platform. &  \hyperref[subsec:veiligheid]{Voordelen: Veiligheid} \\
    Herstelbeheer \& Back-up & Wanneer er zich een ramp afspeelt moet de gegevens zo snel mogelijk kunnen hersteld worden. Hierdoor is het regelmatig back-ups relevant. & \hyperref[subsec:cloud-based-lcnc]{Cloud-based LCNC : Dataherstel} \\
    Integratie mogelijkheden & Hoe meer Integratie mogelijkheden dat het LCNC platform heeft, hoe beter. Flexibiliteit in een platform heeft een impact op de keuze van het platform. & \hyperref[subsec:beperkte-flexibiliteit]{Nadelen: Beperkte Flexibiliteit} \\
    Platformflexibiliteit \& Aanpasbaarheid & Mogelijkheid om high code te implementeren wanneer nodig en de mogelijke functionaliteiten. & \hyperref[subsec:beperkte-flexibiliteit]{Nadelen: Beperkte Flexibiliteit} \\
    Schaalbaarheid & Een applicatie in LCNC platformen zou gemakkelijk uitbreidbaar moeten zijn en wanneer mogelijk meer tegelijk gebruikers toelaten. & \hyperref[subsec:gelimiteerde-schaalbaarheid]{Nadelen: Gelimiteerde schaalbaarheid} \\
    Documentatiekwaliteit &  Documentatie kan leiden tot een oplossing voor het beter verstaan van bijvoorbeeld het opslaan van data. & \hyperref[subsec:lcnc-binnen-agile]{LCNC in Software Development Life Cycle: Application Design} \\
    Test- en debugmogelijkheden  & LCNC platformen verwaarlozen vaak testen, maar dit blijft nogsteeds een belangrijke fase in de Software Development Life Cycle.  &  \hyperref[subsec:lcnc-binnen-agile]{LCNC in Software Development Life Cycle: Testing}\\
\end{longtable}








\section*{Evaluatie en Selectie van alternatieven}
\label{sec:evaluatie-en-selectie-van-alternatieven}

\section*{Vergelijkende analyse}
\label{sec:vergelijkende-analyse}

\section*{Proof of Concept}
\label{sec:proof-of-concept}
\subsection*{Ontwikkelingsfase}
\subsection*{Uitvoer van de Proof of Concept}
\subsection*{Resultaten van de Proof of Concept}

\section*{Conclusie}

