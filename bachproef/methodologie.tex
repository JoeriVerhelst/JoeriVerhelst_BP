%%=============================================================================
%% Methodologie
%%=============================================================================

\chapter{\IfLanguageName{dutch}{Methodologie}{Methodology}}%
\label{ch:methodologie}

%% TODO: In dit hoofstuk geef je een korte toelichting over hoe je te werk bent
%% gegaan. Verdeel je onderzoek in grote fasen, en licht in elke fase toe wat
%% de doelstelling was, welke deliverables daar uit gekomen zijn, en welke
%% onderzoeksmethoden je daarbij toegepast hebt. Verantwoord waarom je
%% op deze manier te werk gegaan bent.
%% 
%% Voorbeelden van zulke fasen zijn: literatuurstudie, opstellen van een
%% requirements-analyse, opstellen long-list (bij vergelijkende studie),
%% selectie van geschikte tools (bij vergelijkende studie, "short-list"),
%% opzetten testopstelling/PoC, uitvoeren testen en verzamelen
%% van resultaten, analyse van resultaten, ...
%%
%% !!!!! LET OP !!!!!
%%
%% Het is uitdrukkelijk NIET de bedoeling dat je het grootste deel van de corpus
%% van je bachelorproef in dit hoofstuk verwerkt! Dit hoofdstuk is eerder een
%% kort overzicht van je plan van aanpak.
%%
%% Maak voor elke fase (behalve het literatuuronderzoek) een NIEUW HOOFDSTUK aan
%% en geef het een gepaste titel.

\section*{Requirements analyse}
\label{sec:requirements-analyse}
Voor de onderzoeksvraag te beantwoorden is het belangrijk om de requirements te analyseren. De opgestelde requirements zullen de basis vormen voor 
de evaluatie, de vergelijkende analyse en de proof of concept. Deze vereisten zullen aan de hand van verschillende bronnen worden opgesteld. Als eerste zal 
er in de voorgaande literatuurstudie worden gezocht naar de belangrijkste vereisten. Vervolgens zal er een vraag worden gesteld die zal worden 
beantwoord door mijn co-promotor. De vraag bestaat uit, namelijk welke vereisten zijn volgens u belangrijk bij het kiezen van een Low-Code platform en rangshik deze vereisten van belangrijk naar minder belangrijk. Op deze
manier kan er een duidelijke beeld worden gevormd van waar er rekening mee moet worden gehouden bij het evalueren van de Low-Code en/of No-Code platformen.
\\
\\
In de literatuurstudie kwam er een aantal criteria's naar boven die impact kunnen hebben op de keuze van een Low-Code en/of No-Code platform.
Om hiervoor een eenvoudig overzicht te krijgen, zal er hieronder een tabel worden opgesteld met de volgende velden; Criteria, Reden/Relevantie, Bron. 
Ter info, de criteria's die in de tabel zullen worden opgenomen staan niet vast en zijn meer bedoelt als een leidraad voor de requirements analyse.
\\
\\


\begin{longtable}{lp{4.4cm}p{3.4cm}}
    \caption{Criteria lijst a.d.h.v. voorgaande literatuurstudie} \label{crouch} \\
    \toprule
    \textbf{Criteria} & \textbf{Reden/Relevantie} & \textbf{Bron} \\
    \midrule
    \endfirsthead

    % Repeat the headers on the next page
    \multicolumn{3}{c}{{\bfseries \tablename\ \thetable{} -- vervolg van de vorige pagina}} \\
    \toprule
    \textbf{Criteria} & \textbf{Reden/Relevantie} & \textbf{Bron} \\
    \midrule
    \endhead

    % Footer at the end of each page, except the last page
    \midrule
    \multicolumn{3}{r}{{Vervolg op volgende pagina}} \\
    \endfoot

    % Footer at the end of the table
    \bottomrule
    \endlastfoot

    % Table content
    Snelheid van het platform & Om snel en eenvoudig ingewikkelde applicaties te ontwikkelen. & Voordelen: snelheid \\
    Snelheid van applicatieontwikkeling & Zorgt dat er meerdere applicaties in een korte tijd kunnen worden ontwikkeld. Vervolgens moeten bedrijven ook steeds snel en flexibel kunnen inspelen op de veranderende markt. &  \hyperref[subsec:snelheid]{Voordelen: snelheid} \& \hyperref[sec:reden-tot-gebruik]{Reden tot gebruik: Tijdrovend} \\
    Leercurve & Heel wat mensen laten zich afschrikken tot het programmeren door de complexiteit van de programmeertalen. Hierdoor is het belangrijk dat het snel en eenvoudig is om te leren. & \hyperref[sec:reden-tot-gebruik]{Reden tot gebruik: Beperkt aantal programmeurs} \\
    Updatebeleid \& Moderniteit & Omdat technologie vaak verandert is de impact van een updatebeleid en moderniteit van het platform belangrijk op de lange termijn. & \hyperref[sec:reden-tot-gebruik]{Reden tot gebruik: Technologische turbulentie} \\
    Kostprijs & Omdat Quivvy Solutions BV een start-up is, zijn er beperkte financiële middelen. Waardoor het belangrijk is dat de kostprijs niet over het budget gaat. & \hyperref[sec:reden-tot-gebruik]{Reden tot gebruik: Hoge kosten} \& \hyperref[sec:lcnc-bedrijven]{LCNC binnen bedrijven} \\
    Klantentevredenheid & Als klein bedrijf is het belangrijk dat de klanten tevreden zijn over de applicaties die worden ontwikkeld. Daarbij moet de eindklant van het bedrijf ook zelf kunnen werken met het LCNC platform, waardoor de klantentevredenheid ook een belangrijk criterium is. &  \hyperref[sec:reden-tot-gebruik]{Reden tot gebruik: Klantentevredenheid} \\
    Veiligheid & Om ervoor te zorgen dat zowel het bedrijf Quivvy Solutions BV als hun eindklant zo weinig mogelijk schade kan krijgen door het gebruikte platform. &  \hyperref[subsec:veiligheid]{Voordelen: Veiligheid} \\
    Herstelbeheer \& Back-up & Wanneer er zich een ramp afspeelt moet de gegevens zo snel mogelijk kunnen hersteld worden. Hierdoor is het regelmatig back-ups relevant. & \hyperref[subsec:cloud-based-lcnc]{Cloud-based LCNC : Dataherstel} \\
    Integratie mogelijkheden & Hoe meer Integratie mogelijkheden dat het LCNC platform heeft, hoe beter. Flexibiliteit in een platform heeft een impact op de keuze van het platform. & \hyperref[subsec:beperkte-flexibiliteit]{Nadelen: Beperkte Flexibiliteit} \\
    Platformflexibiliteit \& Aanpasbaarheid & Mogelijkheid om high code te implementeren wanneer nodig en de mogelijke functionaliteiten. & \hyperref[subsec:beperkte-flexibiliteit]{Nadelen: Beperkte Flexibiliteit} \\
    Schaalbaarheid & Een applicatie in LCNC platformen zou gemakkelijk uitbreidbaar moeten zijn en wanneer mogelijk meer tegelijk gebruikers toelaten. & \hyperref[subsec:gelimiteerde-schaalbaarheid]{Nadelen: Gelimiteerde schaalbaarheid} \\
    Documentatiekwaliteit &  Documentatie kan leiden tot een oplossing voor het beter verstaan van bijvoorbeeld het opslaan van data. & \hyperref[subsec:lcnc-binnen-agile]{LCNC in Software Development Life Cycle: Application Design} \\
    Test- en debugmogelijkheden  & LCNC platformen verwaarlozen vaak testen, maar dit blijft nogsteeds een belangrijke fase in de Software Development Life Cycle.  &  \hyperref[subsec:lcnc-binnen-agile]{LCNC in Software Development Life Cycle: Testing}\\
    \\\endline
\end{longtable}
Bij het antwoord van de copromoter kwam er ook andere criteria's naar boven. Deze bestaan uit de volgende criteria's;
gebruiksvriendelijkheid, integratie met AirTable, en integratie met MAKE.com . Vervolgens heb ik samen met mijn copromoter overlegd welke criteria's de belangrijkste zijn en welke minder belangrijk zijn.
Hieruit volgt een lijst van requirements met een score op 10, waarbij 10 het belangrijkste is en 1 het minst belangrijk.

\begin{table}[H]
    \centering
    \caption{Requirements score}
    \begin{tabular}{llc}
    \toprule
    Criteria & Score \\
    \midrule
    Snelheid van het platform & 10 \\
    Herstelbeheer & 10 \\
    Veiligheid & 9 \\
    Snelheid van applicatieontwikkeling & 8 \\
    Integratie mogelijkheden & 8 \\
    Platformflexibiliteit (bv. high-code kunnen implementeren) & 8 \\
    Schaalbaarheid & 8 \\
    Gebruiksvriendelijkheid & 8 \\
    Integratie met AirTable en/of MAKE.com & 8 \\
    Kostprijs & 7.5 \\
    Updatebeleid & 7.5 \\
    Klantentevredenheid & 7.5 \\
    Test- en debugmogelijkheden & 7.5 \\
    Leercurve & 6 \\
    Documentatiekwaliteit & 6 \\
   
    \bottomrule
 \end{tabular}
\end{table}

\subsubsection*{Officiële requirements}
De officiële requirements bevat de criteria's die een score hebben van 8 of hoger, met uitzondering van kostprijs en updatebeleid. De reden hiervoor is omdat de kostprijs toch wel belangrijk
is voor een start-up en het updatebeleid is belangrijk voor de lange termijn. Hieruit volgt dat de klantentevredenheid, test- en debugmogelijkheden, documentatiekwaliteit en leercurve niet in de officiële requirements zullen worden opgenomen.
Een mogelijke reden waarom leercurve en documentatiekwaliteit niet belangrijk is omdat er hedendaags veel bronnen zijn die kunnen helpen bij het leren van een platform en de documentatiekwaliteit is vaak goed bij de meeste platformen, zo niet dan zijn er 
toturials die meestal kunnen helpen bij een bepaalde probleem.


%% TODO: In dit hoofstuk geef je een korte toelichting over hoe je te werk bent
%% gegaan. Verdeel je onderzoek in grote fasen, en licht in elke fase toe wat
%% de doelstelling was, welke deliverables daar uit gekomen zijn, en welke
%% onderzoeksmethoden je daarbij toegepast hebt. Verantwoord waarom je
%% op deze manier te werk gegaan bent.

\section*{Evaluatie en Selectie van alternatieven}
\label{sec:evaluatie-en-selectie-van-alternatieven}
Bij deze fase zal er onderzoek gedaan worden naar verschillende Low-Code en/of No-Code platformen. Hierbij zal 
er een online onderzoek gebeuren waarbij men bronnen zal gebruiken zoals officiële websites, reviews, en technologieblogs.
Vervolgens zal er per platform de voordelen, nadelen en beschrijvende informatie worden opgenomen. Daarna komt de belangrijkste
eigenschappen van een platform terecht in een tabel waarbij de criteria's van de requirements analyse zullen worden opgenomen. Hieruit volgt een
selectie van een alternatief die verder zal worden geanalyseerd.
\subsection*{Alternatieve platformen}
\label{subsec:alternatieve-platformen}

\subsubsection*{Zoho Creator}
Zoho Creator is een Low-Code platform, wat ook een gemakkelijke platform is om te gebruiken, dat werknemers van bedrijven maar ook mensen zonder programmeerervaring toelaat
om eenvoudig krachtige bedrijf applicaties te ontwikkelen \autocite{Computer2022}. Dit platform is gemaakt door Zoho Corporation. \textcite{ZohoCorporation2024a} gelooft er in dat software de ultieme tool is voor zowel de handen als de hersenen.
Zoho Corporation is een bedrijf dat vooral focust op het verder ontwikkelen van hun product, zoals Zoho Creator, en hun klanten support \autocite{ZohoCorporation2024a}. Dit toont aan dat Zoho Corporation
hun updatebeleid volgens hun website goed is. Volgens \textcite{ZohoCorporation2024a} biedt het bedrijf meer dan alleen een product, maar zoals het bedrijf het noemt 'the operating system for business' \autocite{ZohoCorporation2024a}.
Dit bevat maar liefst 55 integreerde applicaties voor zowel mobile als web voor elke bedrijfsnood \autocite{ZohoCorporation2024a}.
\\
\\
\textbf{Functies van Zoho Creator}
In 2022 was er nog geen enkele Low-Code platform die toeliet om business gebruikers en IT een end to end business oplossingen te maken \autocite{Computer2022}.
De Zoho Creator platform zorgt ervoor dat je applicatie development, business intelligence en analytics, integraties, en automatiseringen kan doen in één platform, wat men ook een end to end business oplossing noemen \autocite{Computer2022}.
Dit geeft verschillende voordelen voor het bedrijf dat dit platform gebruikt zoals het verhogen van veiligheid maar ook omdat het platform een end to end business oplossing is, is het ook mogelijk om uniforme oplossingen te maken via de low-code platform,
waardoor elke werknemer de mogelijkheid krijgt om te innoveren \autocite{Computer2022}. Volgens \textcite{Computer2022} zorgt Zoho Creator er ook voor dat de business gebruikers snel een schaalbare low-code oplossingen kunnen maken zoals het maken van een app of automatisering.
Het ontwikkelen van een low-code oplossing is zelfs 10 keer sneller dan eender welke andere oplossing op de markt \autocite{Computer2022}. Het platform heeft ook hun eigen AI, genaamd Zia, die de gebruiker helpt bij het maken van applicaties \autocite{Computer2022}. Deze AI
kan ontwikkelaars helpen bij het importeren van data door deze te analyseren en te structureren \autocite{Computer2022}. Daarnaast kan het ook relaties tussen data detecteren \autocite{Computer2022}. 
Zoho Creator heeft de mogelijkheid om zelf code te schrijven, wat een groot voordeel is voor bedrijven die toch nog high-code willen implementeren \autocite{Computer2022}. 
De code kan geschreven worden in Deluge, Java of Node.js en zijn ook herbruikbaar om duplicatie te voorkomen \autocite{Computer2022}.
De gebruikers van het platform Zoho Creator kunnen ook zien hoe goed integraties aan het werken zijn aan de hand van de 'Integration Status Dashboard', wat helpt om te anlayseren waar er problemen zijn \autocite{Computer2022}.
\\ %TODO (nu enkel info van Express Computer, moet nog andere bronnen zoeken)
\\
\textbf{Kostprijs}
%per abonnement; herstelbeheer, integratie, automatiseringen (ai)
Zoho Corporation heeft verschillende soorten abonnementen voor hun platform, deze zijn Standard, Professional, Enteprise, en Flex \autocite{ZohoCorporation2024}.
De standaard abonnementen bedraagt 12 euro, per gebruiker, per maand. Hiermee kan je maar maximaal 1 business applicatie maken \autocite{ZohoCorporation2024}.
Daarbij wordt ook je data opgeslagen en geback-upt in de cloud en 20 AI models maken om te trainen voor specifieke taken \autocite{ZohoCorporation2024}. Op vlak van integratie
kan je maar maximum 5 data bronnen integreren en 5 eigen connectie maken met systemen dat niet ondersteund worden door Zoho Creator \autocite{ZohoCorporation2024}. Vervolgens heb je dan het
abonnement Professional, wat 30 euro per gebruiker, per maand bedraagt. Hiermee kan je unlimited business applicaties maken en 100 AI models trainen \autocite{ZohoCorporation2024}. 
Daarbij kan je 15 data bronnen integreren en 10 eigen connecties maken \autocite{ZohoCorporation2024}. Daarnaast heb je dan Enteprise abonnement, wat 37 euro per gebruiker, per maand bedraagt.
Het voordeel van dit abonnement is dat je integratie mogelijkheid hebt met 650+ business apps, 30 data bronnen integreren, en 20 eigen connecties maken \autocite{ZohoCorporation2024}. Vervolgens heb je ook
Business Intelligence and Analytics bij Enteprise om zeer eenvoudig en verschillende anlaysis te maken \autocite{ZohoCorporation2024}. Het handige is dat je ook de mogelijkheid hebt om hun te contacteren in het geval 
van gepersonaliseerde requirements via het Flex abonnement \autocite{ZohoCorporation2024}.

\subsubsection*{OutSystems}
\subsubsection*{Microsoft PowerApps}
\subsubsection*{Appian}
\subsubsection*{Wix}
\subsubsection*{Bubble}


\section*{Vergelijkende analyse}
\label{sec:vergelijkende-analyse}
In deze fase vergelijken we de alternatieve platform met Softr en Stacker aan de hand van verschillende testen. 
Deze soort testen hangen af van de requirements analyse.


\section*{Proof of Concept}
\label{sec:proof-of-concept}
\subsection*{Ontwikkelingsfase}
\subsection*{Uitvoer van de Proof of Concept}
\subsection*{Resultaten van de Proof of Concept}

\section*{Conclusie}

