%%=============================================================================
%% Inleiding
%%=============================================================================

\chapter{\IfLanguageName{dutch}{Inleiding}{Introduction}}%
\label{ch:inleiding}

%De inleiding moet de lezer net genoeg informatie verschaffen om het onderwerp te begrijpen en in te zien waarom de onderzoeksvraag de moeite waard is om te onderzoeken. In de inleiding ga je literatuurverwijzingen beperken, zodat de tekst vlot leesbaar blijft. Je kan de inleiding verder onderverdelen in secties als dit de tekst verduidelijkt. Zaken die aan bod kunnen komen in de inleiding~\autocite{Pollefliet2011}:

%\begin{itemize}
 % \item context, achtergrond
 % \item afbakenen van het onderwerp
 % \item verantwoording van het onderwerp, methodologie
 % \item probleemstelling
 % \item onderzoeksdoelstelling
 % \item onderzoeksvraag
 % \item \ldots
%\end{itemize}

Softwarebedrijven merken op dat projecten telkens over het budget gaan. Daarbij doet het project meestal niet wat de eindklant verwacht en vervolgens doet het opgeleverde product niet wat het zou moeten doen ~\autocite{Moskal_2021}.
Maar volgens ~\textcite{Moskal_2021} zijn deze problemen niet enkel op te merken in de softwarewereld maar ook in andere categorieën binnen de IT-sector. 
Daarnaast moeten softwarebedrijven ook rekening houden dat er in deze tijdsperiode meer opslag en verwerking van data is.
Volgens ~\textcite{Moskal_2021} en ~\textcite{Parviainen_2022} brengt het verwerken en opslaan van data veranderingen in de business mee, 
doordat bedrijven zich meer digitaliseren. Maar digitalisering betekent dat men producten of diensten moet omzetten naar een digitaal product 
of dienst. Het kan ook zijn dat men software producten moet kopen om de business processen te automatiseren ~\autocite{Moskal_2021}. 
Het verkrijgen van een competitief voordeel kan men bereiken door een specifiek ontworpen en ontwikkelde softwareoplossing die voldoet aan de behoeften van de eindklant.
Maar volgens ~\textcite{Moskal_2021} is een toegewijde IT-oplossing zo duur dat bedrijven dit niet kunnen veroorloven. Dit zet No-Code en Low-Code platformen in de spotlight
door de functies zoals het integreren van data via tools en een drag-and-drop systeem om de ontwikkeling te versnellen \autocite{Kulkarni_2021}. Met deze platformen kan je software ontwikkelen door gebruik van een minimale code
of zonder code door middel van het drag-and-drop systeem.

\section{\IfLanguageName{dutch}{Probleemstelling}{Problem Statement}}%
\label{sec:probleemstelling}
Als jong en klein bedrijf is Quivvy Solutions constant opzoek naar de beste en snelste platformen om hun software te ontwikkelen. 
Doordat er constant nieuwe technologieën bijkomen en veranderen heeft Quivvy Solutions zich beperkt tot het ontwikkelen van software 
via low-code en/of no-code platformen. Hieruit volgt dat het bedrijf nog niet heeft kunnen bepalen welke platform nu het beste past bij hun.
%Uit je probleemstelling moet duidelijk zijn dat je onderzoek een meerwaarde heeft voor een concrete doelgroep. De doelgroep moet goed gedefinieerd en afgelijnd zijn. Doelgroepen als ``bedrijven,'' ``KMO's'', systeembeheerders, enz.~zijn nog te vaag. Als je een lijstje kan maken van de personen/organisaties die een meerwaarde zullen vinden in deze bachelorproef (dit is eigenlijk je steekproefkader), dan is dat een indicatie dat de doelgroep goed gedefinieerd is. Dit kan een enkel bedrijf zijn of zelfs één persoon (je co-promotor/opdrachtgever).

\section{\IfLanguageName{dutch}{Onderzoeksvraag}{Research question}}%
\label{sec:onderzoeksvraag}
Met welke Low-Code en/of No-Code platform kan Quivvy Solutions zowel mobile als web toepassingen creëren, die ook kunnen integreren met
zowel MAKE.com als Airtable?
%Wees zo concreet mogelijk bij het formuleren van je onderzoeksvraag. Een onderzoeksvraag is trouwens iets waar nog niemand op dit moment een antwoord heeft (voor zover je kan nagaan). Het opzoeken van bestaande informatie (bv. ``welke tools bestaan er voor deze toepassing?'') is dus geen onderzoeksvraag. Je kan de onderzoeksvraag verder specifiëren in deelvragen. Bv.~als je onderzoek gaat over performantiemetingen, dan 

\section{\IfLanguageName{dutch}{Onderzoeksdoelstelling}{Research objective}}%
\label{sec:onderzoeksdoelstelling}
Voor de onderzoeksvraag zo goed mogelijk te kunnen beantwoorden zal er een uitgebreide onderzoek plaatsnemen waarbij het begint met het opstellen 
van de vereisten waaraan een platform moet voldaan. Vervolgens zal er een analyse gedaan worden tussen verschillende alternatieven om daaruit nog een platform 
te analyseren. Daarna wordt er voor Stacker, Softr en het alternatief platform een vergelijkende analyse gedaan op basis van de vereisten; snelheid van het platform, 
herstelbeheer, veiligheid, snelheid van applicatieontwikkeling, integratie mogelijkheden, platformflexibiliteit, schaalbaarheid, gebruiksvriendelijkheid, integratie met 
Airtable en MAKE.com, kostprijs, en updatebeleid. Om bepaalde criteria zorgvuldig te kunnen testen zal er een Proof of Concept plaatsnemen waarbij drie 
niet-programmeurs en één programmeur de drie platformen uittesten. Men neemt aan dat een platform het beste past bij het softwarebedrijf Quivvy Solutions wanneer het superieur is in alle criteria, en 
waarbij het met zowel Airtable als MAKE.com kan integreren.
%Wat is het beoogde resultaat van je bachelorproef? Wat zijn de criteria voor succes? Beschrijf die zo concreet mogelijk. Gaat het bv.\ om een proof-of-concept, een prototype, een verslag met aanbevelingen, een vergelijkende studie, enz.

\section{\IfLanguageName{dutch}{Opzet van deze bachelorproef}{Structure of this bachelor thesis}}%
\label{sec:opzet-bachelorproef}

% Het is gebruikelijk aan het einde van de inleiding een overzicht te
% geven van de opbouw van de rest van de tekst. Deze sectie bevat al een aanzet
% die je kan aanvullen/aanpassen in functie van je eigen tekst.

De rest van deze bachelorproef is als volgt opgebouwd:

In Hoofdstuk~\ref{ch:stand-van-zaken} wordt een overzicht gegeven van de stand van zaken binnen het onderzoeksdomein, op basis van een literatuurstudie.

In Hoofdstuk~\ref{ch:methodologie} wordt de methodologie toegelicht en worden de gebruikte onderzoekstechnieken besproken om een antwoord te kunnen formuleren op de onderzoeksvragen.

% TODO: Vul hier aan voor je eigen hoofstukken, één of twee zinnen per hoofdstuk

In Hoofdstuk~\ref{ch:conclusie}, tenslotte, wordt de conclusie gegeven en een antwoord geformuleerd op de onderzoeksvragen. Daarbij wordt ook een aanzet gegeven voor toekomstig onderzoek binnen dit domein.