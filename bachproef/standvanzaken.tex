\chapter{\IfLanguageName{dutch}{Stand van zaken}{State of the art}}%
\label{ch:stand-van-zaken}

\section{Inleiding}%
\label{sec:inleiding}
Projecten in softwarebedrijven gaan meestal over het budget. Daarbij doet het project meestal niet volledig wat de eindklant verwacht en vervolgens
doet het opgeleverde product niet wat het zou moeten doen ~\autocite{Moskal_2021}. Maar volgens ~\textcite{Moskal_2021} zijn deze problemen niet alleen op te merken 
in de softwarewereld maar ook in andere categorieën binnen de IT-sector, of bij het implementeren en ontwerpen van software systemen. In deze tijdsperiode worden er steeds meer
data verwerkt en ook opgeslaan ~\autocite{Moskal_2021}. Volgens ~\textcite{Moskal_2021} en ~\textcite{Parviainen_2022} veroorzaakt dit proces van het verwerken van data en opslaan
verandingen in de business, door de adoptie van digtale technologieën als gevolg tot digitalisering. Maar digitalisering betekent dat men bestaande producten of diensten moeten omzetten
naar een digitaal product of dienst, het kan ook zo zijn dat men software producten moeten kopen om de business processen te automatiseren ~\autocite{Moskal_2021}. Het verkrijgen van een competitief voordeel
kan bereikt worden door een specifiek ontworpen en ontwikkeld softwareoplossing dat voldoet aan de unieke behoeften van de eindklant ~\autocite{Moskal_2021}. Maar volgens ~\textcite{Moskal_2021} is een toegewijde IT-oplossing
zeer duur waardoor heel wat bedrijven dit niet kan veroorloven. Dit zet No-Code en Low-Code platformen in de spotlight  ~\autocite{Moskal_2021}.

\section{Reden tot gebruik}%
\label{sec:reden-tot-gebruik}
In dit hoofdstuk van de literatuurstudie gaan we ons verdiepen in de reden waarom Low-Code en No-Code platformen toch stijgen ze in gebruik. Dit zal helpen
om een beter beeld te krijgen waarom dit soort platformen, op dit moment, toch waard zijn om eens een kijkje in te nemen, voor zowel het bedrijf als de eindklant van het bedrijf.

In vergelijking met andere technologie trends zoals AI, Blockchain, Edge Computing en RPA groeit Low-Code en No-Code platformen zeer matig ~\autocite{Kulkarni_2021}.
Dit komt omdat het idee van Low-Code en No-Code development niet nieuw is, maar toch is er een stijging in het gebruik van deze platformen ~\autocite{Elshan2023}.
Volgens ~\textcite{Elshan2023} en ~\textcite{Kulkarni_2021} zijn er verschillende redenen waarom deze platformen toch stijgen in gebruik, vooral in kleine en middelgrote bedrijven.
\begin{itemize}
    \item \textbf{Beperkt aantal programmeurs}: 
    Low-Code platformen werden geïntroduceerd als een oplossing voor de dilemma tussen het te kort aan programmeurs en de hoge vraag naar softwareontwikkeling ~\autocite{ALSAADI_2021}. Volgens 
    ~\textcite{Moskal_2021} is de reden tot te kort aan programmeurs te wijten aan dat heel wat studenten zich laat afschrikken door de complexiteit van het programmeren. Dit geeft als gevolg dat weinig 
    studenten een diepe kennis hebben op het vlak van programmeren. Maar volgens ~\textcite{Moskal_2021} brengt een zeer gekwalificeerde programmeur het project niet altijd tot een goed einde.
    \item \textbf{Technologische turbulentie}:
    Doordat programmeertalen steeds veranderen en er steeds nieuwe technologieën worden geïntroduceerd is de kennis van de programmeurs niet altijd up-to-date ~\autocite{Moskal_2021}.
    \item \textbf{Hoge kosten}:
    De tradionele softwareontwikkeling eist een grote tol op de financiën van een bedrijf. Daarbij beseffen Software engineers dat het bouwen van een applicatie
    niet gemakkelijk is binnen het gegeven budget  ~\autocite{Moskal_2021}. Volgens ~\textcite{Elshan2023} zijn concepten zoals DevOps en BizOps, die de operations, developers en bussiness teams samenbrengen,
    overspoeld met moeilijkheden. Als gevolg van kosten dat over het budget gaat en requirement conflicten tussen de teams ~\autocite{Elshan2023}. Low-Code en No-Code platformen kunnen hier een
    oplossing bieden omdat het minder tijd en geld kost om een applicatie te bouwen ~\autocite{Elshan2023} ~\autocite{Bock_2021} ~\autocite{Rokis_2023}.
    \item \textbf{Tijdrovend}:
    Het ontwikkelen van software is een tijdrovend proces. De geschatte tijd dat nodig is om een applicatie te bouwen, op één operating systeem, is vaak
     zes maanden of langer ~\autocite{Moskal_2021}. Het probleem hiervan is dat snelheid binnen de business een belangrijke factor is ~\autocite{Sanchis_2019}.
     Want volgens ~\textcite{Sanchis_2019} betekent een veranderende markt dat bedrijven snel en flexibel moeten kunnen reageren op verandering om aan de requirements van de omgeving te kunnen voldoen.

    \item \textbf{Klantentevredenheid}:
    Low-Code en No-Code platformen zorgen voor een hogere klantentevredenheid ~\autocite{Elshan2023}.
    Volgens ~\textcite{Elshan2023} ontstaat er een soort van rolomkering plaats in het ontwikkelingsproject. De ontwikkelaars van het bedrijf worden niet langer meer gezien als de enige
    opdrachtgevers van een applicatie, maar ook de eindklanten van het bedrijf. Hierdoor kan de eindklant zelf de applicatie aanpassen naar zijn eigen wensen en behoeften ~\autocite{Elshan2023}.
    Volgens ~\textcite{Elshan2023} reduceert dit ook de misverstanden tussen de ontwikkelaars en de eindklant.
    \item \textbf{Digitale Transformatie}: 
    Niet alleen in de IT-sector maar ook in de bussines sector is er een digitale transformatie aan de gang. 
    Deze transformatie binnen de business omgeving zorgt voor een nood aan automatiseren in verschillende aspecten van de business ~\autocite{ALSAADI_2021}.
    Wat als gevolg heeft dat men papier documenten vervangen door digitale documenten om de fysieke processen te vervangen door digitale processen ~\autocite{ALSAADI_2021}.
    Volgens ~\textcite{ALSAADI_2021} merkt men op dat er een daling is van de nood aan menselijke tussenkomst in de processen omdat de automatisering gebruik maakt van vertrouwlijke software 
    dat minder fouten maakt en ook nog eens minder kosten met zich meebrengt ~\autocite{ALSAADI_2021}.
    \item \textbf{Complexe softwareontwikkeling}
  \end{itemize} 

\section{Voordelen en Nadelen}
\label{sec:voordelen-nadelen}
Op gevolg van het vorige hoofdstuk "Reden tot gebruik" zal er nu gekeken worden naar wat de voordelen en nadelen zijn van Low-Code en No-Code platformen. Dit zal
een overzicht geven van wat er allemaal mogelijk is en wat de limieten zijn van deze platformen.
%Hier komt de "voordelen en nadelen" van low-code en no-code platformen Beperkt aantal programmeurs, .... maar met meer tekst (gebruik voorstel)
\subsection{Voordelen}%
\label{subsec:voordelen}
\subsubsection{Snelheid}
\label{subsec:snelheid}
Zoals eerder besproken in het vorige hoofdstuk neemt de tradionele softwareontwikkeling veel tijd in beslag ~\autocite{Moskal_2021}. Dit is niet het geval 
bij LCNC platformen, wat volgens ~\textcite{Adrian_2020} een groot voordeel is. Vervolgens kan dit ook nog eens versterkt worden door ~\textcite{Yan2021}
die verteld dat heel wat bedrijven die gebruikmaken van Low-Code platformen vaststellen dat hun release van een applicatie bij 5 van de 10 keer sneller was dan voorheen.
In 2019 werd er dan ook een enquête gehouden van OutSystems, waaruit bleek dat gebruikers van Low-Code platformen 68\% van hun webapplicaties en 64\% van hun apps elk konden
bouwen in vier maanden ~\autocite{Yan2021}. Tegenover de traditionele ontwikkeling is dit een positief resultaat, want volgens ~\textcite{Yan2021}
was er maar 57\% van de webapplicaties en 49\% van de apps gebouwd binnen dezelfde tijdspanne.
\\
\\
Niet alleen het bouwen van applicaties gaat sneller, maar volgens ~\textcite{da_Cruz_2021} is het grootste voordeel
van Low-Code en No-Code platformen te danken aan het eenvoudig bouwen van complexe software waardoor bedrijven
sneller kunnen reageren op de veranderende markt. Daarnaast is het leren van LCNC platformen gemakkelijker en sneller dan het leren van 
een programmeertaal. Al deze eigenschappen zorgen ervoor dat developers sneller een prototype kunnen maken, feedback krijgen en een betere
klantenervaring kunnen bieden ~\autocite{da_Cruz_2021}.

\subsubsection{Veiligheid}
\label{subsec:veiligheid}
Het gebrek aan werknemers binnen de IT-sector, die heel wat massa aan software moeten ontwikkelen, zorgt ervoor dat de werknemers buiten de IT-sector
er vaak alleen voor staan waardoor ze gebruik moeten maken van third-party software ~\autocite{Yan2021}. Volgens ~\textcite{Yan2021} is dit een groot probleem,
want dit kan schade brengen op het bedrijf ~\autocite{Yan2021}. De oorzaak hiervan zou kunnen zijn dat ze niet op de hoogte zijn van de licentie en veiligheid, dit 
wordt ook wel "Shadow IT" genoemd ~\autocite{Yan2021} ~\autocite{Rokis_2022}. Daarom zorgt LCNC platformen, die geautoriseerd zijn door de IT-sector,
op vermindering van "Shadow IT" ~\autocite{Yan2021}. Daarnaast kan de werknemers buiten de IT-sector makkelijk een oplossing ontwikkelen met Low-Code en No-Code platformen  ~\autocite{Yan2021}.
Als gevolg dat de IT medewerkers niet telkens verstoord worden door andere werknemers ~\autocite{Yan2021}.  Dit biedt verschillende voordelen aan de IT-sector zoals het verminderen van de werkdruk en het verhogen van de veiligheid, want
LCNC platformen bevat de internationale standaarden voor veiligheid (ISO/IEC 27001, PCIDSS) ~\autocite{Sufi_2023}.
\\
\\
Volgens ~\textcite{Sufi_2023} wordt er ook hedendaags een principe genaamd "Security by Design" toegepast. Dit principe neemt heel wat zorgen af op het vlak van veligheid
in de IT voor de Citizen Developers, ook wel de werknemers buiten de IT-sector genoemd ~\autocite{Sufi_2023}. Vervolgens verteld ~\textcite{Elshan2023} dat deze digitalisering en het automatiseren van werkprocessen
een positieve impact hebben op de kwaliteit van het bedrijf, wat als gevolg de veiligheid van de bedrijfsprocessen verhoogt. De reden dat het ook de veiligheid verhoogt is omdat de standaardisatie van de werkprocessen zowel
bronnen als menselijke fouten vermindert ~\autocite{Elshan2023}.

\subsubsection{Universele toegangkelijkheid}
\label{subsec:universele-toegangkelijkheid}

Eerder in de literatuurstudie werd er al gesproken over dat er een te kort is aan IT-personeel die over de kwaliteiten beschikt. 
Dit kan door ~\textcite{Sufi_2023} nog eens bevestigd worden waarbij 
verteld wordt dat bedrijven vaak falen bij het rekruteren van IT-personeel, door het gebrek aan IT'ers met de nodige kennis en ervaring.
Volgens ~\textcite{Sufi_2023} is hier een oplossing voor, namelijk Low-Code en No-Code platformen. Deze platformen laten niet-programmeurs, met een zeer basis kennis,
toe om te werken aan IT-oplossingen zoals: dashboards, applicaties, en databanken zonder problemen ~\autocite{Sufi_2023}. Dit lost het probleem op van het te kort aan IT-personeel binnen het bedrijf ~\autocite{Sufi_2023}.
Maar volgens ~\textcite{Sufi_2023} is dit niet de enige voordeel van universele toegankelijkheid want door LCNC platformen kunnen de niet-programmeurs ook
snel het benodigde IT-oplossing ontwikkelen, zonder het verspillen van tijd en geld ~\autocite{Sufi_2023}.
\\
\\
De reden waarom we kunnen spreken van universele toegankelijkheid is omdat Low-Code en No-Code platformen makkelijk te leren zijn ~\autocite{Sufi_2023} ~\autocite{ALSAADI_2021}.
Zoals vermeld door ~\textcite{ALSAADI_2021} zullen niet alleen professionele ontwikkelaars LCNC platformen benutten, maar ook de niet-programmeurs.
Beginners ofwel de niet-programmeurs hebben nu ook de mogelijjkheid om applicaties te bouwen zonder enige kennis van programmeertalen ~\autocite{ALSAADI_2021}.
Dit is allemaal mogelijk doordat LCNC-platformen een drag-en-drop systeem hebben, hierbij kunnen de gebruikeres de compenenten van de applicatie slepen en neerzetten waarbij vervolgens in de achtergrond de code wordt 
gegenereerd in de achtergrond ~\autocite{ALSAADI_2021}.

\subsubsection{Cloud-Based LCNC: Dataherstel}
\label{subsec:cloud-based-lcnc}
Hedendaags stappen heel wat bedrijven over naar cloud-based technologieën omdat het heel 
wat voordelen biedt zoals: kostenbesparing, schaalbaarheid, flexibiliteit, herstel van data, en nog heel wat andere voordelen ~\autocite{Sufi_2023}.
Volgens ~\textcites{Sufi_2023} zijn grotendeels alle LCNC platformen cloud-based wat als gevolg heeft dat
de LCNC snelle strategieën kan toepassen voor het migreren naar de cloud.
\\
\\
Doordat het cloud-based is, is het ook mogelijk om data te herstellen in het geval van een ramp ~\autocite{Sufi_2023}.
Deze platformen verzekeren dat het systeem regelmatig is geback-upt en dat de data hersteld kan worden in het geval van een ramp ~\autocite{Sufi_2023}.
Dit is een zeer belangrijk voordeel voor bedrijven want volgens ~\textcite{Sufi_2023} 
kan 20\% van cloud gebruikers hun data herstellen in vier uur of minder herstellen, terwijl 9\%
van de niet cloud gebruikers hun data herstellen in vier uur of minder ~\autocite{Sufi_2023}. Daarnaast riskeren developers, die niet gebruikmaken van de cloud,
dat ze hun data verliezen op hun computer in het geval van een ramp ~\autocite{Sufi_2023}. Dit is niet zo bij cloud-hosted services want volgens ~\textcite{Sufi_2023}
verzekeren deze services dat de data altijd beschikbaar is.

\subsection{Nadelen}%
\label{subsec:nadelen}
%Rood gemarkeerd in paper = nadelen
\subsubsection{Beperkte flexibiliteit}
\label{subsec:beperkte-flexibiliteit}
Niet-programmeurs kunnen snel expert worden in hun gekozen LCNC platform, maar dit betekent ook dat ze vastzitten aan de beperkingen en de framework van het platform ~\autocite{Sufi_2023} ~\autocite{Talesra_2021}.
Deze limitaties kunnen voor problemen zorgen wanneer de niet-programmeurs een applicatie moeten bouwen dat zeer aanpasbaar moet zijn ~\autocite{Talesra_2021}. Vervolgens hebben de meeste LCNC platformen ook een gelimiteerde
opties voor integratie met andere systemen, waardoor het zowel een uitdaging is voor de niet-programmeurs als het bedrijf ~\autocite{Talesra_2021}. Daarnaast bevat LCNC platformen ook third-party afhankelijkheden, wat als gevolg heeft dat
men afhankelijk is van de verkoper van de third-party bij het oplossen van veiligheid en prestatie problemen ~\autocite{Talesra_2021}.
Dit is niet het geval bij tradionele programmeertalen en platformen zoals Java, C\#, en Python ~\autocite{Sufi_2023}, waarbij de developers de vrijheid hebben om de applicatie te bouwen zoals ze zelf willen.
Jammer genoeg is er geen enkele bestaande LCNC-platformen die in 2023 deze flexibiliteit biedt ~\autocite{Sufi_2023}. Als gevolg dat niet-programmeurs
beperkt zijn op het vlak van beschikbare opties bij het ontwikkelen van hun oplossing ~\autocite{Sufi_2023}.
\\
\\
De beperkingen van LCNC platformen hebben te maken met dat de platformen bestaat uit gevisualiseerde componenten die de gebruiker kan slepen en neerzetten ~\autocite{Yan2021}.
Deze componenten zijn vooraf gedefinieerd en staan ook vast, wat als gevolg heeft dat het niet zo aanpasbaar is als een applicatie dat gebouwd is met een programmeertaal ~\autocite{Yan2021}.
Volgens ~\textcite{Yan2021} is het hierdoor moeilijk en tijd spenderend om ingewikkelde of aanpasbare features of functionaliteiten, dat niet door de platformen wordt aangeboden, te ontwikkelen.
\subsubsection*{Gelimiteerde schaalbaarheid}
\label{subsec:gelimiteerde-schaalbaarheid}
Volgens ~\textcite{Elshan2023} en ~\textcite{Sufi_2023} is het bouwen van applicaties met LCNC platformen op dit moment zeer gelimiteerd in schaalbaarheid. Daarom worden deze platformen
in hedendaags enkel gebruikt voor het ontwikkelen van kleine schaalbare applicaties ~\autocite{Sufi_2023}. Vervolgens kunnen de meeste platformen door limitatie niet gebruikt worden voor applicaties waarbij het in 
de toekomst zou moeten uitgebreid worden ~\autocite{Elshan2023}. Volgens ~\textcite{Yan2021} lag, in 2015, de gemiddelde runtime schaal van applicaties, die gemaakt zijn door LCNC-platformen, tussen de 200 en 2000 gelijktijdige gebruikers.

\subsection*{Veiligheidszorgen}
\label{subsec:veiligheidszorgen}
In de literatuur werd eerder verteld dat veiligheid een voordeel is van LCNC platformen, maar volgens ~\textcite{Yan2021} is dit niet altijd het geval. Ze kunnen namelijk moeilijk of zelfs niet aangepast worden, waardoor bedrijven, dat er gebruik van maken,
volledig de diensten van de verkoper moeten vertouwen op het niet genereren van veiligheidsproblemen ~\autocite{Yan2021}. Doordat het bedrijf volledig afhankelijk is van de verkoper, kan de data van het bedrijf in gevaar komen door een data lek bij de verkoper omdat namelijk de data security
en de source code niet beheerst wordt door het bedrijf ~\autocite{Yan2021}.

\subsection*{Vendor Lock-in}
\label{subsec:technische-schulden}
Volgens ~\textcite{Yan2021} betekent vendor lock-in dat je als bedrijf afhankelijk bent van de verkoper voor hun diensten en producten, wat als gevolg heeft dat het moeilijk is om als klant te veranderen naar een andere verkoper.
Het gevaar is dat dit ook zo kan zijn bij LCNC platformen, waarbij het bedrijf in de toekomst meer zal investeren in het platform van de verkoper ~\autocite{Yan2021}. Volgens ~\textcite{Yan2021} zal dit dan kunnen leiden tot stijgende kosten van de diensten en producten van de verkoper, en zal het bedrijf
moeilijker kunnen veranderen. Als het bedrijf toch van plan zou zijn om te veranderen, zal de applicatie helemaal opnieuw moeten gebouwd worden ~\autocite{Sufi_2023}. Volgens ~\textcite{Sufi_2023} is dit omdat de bestaande LCNC platformen beschikken niet over integreren van gebouwden applicaties tussen verschillende LCNC platformen, maar
ook omdat elke verkoper hun eigen ecosysteem hebben voor applicatie ontwikkeling ~\autocite{Sufi_2023}.


\section{LCNC uitdagingen}
\label{sec:lcnc-uitdagingen}
Hieronder lichten we toe wat de obstakels en problemen zijn die overwonnen moeten worden. Deze obstakels en problemen bieden een kans voor groei en
verbetering van de LCNC platformen.
%Oranje gemarkeerd in paper = nadelen
\subsubsection*{Weerstand binnen het bedrijf}
\label{subsec:weerstand-binnen-het-bedrijf}
%(1) in Elshan paper
Wanneer een software bedrijf zou beslissen om over te stappen naar Low-Code platformen, kan er weerstand ontstaan binnen het bedrijf ~\autocite{Elshan2023}.
Volgens ~\textcite{Elshan2023} heeft dit vooral te maken met dat er een misverstand is in verband met Low-Code platformen. Werknemers binnen het bedrijf denken dat Low-Code platformen
hun werk zal vervangen door de stijging van automatisering en digitalisering ~\autocite{Elshan2023}. Volgens ~\textcite{Elshan2023} zullen de IT-medewerkers binnen het bedrijf
het moeilijkste zijn om te overtuigen, omdat ontwikkelaars zich vaak zien als een artiest, die hun eigen foto's schildert en niet automatisch willen laten genereren door een machine ~\autocite{Elshan2023}.
Vervolgens geeft dit de developer meer onderhoud taken dan ontwikkelingstaken ~\autocite{Elshan2023}. Wat als gevolg kan geven dat ze zich niet uitgedaagd voelen op het vlak van werk en uiteindeijk
het bedrijf zullen verlaten ~\autocite{Elshan2023}. Daarnaast kan het gebruik van Low-Code platformen ook een impact hebben op hun salaris en stress dat leidt tot ontevredenheid ~\autocite{Elshan2023}.


\subsubsection*{Shadow IT}
\label{subsec:shadow-it}
%(2) in Elshan paper
De betekenis van Shadow IT werd eerder al besproken in de literatuurstudie. Kortom de mensen buiten de IT zijn niet op de hoogte van de veiligheid en licenties ~\autocite{Yan2021} ~\autocite{Rokis_2022}.
Volgens ~\textcite{Elshan2023} kunnen andere afdelingen binnen het bedrijf hun eigen applicaties bouwen met Low-Code platformen, zonder dat de IT-afdeling hiervan op de hoogte is, wat voor nieuwe problemen kan zorgen ~\autocite{Elshan2023}.
Wat als gevolg heeft dat ze voor elk klein ding iets kunnen ontwikkelen, waardoor er geen overzicht meer is van de applicaties binnen het bedrijf ~\autocite{Elshan2023}. Daarnaast kunnen enkele voordelen mogelijk verloren gaan
door dubbele applicatie development ~\autocite{Elshan2023}. Volgens ~\textcite{Elshan2023} is dit enkel de top van de ijsberg want er zijn nog heel wat andere serieuze gevolgen dat Shadow IT met zich meebrengt.
In eerste instantie zou elke applicatie door verschillende processen moeten gaan om de veiligheid en data bescherming te verzekeren ~\autocite{Elshan2023}. Maar dit is niet het geval bij Shadow IT,
het kan zo zijn dat de werknemers onwetend persoonlijke data gebruiken in een applicatie die niet in overeenstemming komt met de privacywetgeving  ~\autocite{Elshan2023}. Volgens ~\textcite{Elshan2023}
zou dit kunnen opgelost worden door vooropgestelde termen tot het gebruik van Low-Code platformen, vooropgestelde trainingen, en duidelijk te maken wat de veraantwoordelijkheden zijn van de werknemers.

\subsubsection*{Gebrek aan eigen controle}
\label{subsec:gebrek-aan-eigen-controle}
%(3) in Sufi paper Lack of On-Premise Support
Hedendaags beschikken alle No-Code en Low-Code platformen over een cloud forward of cloud-first aanpak ~\autocite{Sufi_2023}.
Hierbij implementeren citizen developers, ook wel niet-programmeurs genoemd, hun algoritmes in een cloud gebasseerde interface ~\autocite{Sufi_2023}.
Maar volgens ~\textcite{Sufi_2023} kunnen niet alle applicaties onderhouden worden in de cloud, zoals bijvoorbeeld applicaties die gevoelige data bevatten (inlichtingdienst applicatie, geheime overheid applicaties, enz.) ~\autocite{Sufi_2023}.
Deze applicaties zijn niet geschikt voor LCNC-platformen. Daarnaast zijn er heel wat private bedrijven dat gegevens hosten die niet in de cloud gehost kunnen worden vanwege
privacywetgeving ~\autocite{Sufi_2023}. Deze soort techonologischeoplossingen zijn enkel mogelijk met een ontwerp die offline of op locatie werkt ~\autocite{Sufi_2023}.
\subsubsection*{LCNC in Software Development Life Cycle}
\label{subsec:lcnc-binnen-agile}
%(3) in Rokis & Kirikova en voorstel
\textbf{Requirement Analyse}
\\
De specificaties van de platformen verschillen binnen software, wat als gevolg gezien kan worden als een
uitdaging ~\autocite{Rokis_2022}. Hierdoor wordt een tool voor eisenbeheer binnen de LCNC platformen
gezien als een waardevolle toevoeging ~\autocite{Rokis_2022}. In deze fasen binnen softwareontwikkeling wordt een verandingen in de eisen
ook gezien als een uitdaging ~\autocite{Rokis_2022}.  Maar  dit zou kunnen opgelost worden met een prototype van de applicatie,
waardoor men telkens de volgende dag een nieuwe versie van de applicatie kan tonen aan de eindklant ~\autocite{Rokis_2022}.
\\
\textbf{Planning}
\\
Volgens ~\textcite{Rokis_2022} bestaat het planningsfase uit het analyseren en plannen van de geschiktheid, complexiteit, risico's, afhankelijkheden en de tijdsduur van de ontwikkeling van de applicatie, in een bepaalde LCNC-platform.
De uitdagingen binnen deze fase is geraleteerd aan de geschiktheid van de LCNC-platformen, omdat er een groot aantal aan platformen beschikbaar is op de markt ~\autocite{Rokis_2022}.
Hierbij wordt er gekeken naar de geraleteerde kosten, de snelheid van het leren, de ondersteunde features, en de functionaliteiten in development van de LCNC-platformen ~\autocite{Rokis_2022}.
Daarnaast zijn ook heel wat bedrijven bezorgd over vendor lock-in ~\autocite{Rokis_2022}.
Maar door de overvloed aan platformen is het moeilijk om de meest compatibele platform te kiezen voor de applicatie ~\autocite{Rokis_2022}.
\\
\textbf{Application Design}
\\
Het ontwerp van een applicatie hangt af van de eisen van de eindklant ~\autocite{Rokis_2022}. Het bestaat uit de architectuur, de modulariteit, schaalbaarheid en andere aspecten van de applicatie ~\autocite{Rokis_2022}.
In deze fasen zijn er verschillende uitdagingen zoals de limieten van uitbreidbaarheid en integratie met andere soorten platformen, data opslag, en de gebruikersinterface ~\autocite{Rokis_2022}.
De limieten van uitbreidbaarheid heeft te maken met dat de platform vooropgestelde functionaliteiten heeft waarbij de gebruiker deze grotendeels niet kan aanpassen ~\autocite{Rokis_2022}. Bij de integratie met andere soorten platformen wordt er een vermogen
van informatie uitgewisseld tussen onderdelen van de applicatie of met externe diensten ~\autocite{Rokis_2022}. Omdat er een gebrek is aan algemene afspraken en vaak de software 'closed source' is maakt het dit een lastig punt ~\autocite{Rokis_2022}. Dit geeft als gevolg dat het moeilijk is
om systemen met elkaar goed te laten samenwerken ~\autocite{Rokis_2022}. Wat als gevolg heeft dat er beperkingen zijn in het ontwerpen van systemen en het delen van ontwikkelde diensten en producten ~\autocite{Rokis_2022}.
Vervolgens hebben we ook nog de uitdaging van de gebruikersinterface en de data opslag ~\autocite{Rokis_2022}.
Volgens ~\textcite{Rokis_2022} ondervinden de gebruikers van Low-Code platformen dat ze tijdens het ontwerpen van applicaties uitdagingen tegenkomen met betrekking tot het ontwerp van de gebruikersinterface en de data opslag.
De reden hiervoor is omdat het ontwerpen van hoe een applicatie eruit ziet en werkt, specialistische kennis vereist ~\autocite{Rokis_2022}. Wat als gevolg heeft dat de gebruikers die geen achtergrond kennis hebben van applicaties ontwikkelen hebben een zeer lastige taak vinden ~\autocite{Rokis_2022}.
Vervolgens is dit hetzelfde verhaal bij het opzetten van een systeem voor de data op te slaan, want men moet namelijk een opslag locatie kiezen en dan ook nog gegevens kunnen verplaatsen naar de opslag locatie ~\autocite{Rokis_2022}.
Volgens ~\textcite{Rokis_2022} zou de documentatie verbeteren en het meer toegankelijk maken voor iedereen een eventuele oplossing kunnen zijn voor het opslaan van data.
\\
\textbf{Development}
\\
In deze fase wordt de applicatie gebouwd ~\autocite{Rokis_2022}.  Volgens ~\textcite{Rokis_2022} ervaren gebruikers hier heel wat uitdagingen zoals het aanpassen van gebruikersinterface, het implementeren van de bussines logica, en het integreren van de applicatie met andere systemen
of zelfs complexere ontwikkelingen waarbij toegang tot de code nodig is. De reden voor de eerste drie opgesommende uitdagingen, is omdat men vaak te maken heeft met niet complete of niet correcte documentatie ~\autocite{Rokis_2022}. Maar niet alleen de documentatie is een probleem,
ook is er een gebreken aan leermateriaal en trainingen zoals toturials van de LCNC platformen ~\autocite{Rokis_2022}.
\\
\textbf{Testing}
\\
Hier wordt de applicatie getest om te verifieëren of de applicatie voldoet aan de eisen van de eindklant ~\autocite{Rokis_2022}.
Volgens ~\textcite{Rokis_2022} ervaren hier de gebruiken ook meerdere uitdagingen zoals een gebrek aan documentatie vooral voor geautomatiseerde testen uit te voeren, test coverage
, en het gebruik kunnen maken van third-party test tools  ~\autocite{Rokis_2022}. Volgens ~\textcite{Rokis_2022} hebben Low-Code platformen een beperkte Low-Code testframeworks en analysemogelijkheden.
Daarnaast wordt het testen van niet-functionele vereisten in Low-Code platformen vaak verwaarloosd ~\autocite{Rokis_2022}.
Volgens ~\textcite{Rokis_2022} zou dit kunnen opgelost worden door het onderzoeken naar een implementatie van een Low-Code testframework dat alle testactiviteiten zou dekken ~\autocite{Rokis_2022}.
\\
\textbf{Deployment}
\\
Bij het publiceren van de applicatie ervaren de gebruikers enkele uitdagingen zoals problemen bij het configureren, toegankelijkheid, en performantie problemen ~\autocite{Rokis_2022}.
Dit heeft weer als oorzaak een gebrek aan goede documentatie van de LCNC platformen ~\autocite{Rokis_2022}.
\\
\textbf{Maintenance}
\\
Uit eerder besproken literatuur kan men opmaken dat de onderhoud van applicaties zeer weinig vraagt, dit kan ook nog eens bevestigd worden door ~\textcite{Rokis_2022}.
Volgens ~\textcite{Rokis_2022} komt men hier weer terecht bij de uitdaging van debuggen in een grafische representatie.


\section{LCNC binnen bedrijven}
\label{sec:lcnc-bedrijven}
Dit hoofdstuk zal meer inzicht geven in waarom deze platformen worden gebruikt binnen bedrijven. Dit zal een betere duidelijkheid geven
waarom deze bachelorproef relevant is voor software bedrijven, meer specifiek Quivvy Solutions BV.
\\
\\
Diverse organisaties hangen af van software om te kunnen functioneren ~\autocite{Hintsch2021}. Dit kan volgens ~\textcite{Rafiq_2022} ook opgemerkt worden bij zowel
start-ups als middelgrote en grote bedrijven, waarbij start-ups jonge bedrijven zijn met een gelimiteerd aantal middelen tot hun beschikking ~\autocite{Rafiq_2022}. Deze soort bedrijven
kom dagelijks in contact met verschillende uitdagingen zoals tijdsdruk, team formatie, en en snel groeiende markten ~\autocite{Rafiq_2022}. Om dit allemaal te weerstaan maken heel wat
start-ups bedrijven gebruik van No-Code en Low-Code applicatie development platformen ~\autocite{Rafiq_2022}. De reden waarom start-ups LCNC paltformen gebruiken en wat het verschil is met 
middelgrote en grote bedrijven kan uitgelegd worden door het onderzoek van ~\textcite{Rafiq_2022}. Het onderzoek van ~\textcite{Rafiq_2022} werd uitgevoerd bij twee bedrijven, een software start-up en een groot bedrijf
dat op meerdere plaatsen bevestigd is. Uit het onderzoek van ~\textcite{Rafiq_2022} maakte de start-up bedrijf gebruik van LCNC platformen maar het grote bedrijf enkel van Low-Code platformen. Het combineren van Low-Code en No-code platformen
brengt voordelen met zich mee zoals ten eerste het maken van een prototype, het gebruiken voor ontwerp, en als laatste voor het uitvoeren van een dienst ~\autocite{Rafiq_2022}. 
Verder in het analyse werd opgemerkt dat start-ups bedrijven deze platformen niet gebruiken voor hun hoofdproduct ~\autocite{Rafiq_2022}. Ten slotte wordt
Low-Code platformen in grote bedrijven gebruikt voor snelle ontwikkeling van de applicatie, snelle feedback, en de mindere werklast. ~\autocite{Rafiq_2022}.




\section{Gebruik van LCNC door eindklanten}
\label{sec:lcnc-eindklanten}
Dit is een opvolging van het vorige hoofdstuk 'LCNC binnen bedrijven'. Omdat het belangrijk is dat de eindklant ook met deze platformen kan werken.
Meer specifiek naar de ervaringen van personen met voldoende  tot geen ervaring in het programmeren die deze platformen gebruiken.
\\
\\
Volgens ~\textcite{Yan2021} kunnen ook de eindklanten, die geen ervaring hebben in het programmeren, gebruik maken van Low-Code en No-Code platformen.
Vervolgens werd een studie over hoe eindegebruikers tegenover ervaren ontwikkelaars performeren bij het gebruik van Low-Code Development  in 2020, volgens ~\textcite{Hintsch2021} 
Uit de studie bleek dat ervaren ontwikkelaars moeilijkheden hadden bij het identificeren van belangrijke concepten binnen software engineering in het platform ~\autocite{Hintsch2021}.
Hoewel ontwikkelaar uitdagingen ervaarden was dit ook het geval bij eindegebruikers, waarbij de eindgebruikers moeilijkeheden hadden bij eenvoudige taken zoals het maken van een scherm  ~\autocite{Hintsch2021}.
Daarnaast was de connectie met de database leggen en de 'parameter passing' moeilijke en verwarrende delen bij het ontwikkelen van de applicatie  ~\autocite{Hintsch2021}.

% Tip: Begin elk hoofdstuk met een paragraaf inleiding die beschrijft hoe
% dit hoofdstuk past binnen het geheel van de bachelorproef. Geef in het
% bijzonder aan wat de link is met het vorige en volgende hoofdstuk.

% Pas na deze inleidende paragraaf komt de eerste sectiehoofding.

%Dit hoofdstuk bevat je literatuurstudie. De inhoud gaat verder op de inleiding, maar zal het onderwerp van de bachelorproef *diepgaand* uitspitten. De bedoeling is dat de lezer na lezing van dit hoofdstuk helemaal op de hoogte is van de huidige stand van zaken (state-of-the-art) in het onderzoeksdomein. Iemand die niet vertrouwd is met het onderwerp, weet nu voldoende om de rest van het verhaal te kunnen volgen, zonder dat die er nog andere informatie moet over opzoeken \autocite{Pollefliet2011}.

%Je verwijst bij elke bewering die je doet, vakterm die je introduceert, enz.\ naar je bronnen. In \LaTeX{} kan dat met het commando \texttt{$\backslash${textcite\{\}}} of \texttt{$\backslash${autocite\{\}}}. Als argument van het commando geef je de ``sleutel'' van een ``record'' in een bibliografische databank in het Bib\LaTeX{}-formaat (een tekstbestand). Als je expliciet naar de auteur verwijst in de zin (narratieve referentie), gebruik je \texttt{$\backslash${}textcite\{\}}. Soms is de auteursnaam niet expliciet een onderdeel van de zin, dan gebruik je \texttt{$\backslash${}autocite\{\}} (referentie tussen haakjes). Dit gebruik je bv.~bij een citaat, of om in het bijschrift van een overgenomen afbeelding, broncode, tabel, enz. te verwijzen naar de bron. In de volgende paragraaf een voorbeeld van elk.

%\textcite{Knuth1998} schreef een van de standaardwerken over sorteer- en zoekalgoritmen. Experten zijn het erover eens dat cloud computing een interessante opportuniteit vormen, zowel voor gebruikers als voor dienstverleners op vlak van informatietechnologie~\autocite{Creeger2009}.

%Let er ook op: het \texttt{cite}-commando voor de punt, dus binnen de zin. Je verwijst meteen naar een bron in de eerste zin die erop gebaseerd is, dus niet pas op het einde van een paragraaf.

%\lipsum[7-20]
