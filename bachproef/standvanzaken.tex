\chapter{\IfLanguageName{dutch}{Stand van zaken}{State of the art}}%
\label{ch:stand-van-zaken}

\section{Inleiding}%
\label{sec:inleiding}
Softwarebedrijven merken op dat projecten telkens over het budget gaan. Daarbij doet het project meestal niet wat de eindklant verwacht en vervolgens doet het opgeleverde product niet wat het zou moeten doen ~\autocite{Moskal_2021}.
Maar volgens ~\textcite{Moskal_2021} zijn deze problemen niet enkel op te merken in de softwarewereld maar ook in andere categorieën binnen de IT-sector. 
Daarnaast moeten softwarebedrijven ook rekeninghouden met dat er in deze tijdsperiode meer opslag en verwerking van data is.
Volgens ~\textcite{Moskal_2021} en ~\textcite{Parviainen_2022} brengt het verwerken van data en opslaan veranderingen in de business, 
doordat bedrijven zich meer digitaliseren. Maar digitalisering betekent dat men producten of diensten moeten omzetten naar een digitaal product 
of dienst. Het kan ook zo zijn dat men software producten moeten kopen om de business processen te automatiseren ~\autocite{Moskal_2021}. 
Het verkrijgen van een competitief voordeel kan men bereiken door een specifiek ontworpen en ontwikkelde softwareoplossing die voldoet aan de behoeften van de eindklant.
Maar volgens ~\textcite{Moskal_2021} is een toegewijde IT-oplossing duur waardoor bedrijven zich dit niet kunnen veroorloven. Dit zet No-Code en Low-Code platformen in de spotlight
door de functies zoals het integreren van data via tools en een drag-and-drop systeem om de ontwikkeling te versnellen \autocite{Kulkarni_2021}. Met deze platformen kan je software ontwikkelen zonder dat je
zelf code moet schrijf of met een minimale hoeveelheid code, aan de hand van het drag-and-drop systeem.
\section{Reden tot gebruik}%
\label{sec:reden-tot-gebruik}
In dit hoofdstuk van de literatuurstudie wordt er verteld waarom Low-Code en No-Code platformen stijgen in gebruik.
\\
\\
\\
In vergelijking met andere technologie trends zoals AI, Blockchain, Edge Computing en RPA groeit Low-Code en No-Code matig ~\autocite{Kulkarni_2021}.
Dit komt omdat het idee van Low-Code en No-Code development niet nieuw is, maar toch is er een stijging in het gebruik van deze platformen op te merken ~\autocite{Elshan2023}.
Volgens ~\textcite{Elshan2023} en ~\textcite{Kulkarni_2021} zijn er verschillende redenen waarom deze platformen toch stijgen in gebruik, vooral in kleine en middelgrote bedrijven:
\begin{itemize}
    \item \textbf{Beperkt aantal programmeurs}: 
    Low-Code platformen zijn geïntroduceerd als een oplossing voor het dilemma tussen het te kort aan programmeurs en de hoge vraag naar softwareontwikkeling ~\autocite{ALSAADI_2021}. Volgens 
    ~\textcite{Moskal_2021} is het te kort aan programmeurs te wijten aan het feit dat studenten zich laten afschrikken door de complexiteit van programmeren. Dit geeft tot gevolg dat weinig 
    studenten een diepe kennis hebben op het vlak van programmeren.
    \item \textbf{Technologische turbulentie}:
    Volgens ~\textcite{Moskal_2021} brengt een gekwalificeerde programmeur het project niet altijd tot een goed einde.
    Dit komt omdat programmeertalen veranderen en er altijd nieuwe technologieën geïntroduceerd worden is de kennis van de programmeurs niet altijd up-to-date.
    \item \textbf{Kosten}:
    De traditionele softwareontwikkeling legt een zware financiële last op het bedrijf. 
    Daarbij beseffen software engineers dat bouwen van een applicatie niet gemakkelijk is binnen het gegeven budget ~\autocite{Moskal_2021}. 
    Volgens ~\textcite{Elshan2023} zijn concepten zoals DevOps en BizOps, die de operations, developers en business teams samenbrengen, overspoelt met moeilijkheden. 
    Door de kosten dat over het budget gaan en de conflicten over de requirements tussen de teams.
    Low-Code en No-Code platformen kunnen hier een oplossing bieden omdat het minder tijd en geld kost om een applicatie te bouwen ~\autocite{Elshan2023} ~\autocite{Bock_2021} ~\autocite{Rokis_2023}.
    \item \textbf{Tijdsduur}:
    Het ontwikkelen van software is een tijdrovend proces. De geschatte tijd die nodig is om een applicatie te bouwen, op één besturingssysteem, is zes maanden of langer ~\autocite{Moskal_2021}. 
    Het probleem hierbij is dat snelheid binnen de business een belangrijke factor is ~\autocite{Sanchis_2019}. 
    Volgens ~\textcite{Sanchis_2019} betekent een veranderende markt dat bedrijven snel en flexibel moeten kunnen reageren op veranderingen om aan de requirements van de omgeving te kunnen voldoen.

    \item \textbf{Klantentevredenheid}:
    Low-Code en No-Code platformen zorgen voor een hogere klantentevredenheid ~\autocite{Elshan2023}. 
    De ontwikkelaars van het bedrijf worden niet langer beschouwd als de enige belanghebbenden van een applicatie; de eindklanten van het bedrijf worden nu ook als belangrijke stakeholders erkend.
    Hierdoor kan de eindgebruiker zelf de applicatie aanpassen naar zijn eigen wensen en behoeften. 
    Volgens ~\textcite{Elshan2023} reduceert dit ook de misverstanden tussen de ontwikkelaars en de eindklant.

    \item \textbf{Digitale Transformatie}: 
    Niet enkel in de IT-sector maar ook in de business sector is er een digitale transformatie aan de gang. 
    Deze omvorming binnen de zakenwereld zorgt voor een nood aan automatiseren bij verschillende aspecten van de business ~\autocite{ALSAADI_2021}. 
    Dit heeft als gevolg dat papieren vervangen worden door digitale documenten om de fysieke processen om te zetten dan naar digitale processen. 
    Volgens ~\textcite{ALSAADI_2021} neemt de behoefte aan menselijke tussenkomst af bij de processen omdat de automatisering gebruik maakt van vertrouwelijke software die minder fouten maakt en ook minder kosten met zich meebrengt ~\autocite{ALSAADI_2021}.
  \end{itemize} 

\section{Voordelen en Nadelen}
\label{sec:voordelen-nadelen}
In deze paragraaf zullen de voordelen en nadelen van Low-Code en No-Code platformen besproken worden. Dit zal
een overzicht geven van wat er allemaal mogelijk is en wat de limieten zijn van deze platformen.
%Hier komt de "voordelen en nadelen" van low-code en no-code platformen Beperkt aantal programmeurs, .... maar met meer tekst (gebruik voorstel)
\subsection{Voordelen}%
\label{subsec:voordelen}
\subsubsection{Snelheid}
\label{subsec:snelheid}
Zoals besproken in het vorig hoofdstuk neemt de traditionele softwareontwikkeling veel tijd in beslag ~\autocite{Moskal_2021}. 
Dit is niet het geval bij LCNC platformen, wat volgens ~\textcite{Adrian_2020} een groot voordeel is. 
Vervolgens kan dit ook nog versterkt worden door ~\textcite{Yan2021} die vertelt dat bedrijven die gebruikmaken van 
Low-Code platformen vaststellen dat hun release van een applicatie bij 5 van de 10 keer sneller was dan vroeger. In 2019 is er dan ook een enquête gehouden door OutSystems, 
waaruit bleek dat gebruikers van Low-Code platformen 68\% van hun webapplicaties en 64\% van hun apps konden bouwen in 4 maanden ~\autocite{Yan2021}. 
Tegenover de traditionele ontwikkeling is dit een positief resultaat, want volgens ~\textcite{Yan2021} zou er maar 57\% van de webapplicaties en 49\% van de apps gebouwd zijn binnen dezelfde tijdspanne.
\\
\\
Niet enkel applicaties bouwen gaat sneller, maar volgens ~\textcite{da_Cruz_2021} is het grootste voordeel van Low-Code en No-Code platformen 
te danken aan het eenvoudig bouwen van ingewikkelde software waardoor bedrijven direct kunnen reageren op de veranderende markt. 
Daarnaast is het aanleren van LCNC platformen gemakkelijker en sneller dan een programmeertaal leren. 
Al deze eigenschappen zorgen ervoor dat developers vlot een prototype kunnen maken, feedback krijgen en een betere klantenervaring kunnen bieden ~\autocite{da_Cruz_2021}.
\subsubsection{Veiligheid}
\label{subsec:veiligheid}
Het gebrek aan werknemers binnen de IT-sector, die een massa aan software moeten ontwikkelen, 
zorgt ervoor dat de werknemers buiten de IT-sector er alleen voorstaan waardoor ze gebruik moeten maken van third party software ~\autocite{Yan2021}. 
Volgens ~\textcite{Yan2021} is dit een groot probleem, want dit kan schade brengen aan het bedrijf. 
De oorzaak hiervan zou kunnen zijn dat ze niet op de hoogte zijn van de licentie en veiligheid, dit wordt ook wel "Shadow IT" genoemd 
~\autocite{Rokis_2022}. Daarom zorgen LCNC platformen, die geautoriseerd zijn door de IT-sector, voor vermindering van "Shadow IT" ~\autocite{Yan2021}. 
Daarnaast kunnen de werknemers buiten de IT-sector makkelijk een oplossing ontwikkelen met Low-Code en No-Code platformen.
Dit biedt voordelen voor de IT-sector zoals de werkdruk verminderen en securiteit verhogen, 
want LCNC platformen bevatten de internationale standaarden voor veiligheid (ISO/IEC 27001, PCIDSS) ~\autocite{Sufi_2023}.
\\
\\
Volgens ~\textcite{Sufi_2023} wordt er ook hedendaags een principe genaamd "Security by Design" toegepast. 
Dit principe neemt zorgen af op het vlak van securiteit in de IT voor de Citizen Developers, ook wel de werknemers buiten de IT-sector genoemd ~\autocite{Sufi_2023}. 
Volgens ~\textcite{Elshan2023} heeft deze digitalisering en het automatiseren van werkprocessen een positieve invloed op de kwaliteit van het bedrijf, wat 
de securiteit van de bedrijfsprocessen laat stijgen. De reden dat het ook de veiligheid verhoogt is omdat de standaardisatie van de werkprocessen menselijke fouten vermindert ~\autocite{Elshan2023}.
\subsubsection{Universele toegankelijkheid}
\label{subsec:universele-toegankelijkheid}
Eerder in de literatuurstudie is er vermeld dat er een tekort is aan IT-personeel dat over de kwaliteiten beschikt. 
Dit kan door ~\textcite{Sufi_2023} nog eens bevestigd worden waarbij verteld is dat bedrijven falen bij het rekruteren van IT-personeel, door het gebrek aan 
IT'ers met de nodige kennis en ervaring. Volgens ~\textcite{Sufi_2023} is hier een oplossing voor, namelijk Low-Code en No-Code platformen. 
Deze platformen laten niet-programmeurs toe om te werken aan IT-oplossingen zoals: dashboards, applicaties, 
en databanken zonder complicaties ~\autocite{Sufi_2023}. Dit lost het probleem op van het tekort aan IT-personeel binnen het bedrijf, 
maar volgens ~\textcite{Sufi_2023} is dit niet de enige voordeel van universele toegankelijkheid want door LCNC kunnen de niet-programmeurs ook snel de 
benodigde IT-oplossing ontwikkelen, zonder tijd en geld te verspillen.
\\
\\
De reden waarom we kunnen spreken van universele toegankelijkheid is omdat Low-Code en No-Code platformen makkelijk te leren zijn ~\autocite{ALSAADI_2021} ~\autocite{Sufi_2023}. 
Zoals vermeld door ~\textcite{ALSAADI_2021} zullen niet alleen professionele ontwikkelaars LCNC platformen benutten, maar ook de niet-programmeurs. 
Beginners ofwel de niet-programmeurs hebben nu ook de mogelijkheid om applicaties te bouwen zonder kennis van programmeertalen. 
Dit is allemaal mogelijk doordat LCNC platformen een drag-en- drop systeem hebben, hierbij kunnen 
de gebruikers de componenten van de applicatie slepen en neerzetten waarbij vervolgens in de achtergrond de code gegenereerd wordt ~\autocite{ALSAADI_2021}.
\subsubsection{Cloud-Based LCNC: Dataherstel}
\label{subsec:cloud-based-lcnc}
Hedendaags stappen bedrijven over naar cloud-based technologieën omdat ze heel wat voordelen bieden zoals: 
kostenbesparing, schaalbaarheid, flexibiliteit, herstel van data ~\autocite{Sufi_2023}. 
Volgens ~\textcites{Sufi_2023} zijn grotendeels alle LCNC platformen cloud-based wat als gevolg heeft dat de projecten in LCNC platformen telkens automatisch worden opgeslagen.
\\
\\
Doordat het cloud-based is, is het ook mogelijk om data te recupereren in het geval van een ramp ~\autocite{Sufi_2023}. 
Deze platformen verzekeren dat het systeem is geback-upt en dat de data teruggevonden kan worden in het geval van een ramp. 
Dit is een zeer belangrijk voordeel voor bedrijven want volgens ~\textcite{Sufi_2023} kan 20\% van cloud gebruikers hun data in vier uur of minder recupereren, 
terwijl 9\% van de niet cloud gebruikers hun data herstellen in vier uur of minder. Daarnaast riskeren developers, die niet gebruikmaken van de cloud, 
dat ze hun data verliezen op hun computer in het geval van een ramp. 
Dit is niet zo bij cloud-hosted services want volgens ~\textcite{Sufi_2023} verzekeren deze services dat de data altijd beschikbaar is.
\subsection{Nadelen}%
\label{subsec:nadelen}
%Rood gemarkeerd in paper = nadelen
\subsubsection{Beperkte flexibiliteit}
\label{subsec:beperkte-flexibiliteit}
Niet-programmeurs kunnen snel expert zijn in hun gekozen LCNC platform, maar dit betekent ook dat ze vastzitten aan de beperkingen en het framework van het platform 
~\autocite{Sufi_2023} ~\autocite{Talesra_2021}. 
Deze limieten kunnen voor problemen zorgen als de niet-programmeurs een applicatie moeten bouwen dat aanpasbaar moet zijn ~\autocite{Talesra_2021}. 
Vervolgens hebben LCNC platformen ook gelimiteerde opties voor integratie met andere systemen, waardoor het zowel een uitdaging is voor de niet-programmeurs als 
het bedrijf ~\autocite{Talesra_2021}. Daarnaast bevat LCNC ook third party afhankelijkheden, wat als gevolg heeft dat de gebruiker afhankelijk is van de verkoper van de 
third party bij het oplossen van veiligheid en prestatie problemen ~\autocite{Talesra_2021}. Dit is niet het geval bij traditionele programmeertalen zoals Java, C\#, en Python, 
waarbij de developers de vrijheid hebben om de applicatie te bouwen zoals ze zelf willen ~\autocite{Sufi_2023}. Jammer genoeg is er geen bestaande LCNC die in 2023 deze 
flexibiliteit biedt ~\autocite{Sufi_2023}. Dit houdt in dat niet-programmeurs beperkt zijn op het vlak van beschikbare opties bij het ontwikkelen van hun oplossing ~\autocite{Sufi_2023}. 
\\
\\
De beperkingen van LCNC hebben te maken met de platformen die bestaan uit gevisualiseerde componenten die de gebruiker kan slepen en
 neerzetten ~\autocite{Yan2021}. Deze zijn vooraf gedefinieerd, wat als gevolg heeft dat het niet zo aanpasbaar is als een applicatie 
 dat gebouwd is met een programmeertaal ~\autocite{Yan2021}. Volgens ~\textcite{Yan2021} 
is het hierdoor moeilijk en tijdspenderend om ingewikkelde of aanpasbare features of functionaliteiten, die niet door het platform is aangeboden, te ontwikkelen.
\subsubsection*{Gelimiteerde schaalbaarheid}
\label{subsec:gelimiteerde-schaalbaarheid}
Volgens ~\textcite{Elshan2023} en ~\textcite{Sufi_2023} zijn applicaties bouwen met LCNC op dit moment gelimiteerd in schaalbaarheid. Daarom worden deze platformen enkel gebruikt om kleine schaalbare applicaties te ontwikkelen ~\autocite{Sufi_2023}. 
De platformen kunnen door limitatie niet gebruikt worden voor applicaties die in de toekomst zouden moeten uitbreiden ~\autocite{Elshan2023}. 
Volgens ~\textcite{Yan2021} lag, in 2015, de applicaties die gemaakt zijn door LCNC tussen de 200 en 2000 gelijktijdige gebruikers zonder prestatieproblemen.

\subsection*{Veiligheidszorgen}
\label{subsec:veiligheidszorgen}
In de literatuur wordt vermeld dat veiligheid een voordeel is van LCNC platformen, maar volgens ~\textcite{Yan2021} is dit niet altijd het geval. 
Ze kunnen namelijk moeilijk of zelfs niet aangepast worden, waardoor bedrijven, die er gebruik van maken, volledig de diensten van de verkoper 
moeten vertrouwen op het niet maken van veiligheidsproblemen ~\autocite{Yan2021}. 
Doordat de firma volledig afhankelijk is van de verkoper, kan de data van de onderneming in gevaar komen door een datalek bij de verkoper, gezien 
de data security en de source code niet beheerst is door het bedrijf ~\autocite{Yan2021}.

\subsection*{Vendor Lock-in}
\label{subsec:technische-schulden}
Volgens ~\textcite{Yan2021} betekent vendor lock-in dat je als bedrijf afhankelijk bent van de verkoper voor hun diensten en producten, 
wat als gevolg heeft dat het moeilijk is om als klant te veranderen naar een andere verkoper. Het gevaar is dat dit ook zo kan zijn bij LCNC platformen, 
waardoor het bedrijf in de toekomst meer zal investeren in het platform van de verkoper ~\autocite{Yan2021}. 
Volgens ~\textcite{Yan2021} zal dit dan kunnen leiden tot stijgende kosten van de diensten en producten van de verkoper, 
en zal het bedrijf moeilijker kunnen overgaan naar een andere verkoper. Als het bedrijf toch van plan zou zijn om te veranderen, 
zal het bedrijf de applicatie helemaal opnieuw moeten bouwen ~\autocite{Sufi_2023}.  
Volgens ~\textcite{Sufi_2023} is dit te wijten aan twee belangrijke factoren. 
Ten eerste ontbreken bij bestaande LCNC-platformen de mogelijkheden om applicaties die op verschillende LCNC-platformen ontwikkeld zijn, met elkaar te integreren.
Ten tweede heeft elke leverancier zijn eigen unieke ecosysteem voor het ontwikkelen van applicaties, wat uniformiteit en compatibiliteit tussen verschillende platformen beperkt.


\section{LCNC uitdagingen}
\label{sec:lcnc-uitdagingen}
In deze paragraaf komt aanbod wat de obstakels en problemen zijn die overwonnen moeten worden. Deze obstakels en problemen bieden een kans voor groei en
verbetering van de LCNC platformen.
%Oranje gemarkeerd in paper = nadelen
\subsubsection*{Weerstand binnen het bedrijf}
\label{subsec:weerstand-binnen-het-bedrijf}
%(1) in Elshan paper
Wanneer een softwarebedrijf zou beslissen om over te stappen naar Low-Code platformen, kan er weerstand ontstaan binnen het bedrijf ~\autocite{Elshan2023}. 
Volgens ~\textcite{Elshan2023} heeft dit te maken door dat er een misverstand is over Low-Code. 
Werknemers binnen het bedrijf denken dat Low-Code platformen hun werk zal vervangen door de toename van automatisering en digitalisering. 
Volgens ~\textcite{Elshan2023} zullen de IT-medewerkers binnen het bedrijf het moeilijkste zijn om te overtuigen, omdat ontwikkelaars zich zien als een artiest, 
die hun eigen foto's schilderen en deze niet automatisch willen laten maken door een machine ~\autocite{Elshan2023}. 
Vervolgens geeft dit de developer meer onderhoudswerkzaamheden dan ontwikkelingstaken, 
wat als gevolg kan hebben dat ze zich niet uitgedaagd voelen op het vlak van werk en uiteindelijk het bedrijf zullen verlaten. 
Daarnaast kan het gebruik van Low-Code platformen ook een invloed hebben op hun salaris en stress dat bijgevolg leidt tot ontevredenheid.
\subsubsection*{Shadow IT}
\label{subsec:shadow-it}
%(2) in Elshan paper
De betekenis van Shadow IT is al besproken in de literatuurstudie \hyperref[subsec:veiligheid]{Voordelen: Veiligheid}. Kortom, de mensen buiten de IT die gebruikmaken van third party software
zonder de goedkeuring of het toezicht van de IT-afdeling ~\autocite{Yan2021} ~\autocite{Rokis_2022}. 
 Volgens ~\textcite{Elshan2023} kunnen andere afdelingen binnen het bedrijf hun eigen applicaties bouwen met Low-Code platformen, 
 zonder dat de IT-afdeling hiervan op de hoogte is, wat als gevolg heeft dat ze voor elk klein ding iets kunnen ontwikkelen, waardoor er geen overzicht is van de applicaties binnen het bedrijf ~\autocite{Elshan2023}. 
 Daarnaast kunnen voordelen verloren gaan, zoals mindere tijdsduur en kosten, door dubbele applicatie development. 
 Volgens ~\textcite{Elshan2023} is dit alleen de top van de ijsberg want er zijn nog andere serieuze gevolgen die Shadow IT met zich meebrengt. 
 In eerste instantie zou elke applicatie grondige controles moeten ondergaan gaan om de veiligheid en data bescherming te verzekeren, 
 maar dit is niet het geval bij Shadow IT. Het kan zo zijn dat de werknemers onwetend persoonlijke data gebruiken in een applicatie wat niet in overeenstemming 
 komt met de privacywetgeving ~\autocite{Elshan2023}. Volgens ~\textcite{Elshan2023} zou dit kunnen opgelost worden door vooropgestelde regels tot het 
 gebruik van Low-Code platformen, vooropgestelde trainingen, en duidelijk te maken wat de verantwoordelijkheden zijn van de werknemers.

\subsubsection*{Gebrek aan eigen controle}
\label{subsec:gebrek-aan-eigen-controle}
Hedendaags beschikken alle No-Code en Low-Code platformen over een cloud forward of cloud-first aanpak ~\autocite{Sufi_2023}. 
Hierbij implementeren citizen developers, ook wel niet-programmeurs genoemd , hun algoritmes in een cloud gebaseerde interface. 
Maar volgens ~\textcite{Sufi_2023} kunnen niet alle applicaties onderhouden worden in de cloud, zoals bijvoorbeeld applicaties die gevoelige data bevatten 
(inlichtingendienst applicatie, geheime overheid applicaties, enz. ). Deze applicaties zijn niet geschikt voor LCNC-platformen. 
Daarnaast zijn er private bedrijven die gegevens hosten die niet in de cloud gehost kunnen worden vanwege privacywetgeving ~\autocite{Sufi_2023}. 
Dit soort technologische oplossingen zijn alleen mogelijk met een systeem die offline of op locatie werkt.
\subsubsection*{LCNC in Software Development Life Cycle}
\label{subsec:lcnc-binnen-agile}
%(3) in Rokis & Kirikova en voorstel
\textbf{Requirement Analyse}
\\
De specificaties van de platformen verschillen binnen software, wat als gevolg gezien kan worden als een uitdaging ~\autocite{Rokis_2022}. 
Hierdoor is een tool voor eisenbeheer binnen de LCNC platformen gezien als een waardevolle toevoeging ~\autocite{Rokis_2022}. 
In deze fasen binnen softwareontwikkeling is een verandering in de eisen ook gezien als een uitdaging ~\autocite{Rokis_2022}. 
Maar dit zou kunnen opgelost worden met een prototype van de applicatie, 
waardoor altijd de volgende dag een nieuwe versie van de applicatie kan tonen aan de eindklant ~\autocite{Rokis_2022}.
\\
\textbf{Planning}
\\
Volgens ~\textcite{Rokis_2022} bestaat het planningsfase uit het analyseren en plannen van de geschiktheid, complexiteit, risico's, afhankelijkheden en 
de tijdsduur van de ontwikkeling van de applicatie, in een LCNC platform. De uitdagingen binnen deze fase is gerelateerd aan de geschiktheid van de LCNC, 
omdat er een groot aantal aan platformen beschikbaar is op de markt ~\autocite{Rokis_2022}. Hierbij is er gekeken naar de gerelateerde kosten , de 
snelheid van het leren, de ondersteunde features, en de functionaliteiten in development van de LCNC platformen ~\autocite{Rokis_2022}. 
Daarnaast zijn ook bedrijven bezorgt over vendor lock-in ~\autocite{Rokis_2022}. Maar door de overvloed aan platformen is het moeilijk om 
de meest verenigbare platform te kiezen voor de applicatie ~\autocite{Rokis_2022}.
\\
\textbf{Application Design}
\\
Het ontwerp van een applicatie hangt af van de eisen van de eindklant ~\autocite{Rokis_2022}. 
Het bestaat uit de architectuur, de modulariteit, schaalbaarheid en andere aspecten van de applicatie ~\autocite{Rokis_2022}. 
In deze fasen zijn er verschillende uitdagingen zoals de limieten van uitbreidbaarheid en integratie met andere soorten platformen, data opslag, en de gebruikersinterface 
~\autocite{Rokis_2022}. De limieten van uitbreidbaarheid heeft te maken met dat de platform vooropgestelde functionaliteiten heeft waarbij de gebruiker deze grotendeels 
niet kan aanpassen ~\autocite{Rokis_2022}. Bij de integratie met andere soorten platformen is er een vermogen van informatie uitgewisseld tussen onderdelen van de applicatie 
of met externe diensten ~\autocite{Rokis_2022}. Omdat er een gebrek is aan algemene afspraken en de software 'closed source' is maakt het dit een lastig punt ~\autocite{Rokis_2022}.
 Dit geeft tot gevolg dat het moeilijk is om systemen met elkaar goed te laten samenwerken ~\autocite{Rokis_2022}. 
 Wat als gevolg heeft dat er beperkingen zijn in systemen ontwerpen en het delen van ontwikkelde diensten en producten ~\autocite{Rokis_2022}. 
 Vervolgens hebben we ook nog de uitdaging van de gebruikersinterface en de data opslag ~\autocite{Rokis_2022}. Volgens ~\textcite{Rokis_2022} 
 ondervinden de gebruikers van Low-Code platformen dat ze tijdens het maken van applicaties uitdagingen tegenkomen met betrekking tot het ontwerp 
 van de gebruikersinterface en de data opslag. De reden hiervoor is omdat het maken van hoe een applicatie eruitziet en werkt, specialistische kennis vereist ~\autocite{Rokis_2022}. 
 Wat als gevolg heeft dat de gebruikers die geen achtergrond ervaring hebben van applicaties ontwikkelen dit een lastige taak vinden ~\autocite{Rokis_2022}. 
 Vervolgens is dit hetzelfde verhaal bij het opzetten van een systeem voor de data op te slaan, want men moet namelijk een opslaglocatie kiezen en dan ook nog gegevens 
 kunnen verplaatsen naar de opslaglocatie ~\autocite{Rokis_2022}. Volgens ~\textcite{Rokis_2022} zou de documentatie verbeteren en het toegankelijk maken voor iedereen 
 een oplossing kunnen zijn voor het opslaan van data.
\\
\textbf{Development}
\\
In deze fase is de applicatie gebouwd ~\autocite{Rokis_2022}. 
Volgens ~\textcite{Rokis_2022} ervaren gebruikers hier uitdagingen zoals de gebruikersinterface aanpassen, de business logica uitvoeren, de applicatie met andere systemen integreren of zelfs complexere ontwikkelingen waarbij toegang tot de code nodig is. 
De reden voor de eerste drie opgesomde uitdagingen, is omdat er niet complete of niet-correcte documentatie is ~\autocite{Rokis_2022}. 
Maar niet enkel de documentatie is een probleem, ook is er een gebrek aan leermateriaal en trainingen zoals tutorials van de LCNC platformen ~\autocite{Rokis_2022}.
\\
\textbf{Testing}
\\
Hier wordt de applicatie getest om te verifiëren of het voldoet aan de eisen van de eindklant ~\autocite{Rokis_2022}. 
Volgens ~\textcite{Rokis_2022} ervaren hier de gebruikers ook uitdagingen zoals een gebrek aan documentatie voor geautomatiseerde testen uit te voeren, 
test coverage, en het gebruik kunnen maken van third party test tools ~\autocite{Rokis_2022}. 
Volgens ~\textcite{Rokis_2022} hebben Low-Code platformen een beperkte Low-Code test frameworks en analysemogelijkheden. 
Daarnaast wordt testen van niet-functionele vereisten in Low-Code platformen vaak verwaarloost ~\autocite{Rokis_2022}. 
Volgens ~\textcite{Rokis_2022} zou dit kunnen opgelost worden door het onderzoeken naar een invoering van een Low-Code test framework dat alle testactiviteiten zou dekken ~\autocite{Rokis_2022}.
\\
\textbf{Deployment}
\\
Bij het publiceren van de applicatie ervaren de gebruikers uitdagingen zoals complicaties bij het configureren, toegankelijkheid, en performantie problemen ~\autocite{Rokis_2022}. 
Dit heeft weer als oorzaak een gebrek aan goede documentatie van de LCNC platformen ~\autocite{Rokis_2022}.
\\
\textbf{Maintenance}
\\
Uit eerder besproken literatuur kan men opmaken dat de onderhoud van applicaties zeer weinig vraagt, dit kan ook nog eens bevestigd worden door ~\textcite{Rokis_2022}.
Volgens ~\textcite{Rokis_2022} komt men hier weer terecht bij de uitdaging van debuggen in een grafische representatie.


\section{LCNC binnen bedrijven}
\label{sec:lcnc-bedrijven}
Dit zal inzicht geven in waarom deze platformen worden gebruikt binnen bedrijven. 
Het zal een betere duidelijkheid geven waarom deze bachelorproef belangrijk is voor software bedrijven, vooral Quivvy Solutions BV.
\\
\\
Diverse organisaties hangen af van software om te kunnen functioneren ~\autocite{Hintsch2021}. 
Dit kan volgens ~\textcite{Rafiq_2022} ook opgemerkt worden bij zowel start-ups als middelgrote en grote bedrijven, waarbij start-ups jonge bedrijven 
zijn met een gelimiteerd aantal middelen tot hun beschikking ~\autocite{Rafiq_2022}. Deze soort bedrijven kom dagelijks in contact met uitdagingen zoals tijdsdruk, 
team formatie, en snelgroeiende markten ~\autocite{Rafiq_2022}. Om dit allemaal te weerstaan maken start-ups bedrijven gebruik van No-Code en Low-Code applicatie 
development platform ~\autocite{Rafiq_2022}. De reden waarom start-ups LCNC platformen gebruiken en wat het verschil is met middelgrote en grote bedrijven kan 
uitgelegd worden door het onderzoek van ~\textcite{Rafiq_2022}. Het onderzoek van ~\textcite{Rafiq_2022} is uitgevoerd bij twee bedrijven, een software start-up en 
een groot bedrijf dat op meerdere plaatsen gevestigd is. Uit het onderzoek van ~\textcite{Rafiq_2022} maakte het start-up bedrijf gebruik van LCNC platformen maar 
het grote bedrijf alleen van Low-Code platformen. Het combineren van Low-Code en No-code platformen brengt voordelen met zich mee zoals ten eerste het maken 
an een prototype, het gebruiken voor ontwerp, en als laatste voor het uitvoeren van een dienst ~\autocite{Rafiq_2022}. Verder in het analyse is er opgemerkt dat start-ups 
bedrijven deze platformen niet gebruiken voor hun hoofdproduct ~\autocite{Rafiq_2022}. Ten slotte is Low-Code platformen in grote bedrijven gebruikt voor snelle ontwikkeling 
van de applicatie, snelle feedback, en de mindere werklast. ~\autocite{Rafiq_2022}.
\section{Gebruik van LCNC door niet-programmeurs}
\label{sec:lcnc-niet-programmeurs}
Dit is een opvolging van het vorige hoofdstuk 'LCNC binnen bedrijven'. Omdat het belangrijk is dat niet-programmeurs ook met deze platformen kan werken.
Meer specifiek naar de ervaringen van personen met voldoende  tot geen ervaring in het programmeren die deze platformen gebruiken.
\\
\\
Volgens ~\textcite{Yan2021} kunnen ook niet-programmeurs gebruik maken van Low-Code en No-Code platformen.
Vervolgens werd een studie gedaan over hoe niet-programmeurs tegenover ervaren ontwikkelaars performeren bij het gebruik van Low-Code Development  in 2020, volgens ~\textcite{Hintsch2021} 
.Uit de studie bleek dat ervaren ontwikkelaars moeilijkheden hadden bij het identificeren van belangrijke concepten binnen software engineering in het platform ~\autocite{Hintsch2021}.
Hoewel ontwikkelaars uitdagingen ervaarden was dit ook het geval bij niet-programmeurs, waarbij de niet-programmeurs moeilijkeheden hadden bij eenvoudige taken zoals het maken van een scherm  ~\autocite{Hintsch2021}.
Daarnaast was de connectie met de database leggen en de 'parameter passing' ook moeilijk en verwarrend voor de niet-programmeur, bij het ontwikkelen van de applicatie  ~\autocite{Hintsch2021}.

% Tip: Begin elk hoofdstuk met een paragraaf inleiding die beschrijft hoe
% dit hoofdstuk past binnen het geheel van de bachelorproef. Geef in het
% bijzonder aan wat de link is met het vorige en volgende hoofdstuk.

% Pas na deze inleidende paragraaf komt de eerste sectiehoofding.

%Dit hoofdstuk bevat je literatuurstudie. De inhoud gaat verder op de inleiding, maar zal het onderwerp van de bachelorproef *diepgaand* uitspitten. De bedoeling is dat de lezer na lezing van dit hoofdstuk helemaal op de hoogte is van de huidige stand van zaken (state-of-the-art) in het onderzoeksdomein. Iemand die niet vertrouwd is met het onderwerp, weet nu voldoende om de rest van het verhaal te kunnen volgen, zonder dat die er nog andere informatie moet over opzoeken \autocite{Pollefliet2011}.

%Je verwijst bij elke bewering die je doet, vakterm die je introduceert, enz.\ naar je bronnen. In \LaTeX{} kan dat met het commando \texttt{$\backslash${textcite\{\}}} of \texttt{$\backslash${autocite\{\}}}. Als argument van het commando geef je de ``sleutel'' van een ``record'' in een bibliografische databank in het Bib\LaTeX{}-formaat (een tekstbestand). Als je expliciet naar de auteur verwijst in de zin (narratieve referentie), gebruik je \texttt{$\backslash${}textcite\{\}}. Soms is de auteursnaam niet expliciet een onderdeel van de zin, dan gebruik je \texttt{$\backslash${}autocite\{\}} (referentie tussen haakjes). Dit gebruik je bv.~bij een citaat, of om in het bijschrift van een overgenomen afbeelding, broncode, tabel, enz. te verwijzen naar de bron. In de volgende paragraaf een voorbeeld van elk.

%\textcite{Knuth1998} schreef een van de standaardwerken over sorteer- en zoekalgoritmen. Experten zijn het erover eens dat cloud computing een interessante opportuniteit vormen, zowel voor gebruikers als voor dienstverleners op vlak van informatietechnologie~\autocite{Creeger2009}.

%Let er ook op: het \texttt{cite}-commando voor de punt, dus binnen de zin. Je verwijst meteen naar een bron in de eerste zin die erop gebaseerd is, dus niet pas op het einde van een paragraaf.

%\lipsum[7-20]
