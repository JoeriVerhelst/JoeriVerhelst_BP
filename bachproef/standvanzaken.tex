\chapter{\IfLanguageName{dutch}{Stand van zaken}{State of the art}}%
\label{ch:stand-van-zaken}

\section{Inleiding}%
\label{sec:inleiding}
Projecten in softwarebedrijven gaan meestal over het budget. Daarbij doet het project meestal niet volledig wat de eindklant verwacht en vervolgens
doet het opgeleverde product niet wat het zou moeten doen ~\autocite{Moskal_2021}. Maar volgens ~\textcite{Moskal_2021} zijn deze problemen niet alleen op te merken 
in de softwarewereld maar ook in andere categorieën binnen de IT-sector, of bij het implementeren en ontwerpen van software systemen. In deze tijdsperiode worden er steeds meer
data verwerkt en ook opgeslaan ~\autocite{Moskal_2021}. Volgens ~\textcite{Moskal_2021} en ~\textcite{Parviainen_2022} veroorzaakt dit proces van het verwerken van data en opslaan
verandingen in de business, door de adoptie van digtale technologieën als gevolg tot digitalisering. Maar digitalisering betekent dat men bestaande producten of diensten moeten omzetten
naar een digitaal product of dienst, het kan ook zo zijn dat men software producten moeten kopen om de business processen te automatiseren ~\autocite{Moskal_2021}. Het verkrijgen van een competitief voordeel
kan bereikt worden door een specifiek ontworpen en ontwikkeld softwareoplossing dat voldoet aan de unieke behoeften van de eindklant ~\autocite{Moskal_2021}. Maar volgens ~\textcite{Moskal_2021} is een toegewijde IT-oplossing
zeer duur waardoor heel wat bedrijven dit niet kan veroorloven. Dit zet No-Code en Low-Code platformen in de spotlight  ~\autocite{Moskal_2021}.

\section{Reden tot gebruik}%
\label{sec:reden-tot-gebruik}
In dit hoofdstuk van de literatuurstudie gaan we ons verdiepen in de reden waarom Low-Code en No-Code platformen toch stijgen ze in gebruik. Dit zal helpen
om een beter beeld te krijgen waarom dit soort platformen, op dit moment, toch waard zijn om eens een kijkje in te nemen, voor zowel het bedrijf als de eindklant van het bedrijf.

In vergelijking met andere technologie trends zoals AI, Blockchain, Edge Computing en RPA groeit Low-Code en No-Code platformen zeer matig ~\autocite{Kulkarni_2021}.
Dit komt omdat het idee van Low-Code en No-Code development niet nieuw is, maar toch is er een stijging in het gebruik van deze platformen ~\autocite{Elshan2023}.
Volgens ~\textcite{Elshan2023} en ~\textcite{Kulkarni_2021} zijn er verschillende redenen waarom deze platformen toch stijgen in gebruik, vooral in kleine en middelgrote bedrijven.
\begin{itemize}
    \item \textbf{Beperkt aantal programmeurs}: 
    Low-Code platformen werden geïntroduceerd als een oplossing voor de dilemma tussen het te kort aan programmeurs en de hoge vraag naar softwareontwikkeling ~\autocite{ALSAADI_2021}. Volgens 
    ~\textcite{Moskal_2021} is de reden tot te kort aan programmeurs te wijten aan dat heel wat studenten zich laat afschrikken door de complexiteit van het programmeren. Dit geeft als gevolg dat weinig 
    studenten een diepe kennis hebben op het vlak van programmeren. Maar volgens ~\textcite{Moskal_2021} brengt een zeer gekwalificeerde programmeur het project niet altijd tot een goed einde.
    \item \textbf{Technologische turbulentie}:
    Doordat programmeertalen steeds veranderen en er steeds nieuwe technologieën worden geïntroduceerd is de kennis van de programmeurs niet altijd up-to-date ~\autocite{Moskal_2021}.
    \item \textbf{Hoge kosten}:
    De tradionele softwareontwikkeling eist een grote tol op de financiën van een bedrijf. Daarbij beseffen Software engineers dat het bouwen van een applicatie
    niet gemakkelijk is binnen het gegeven budget  ~\autocite{Moskal_2021}. Volgens ~\textcite{Elshan2023} zijn concepten zoals DevOps en BizOps, die de operations, developers en bussiness teams samenbrengen,
    overspoeld met moeilijkheden. Als gevolg van kosten dat over het budget gaat en requirement conflicten tussen de teams ~\autocite{Elshan2023}. Low-Code en No-Code platformen kunnen hier een
    oplossing bieden omdat het minder tijd en geld kost om een applicatie te bouwen ~\autocite{Elshan2023} ~\autocite{Bock_2021} ~\autocite{Rokis_2023}.
    \item \textbf{Tijdrovend}:
    Het ontwikkelen van software is een tijdrovend proces. De geschatte tijd dat nodig is om een applicatie te bouwen, op één operating systeem, is vaak
     zes maanden of langer ~\autocite{Moskal_2021}. Het probleem hiervan is dat snelheid binnen de business een belangrijke factor is ~\autocite{Sanchis_2019}.
     Want volgens ~\textcite{Sanchis_2019} betekent een veranderende markt dat bedrijven snel en flexibel moeten kunnen reageren op verandering om aan de requirements van de omgeving te kunnen voldoen.

    \item \textbf{Klantentevredenheid}:
    Low-Code en No-Code platformen zorgen voor een hogere klantentevredenheid ~\autocite{Elshan2023}.
    Volgens ~\textcite{Elshan2023} ontstaat er een soort van rolomkering plaats in het ontwikkelingsproject. De ontwikkelaars van het bedrijf worden niet langer meer gezien als de enige
    opdrachtgevers van een applicatie, maar ook de eindklanten van het bedrijf. Hierdoor kan de eindklant zelf de applicatie aanpassen naar zijn eigen wensen en behoeften ~\autocite{Elshan2023}.
    Volgens ~\textcite{Elshan2023} reduceert dit ook de misverstanden tussen de ontwikkelaars en de eindklant.
    \item \textbf{Digitale Transformatie}: 
    Niet alleen in de IT-sector maar ook in de bussines sector is er een digitale transformatie aan de gang. 
    Deze transformatie binnen de business omgeving zorgt voor een nood aan automatiseren in verschillende aspecten van de business ~\autocite{ALSAADI_2021}.
    Wat als gevolg heeft dat men papier documenten vervangen door digitale documenten om de fysieke processen te vervangen door digitale processen ~\autocite{ALSAADI_2021}.
    Volgens ~\textcite{ALSAADI_2021} merkt men op dat er een daling is van de nood aan menselijke tussenkomst in de processen omdat de automatisering gebruik maakt van vertrouwlijke software 
    dat minder fouten maakt en ook nog eens minder kosten met zich meebrengt ~\autocite{ALSAADI_2021}.
    \item \textbf{Complexe softwareontwikkeling}
  \end{itemize} 

\section{Voordelen en Nadelen}
\label{sec:voordelen-nadelen}
Op gevolg van het vorige hoofdstuk "Reden tot gebruik" zal er nu gekeken worden naar wat de voordelen en nadelen zijn van Low-Code en No-Code platformen. Dit zal
een overzicht geven van wat er allemaal mogelijk is en wat de limieten zijn van deze platformen.
%Hier komt de "voordelen en nadelen" van low-code en no-code platformen Beperkt aantal programmeurs, .... maar met meer tekst (gebruik voorstel)
\subsection{Voordelen}%
\label{subsec:voordelen}
\subsubsection{Snelheid}
\label{subsec:snelheid}
Zoals eerder besproken in het vorige hoofdstuk neemt de tradionele softwareontwikkeling veel tijd in beslag ~\autocite{Moskal_2021}. Dit is niet het geval 
bij LCNC platformen, wat volgens ~\textcite{Adrian_2020} een groot voordeel is. Vervolgens kan dit ook nog eens versterkt worden door ~\textcite{Yan2021}
die verteld dat heel wat bedrijven die gebruikmaken van Low-Code platformen vaststellen dat hun release van een applicatie bij 5 van de 10 keer sneller was dan voorheen.
In 2019 werd er dan ook een enquête gehouden van OutSystems, waaruit bleek dat gebruikers van Low-Code platformen 68\% van hun webapplicaties en 64\% van hun apps elk konden
bouwen in vier maanden ~\autocite{Yan2021}. Tegenover de traditionele ontwikkeling is dit een positief resultaat, want volgens ~\textcite{Yan2021}
was er maar 57\% van de webapplicaties en 49\% van de apps gebouwd binnen dezelfde tijdspanne.
\\
\\
Niet alleen het bouwen van applicaties gaat sneller, maar volgens ~\textcite{da_Cruz_2021} is het grootste voordeel
van Low-Code en No-Code platformen te danken aan het eenvoudig bouwen van complexe software waardoor bedrijven
sneller kunnen reageren op de veranderende markt. Daarnaast is het leren van LCNC platformen gemakkelijker en sneller dan het leren van 
een programmeertaal. Al deze eigenschappen zorgen ervoor dat developers sneller een prototype kunnen maken, feedback krijgen en een betere
klantenervaring kunnen bieden ~\autocite{da_Cruz_2021}.

\subsubsection{Veiligheid}
\label{subsec:veiligheid}
Het gebrek aan werknemers binnen de IT-sector, die heel wat massa aan software moeten ontwikkelen, zorgt ervoor dat de werknemers buiten de IT-sector
er vaak alleen voor staan waardoor ze gebruik moeten maken van third-party software ~\autocite{Yan2021}. Volgens ~\textcite{Yan2021} is dit een groot probleem,
want dit kan schade brengen op het bedrijf ~\autocite{Yan2021}. De oorzaak hiervan zou kunnen zijn dat ze niet op de hoogte zijn van de licentie en veiligheid, dit 
wordt ook wel "Shadow IT" genoemd ~\autocite{Yan2021} ~\autocite{Rokis_2022}. Daarom zorgt LCNC platformen, die geautoriseerd zijn door de IT-sector,
op vermindering van "Shadow IT" ~\autocite{Yan2021}. Daarnaast kan de werknemers buiten de IT-sector makkelijk een oplossing ontwikkelen met Low-Code en No-Code platformen  ~\autocite{Yan2021}.
Als gevolg dat de IT medewerkers niet telkens verstoord worden door andere werknemers ~\autocite{Yan2021}.  Dit biedt verschillende voordelen aan de IT-sector zoals het verminderen van de werkdruk en het verhogen van de veiligheid, want
LCNC platformen bevat de internationale standaarden voor veiligheid (ISO/IEC 27001, PCIDSS) ~\autocite{Sufi_2023}.
\\
\\
Volgens ~\textcite{Sufi_2023} wordt er ook hedendaags een principe genaamd "Security by Design" toegepast. Dit principe neemt heel wat zorgen af op het vlak van veligheid
in de IT voor de Citizen Developers, ook wel de werknemers buiten de IT-sector genoemd ~\autocite{Sufi_2023}.

\subsection{Nadelen}%
\label{subsec:nadelen}

\section{LCNC uitdagingen}%
\label{sec:lcnc-uitdagingen}

\section{LCNC binnen bedrijven}
\label{sec:lcnc-bedrijven}
Dit hoofdstuk zal meer inzicht geven in waarom deze platformen worden gebruikt binnen bedrijven. Dit zal een betere duidelijkheid geven
waarom deze bachelorproef relevant is voor software bedrijven, meer specifiek Quivvy Solutions BV.

\section{Gebruik van LCNC door eindklanten}
\label{sec:lcnc-eindklanten}
Dit is een opvolging van het vorige hoofdstuk "LCNC binnen bedrijven". Omdat het belangrijk is dat de eindklant ook met deze platformen moet kunnen werken
zal er in dit hoofdstuk gekeken worden naar vorige onderzoeken waarbij men spreekt over de eindklant. 
Meer specifiek naar de ervaringen van personen met weinig tot geen ervaring in het programmeren die deze platformen gebruiken.


% Tip: Begin elk hoofdstuk met een paragraaf inleiding die beschrijft hoe
% dit hoofdstuk past binnen het geheel van de bachelorproef. Geef in het
% bijzonder aan wat de link is met het vorige en volgende hoofdstuk.

% Pas na deze inleidende paragraaf komt de eerste sectiehoofding.

Dit hoofdstuk bevat je literatuurstudie. De inhoud gaat verder op de inleiding, maar zal het onderwerp van de bachelorproef *diepgaand* uitspitten. De bedoeling is dat de lezer na lezing van dit hoofdstuk helemaal op de hoogte is van de huidige stand van zaken (state-of-the-art) in het onderzoeksdomein. Iemand die niet vertrouwd is met het onderwerp, weet nu voldoende om de rest van het verhaal te kunnen volgen, zonder dat die er nog andere informatie moet over opzoeken \autocite{Pollefliet2011}.

Je verwijst bij elke bewering die je doet, vakterm die je introduceert, enz.\ naar je bronnen. In \LaTeX{} kan dat met het commando \texttt{$\backslash${textcite\{\}}} of \texttt{$\backslash${autocite\{\}}}. Als argument van het commando geef je de ``sleutel'' van een ``record'' in een bibliografische databank in het Bib\LaTeX{}-formaat (een tekstbestand). Als je expliciet naar de auteur verwijst in de zin (narratieve referentie), gebruik je \texttt{$\backslash${}textcite\{\}}. Soms is de auteursnaam niet expliciet een onderdeel van de zin, dan gebruik je \texttt{$\backslash${}autocite\{\}} (referentie tussen haakjes). Dit gebruik je bv.~bij een citaat, of om in het bijschrift van een overgenomen afbeelding, broncode, tabel, enz. te verwijzen naar de bron. In de volgende paragraaf een voorbeeld van elk.

\textcite{Knuth1998} schreef een van de standaardwerken over sorteer- en zoekalgoritmen. Experten zijn het erover eens dat cloud computing een interessante opportuniteit vormen, zowel voor gebruikers als voor dienstverleners op vlak van informatietechnologie~\autocite{Creeger2009}.

Let er ook op: het \texttt{cite}-commando voor de punt, dus binnen de zin. Je verwijst meteen naar een bron in de eerste zin die erop gebaseerd is, dus niet pas op het einde van een paragraaf.

\lipsum[7-20]
