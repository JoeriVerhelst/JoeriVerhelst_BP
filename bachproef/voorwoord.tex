%%=============================================================================
%% Voorwoord
%%=============================================================================

\chapter*{\IfLanguageName{dutch}{Woord vooraf}{Preface}}%
\label{ch:voorwoord}

Hedendaags spelen softwarebedrijven een cruciale rol in de digitale wereld. 
Door de exponentiele groei van digitalisatie moeten softwarebedrijven, vooral kleine bedrijven, 
een stapje hogerop op het vlak van snelle applicatieontwikkeling. Hierdoor hebben softwarebedrijven nood aan 
een platform die toelaat om web en mobile toepassingen te creëren zonder diepgaande kennis voor programmeren. 
Twee prominente spelers zijn Stacker en Softr. Deze platformen bieden de mogelijkheid aan om krachtige apps te bouwen 
door middel van een drag-and-drop systeem met integratie mogelijkheden. In deze studie worden deze platformen op de proef gezet, 
met nog een alternatief platform, om te bepalen welke platform het meest geschikt is voor Quivvy Solutions.
\\
\\
Het onderwerp vond ik interessant doordat mijn opleiding Toegepaste Informatica aan de Hogeschool Gent niks geeft over Low-Code en No-Code platformen.
Dit bracht me in een onbekend terrein waar ik zelf moest uitzoeken wat deze platformen inhouden en hoe ze werken. Dit onderwerp werd dan ook toegelicht aan mij via 
het softwarebedrijf Quivvy Solutions, waar ik mijn stage heb gedaan.
\\
Ik zou graag als eerste mijn copromotor bedanken die mij geholpen heeft met het opstellen van de vereisten en de testen maar ook
in het algemeen professionele advies gaf. Vervolgens wil ik mijn promotor bedanken die mij heeft bijgestaan met het opstellen en verbeteren van de bachelorproef. Daarnaast 
wil ten zeerste mijn ouders en broer bedanken voor hun steun en hulp bij mijn Proof of Concept. Als laatste wil ik mijn moeder bedanken voor het nalezen en advies geven op vlak van grammatica van mijn bachelorproef.
%% TODO:
%% Het voorwoord is het enige deel van de bachelorproef waar je vanuit je
%% eigen standpunt (``ik-vorm'') mag schrijven. Je kan hier bv. motiveren
%% waarom jij het onderwerp wil bespreken.
%% Vergeet ook niet te bedanken wie je geholpen/gesteund/... heeft