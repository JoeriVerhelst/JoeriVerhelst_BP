%---------- Inleiding ---------------------------------------------------------

\section{Introductie}%
\label{sec:introductie}

Bedrijven streven naar effeciëntie, kosten  verminderen, en veiligheid. Dit is ook zo bij het softwarebedrijf genaamd Quivvy waarbij de trend No-Code en Low-Code
platformen voor het bouwen van applicaties de laatste jaren wordt gehanteerd. Momenteel gebruikt Quivvy FlutterFlow voor het bouwen van Mobile \& Web Portals. 
Hierbij interageert FlutterFlow met AirTable, een database platform. Maar bedrijven zijn voortdurend op zoek naar de beste softwareplatformen om hun bedrijf te runnen.
Daarom zoekt Quivvy naar een alternatief voor het bouwen van Mobile \& Web Portals dat zowel goed werkt voor bedrijf als eindklant, maar ook interageert met AirtTable of Podio.
De markt van No-Code en Low-Code platforms worden overspoeld met verschillende platformen waaronder Microsoft PowerApps, Bubble, Webflow, Softr en Stacker.
Als gevolg dat drie op de vier van de grootste bedrijven tegen 2024 gebruik zullen maken van Low-Code platforms ~\autocite{Moskal_2021} ~\autocite{Kulkarni_2021}.
Deze No-Code en Low-Code platforms hebben ook een reden tot de populariteit, namelijk dat
deze platforms het mogelijk maken om applicaties te bouwen zonder enige tot weinig kennis van programmeren. Dit is niet het enige waarom deze platforms een trend zijn,
want het zorgt ook voor een snelle ontwikkeling met lage kosten door het effeciënte gebruik van de ontwikkelaars.
Maar welke bestaande softwareplatform is geschikt om Mobile \& Web portals te creëren, dat zowel goed werkt voor bedrijf als eindklant?
In deze bachelorproef hanteren we hiervoor het meest geschikte.
Hierbij maken we een vergelijking tussen Softr en Stacker, beide No-Code en/of Low-Code platforms. Daarbij wordt er rekening gehouden met verschillende 
factoren zoals snelheid, gebruiksvriendelijkheid, enzovoort.


%Waarover zal je bachelorproef gaan? Introduceer het thema en zorg dat volgende zaken zeker duidelijk aanwezig zijn:

% \begin{itemize}
%   \item kaderen thema
%   \item de doelgroep
%   \item de probleemstelling en (centrale) onderzoeksvraag
%   \item de onderzoeksdoelstelling
% \end{itemize}

% Denk er aan: een typische bachelorproef is \textit{toegepast onderzoek}, wat betekent dat je start vanuit een concrete probleemsituatie in bedrijfscontext, een \textbf{casus}. Het is belangrijk om je onderwerp goed af te bakenen: je gaat voor die \textit{ene specifieke probleemsituatie} op zoek naar een goede oplossing, op basis van de huidige kennis in het vakgebied.

% De doelgroep moet ook concreet en duidelijk zijn, dus geen algemene of vaag gedefinieerde groepen zoals \emph{bedrijven}, \emph{developers}, \emph{Vlamingen}, enz. Je richt je in elk geval op it-professionals, een bachelorproef is geen populariserende tekst. Eén specifiek bedrijf (die te maken hebben met een concrete probleemsituatie) is dus beter dan \emph{bedrijven} in het algemeen.

% Formuleer duidelijk de onderzoeksvraag! De begeleiders lezen nog steeds te veel voorstellen waarin we geen onderzoeksvraag terugvinden.

% Schrijf ook iets over de doelstelling. Wat zie je als het concrete eindresultaat van je onderzoek, naast de uitgeschreven scriptie? Is het een proof-of-concept, een rapport met aanbevelingen, \ldots Met welk eindresultaat kan je je bachelorproef als een succes beschouwen?

%---------- Stand van zaken ---------------------------------------------------

\section{State-of-the-art}%
\label{sec:state-of-the-art}

 In de software wereld merkt men in projecten dat er telkens over het budget wordt gegaan. Daarnaast doet het project meestal niet wat de klant verwacht 
 en vervolgens  doet het opgeleverde product niet wat het moet doen ~\autocite{Moskal_2021}. Volgens ~\textcite{Moskal_2021} zijn deze problemen niet alleen  te zien in de software wereld, 
 maar ook in andere categorieën binnen de IT-sector, of bij het implementeren en ontwerpen van software systemen. 
 Dit brengt het onderwerp Low-Code en No-Code naar boven ~\autocite{Kulkarni_2021}. 
 Deze platforms zorgen ervoor dat u geen tot weinig kennis nodig heeft over programmeren waardoor deze wordt beschouwd als een trend waar heel wat mensen in geïnteresseerd zijn ~\autocite{Kulkarni_2021}.
 Maar in vergelijking met andere technologie trends zoals AI, Blockchain, Edge Computing, en RPA groeit LCNC zeer matig in vergelijking met de andere trends ~\autocite{Kulkarni_2021}.
 Toch stijgt het gebruik van No-Code platforms, maar waarom? Dit kan men onderverdelen in vier categorieën:
\begin{itemize}
  \item \textbf{Beperkt aantal programmeurs}: 
  Heel wat studenten laten zich afschrikken door het programmeren. 
  Hierdoor is er een te kort aan studenten die werkelijk diepe kennis hebben op vlak van programmeren. 
  Maar het vinden van een zeer gekwalificeerde programmeur brengt projecten niet altijd tot een goed einde ~\autocite{Moskal_2021}.
  \item \textbf{Technologische turbulentie}: De constante evolutie van programmeertalen zorgt ervoor dat de kennis van programmeurs niet altijd de meest recente zijn  ~\autocite{Moskal_2021}.
  \item \textbf{Hoge kosten}: De traditionele softwareontwikkeling eist een grote tol op de financiën van een bedrijf. Daarbij beseffen Software engineers dat het bouwen van een applicatie niet gemakklijk is binnen een budget ~\autocite{Moskal_2021}.
  \item \textbf{Tijdrovend}: De traditionele softwareontwikkeling is een tijdrovend proces. De geschatte tijd om Software te maken op één operating systeem is zes maanden of zelfs langer ~\autocite{Moskal_2021}.
  \item \textbf{Complexe softwareontwikkeling}
\end{itemize} 
Het gebruik van het woord No-Code development wordt gezien als een synoniem voor Low-Code development,
daarom kan men verwijzen naar low-code als no-code maar ook omgekeerd ~\autocite{Rokis_2022}. 

\subsection*{LCNC Uitdagingen}
\label{sub:lcnc-uitdagingen}
Om grondig door alle LCNC uitdagingen te gaan, zullen we ze opdelen in bepaalde fasen
die gebeuren tijdens het ontwikkelen van een applicatie. 
Deze zijn; requirements analyse, planning, application design, development, testing, deployment, en maintenance ~\autocite{Rokis_2022}.
\subsubsection*{Requirements Analyse}
\label{sub:requirements-analyse}
Doordat specificaties verschillen per platform binnen software, kan dit als een uitdaging worden gezien ~\autocite{Rokis_2022}.
Hierdoor wordt een tool voor eisenbeheer binnen LCNC beschouwd als een waardevolle toevoeging ~\autocite{Rokis_2022}. 
In deze fase worden veranderingen in de eisen ook gezien als een struikelblok. Maar door het gebruik van LCNC kan dit worden opgelost 
met een prototype, waarbij men steeds op de volgende dag snel een oplossing kan leveren en onderhouden ~\autocite{Rokis_2022}.
\subsubsection*{Planning}
\label{sub:planning}
In de planningsfase komen we terecht bij het kiezen van de meest compatibele LCNC platform die aan de eisen goed voldoen ~\autocite{Rokis_2022}.
Volgens ~\textcite{Rokis_2022} maakt de overvloed van de beschikbare platforms op de markt het zeer moeilijk om de meest compatibele te vinden. De 
kosten, leercurve, ondersteuning van functies en functionaliteiten in ontwikkeling spelen allemaal een belangrijke rol ~\autocite{Rokis_2022}. Daarnaast zijn vele
bedrijven bezorgd over "vendor-lock-in". De klant is daarbij sterk afhankelijk van een bepaalde levarancier, waardoor het moeilijk is om over te stappen ~\autocite{Rokis_2022} ~\autocite{Yan2021}.
\subsubsection*{Application Design}
\label{sub:application-design}
De applicatie ontwerp is redelijk gelimmiteerd binnen LCNC.
Ten eerste hebben we te maken met het probleem dat de meeste LCNC platforms niet uitbreidbaar zijn ~\autocite{Rokis_2022}.
Als tweede zijn er overwegingen over de betrekking tot schaalbaarheid ~\autocite{Rokis_2022}. 
Vervolgens kan er een beperking zijn op het vlak gegevens opslaan en het ontwerp van de gebruikersinterface ~\autocite{Rokis_2022}.
\subsubsection*{Development}
\label{sub:development}
Bij het ontwikkelen van de software kunnen verschillende uitdagingen voorkomen.
Volgens ~\textcite{Rokis_2022} kan de gelimmiteerde functionaliteit van LCNC platformen een probleem vormen.
Hierdoor moet men dan zelf code schrijven waardoor men meer tijd moet spenderen en complexiteit toevoegen aan het project ~\autocite{Rokis_2022}.
Daarnaast zouden we ook rekening moeten houden met de moeilijkheid op het vlak debuggen in een grafische representatie ~\autocite{Rokis_2022}.
\subsubsection*{Maintenance}
\label{sub:maintenance}
Een voordeel van LCNC is dat het onderhoud van de applicatie zeer weinig vraagt ~\autocite{Rokis_2022}.
Volgens ~\textcite{Rokis_2022} komen we hier weer terecht bij de uitdaging van debuggen in een grafische representatie.

\subsection*{LCNC binnen bedrijven}
\label{sub:lcnc-binnen-bedrijven}
Verschillende soorten organisaties hangen af van software zodat de organisaties kunnen functioneren ~\autocite{Hintsch2021}.
Volgens ~\textcite{Rafiq_2022} is dit ook het geval bij software start-ups en midden tot grote bedrijven. 
Software start-ups zijn jonge bedrijven met een gelimmiteerd aantal middelen ~\autocite{Rafiq_2022}. 
Deze hebben dan ook enkele uitdagingen dat ze moeten weerstaan waaronder tijdsdruk, team formatie en snel groeiende markten ~\autocite{Rafiq_2022}.
Hiervoor maken ze gebruik van Low-Code en No-Code applicatie development platformen ~\autocite{Rafiq_2022}. 
Maar waarom maken ze gebruik van LCNC? en Wat is het verschil tegenover midden tot grote bedrijven?
~\textcite{Rafiq_2022} heeft deze vragen beantwoord door een onderzoek te voeren bij twee bedrijven, een software start-up en een groot bedrijf met meerde kantoren.
Hierbij kwam naar voor dat de software start-up zowel gebruik maakt van Low-Code en No-Code platformen, terwijl het grote bedrijf enkel gebruik maakt van Low-Code platformen ~\textcite{Rafiq_2022}.
De combinatie van deze platformen zorgt voor verschillende doelen die bereikt kunnen worden ~\autocite{Rafiq_2022}.
Ten eerste voor het maken van een prototype, daarnaast bij het ontwerpen, en als laatste bij het uitvoeren van de dienst ~\autocite{Rafiq_2022}.
Maar in de software start-up wordt dit niet gebruikt voor het hoofdproduct ~\autocite{Rafiq_2022}. In het grote bedrijf wordt Low-Code
gebruikt door de snelle ontwikkeling van applicatie, snelle feedback, en de mindere werklast ~\autocite{Rafiq_2022}.
\subsection*{Gebruik van LCNC door eindklanten}
\label{sub:gebruik-van-lcnc-door-eindklanten}
In de voorgaande literatuur werd aangehaald dat LCNC gebruikt zou kunnen worden door eindklanten. Dit kan ook nog eens bevestigd
worden door ~\textcite{Yan2021} waarbij men vertelt dat LCNC kan gebruikt worden door gebruikers die visuele applicaties maken, zonder enige tot weinig kennis van programmeren. 
Volgens ~\textcite{Hintsch2021} is er in 2020 een studie gedaan over hoe eindgebruikers tegenover ervaren ontwikkelaars performeren bij het gebruik van Low-Code Development.
Hier uit bleek dat ervaren ontwikkelaars moeilijkheden hadden bij het identificeren van belangrijke concepten binnen software engineering ~\autocite{Hintsch2021}. Hoewel ontwikkelaars problemen ervaarden was dit ook het geval bij eindgebruikers ~\autocite{Hintsch2021}. Deze hadden een probleem met meer envoudige taken zoals het maken van een scherm,
de connectie met de database, en "parameter passing" ~\autocite{Hintsch2021}. Dit leidt tot de vraag of eindgebruikers wel in staat zijn om applicaties te maken met Softr of Stacker.
\subsection*{LCNC Voordelen}
\label{sub:lcnc-voordelen}
\subsubsection*{Snelheid}
\label{sub:snelheid}
De mindere werklast door de snelle ontwikkeling van applicaties is een groot voordeel van LCNC ~\autocite{Adrian_2020}.
Heel wat bedrijven dat gebruikmaken van Low-Code platformen stelde vast dat hun release van de applicatie sneller was dan voorheen, bij 5 van de 10 keer ~\autocite{Yan2021}.
Volgens ~\textcite{Yan2021} is er ook een enquête van OutSystems in 2019, bleek dat gebruikers van Low-Code plaftormen 68\% van hun webapplicaties en 64\% van hun apps elk konden bouwen in vier maanden.
Dit was niet het geval bij traditionele ontwikkeling waarbij 57\% van de webapplicaties en 49\% van de apps elk werden gebouwd in vier maanden ~\autocite{Yan2021}.

\subsubsection*{Veiligheid}
\label{sub:veiligheid}
Door het gebrek aan mensen in de IT-sector, die een massa aan software moeten ontwikkelen, moeten de mensen buiten 
de IT-sector gebruikmaken van third-party software ~\autocite{Yan2021}. Dit kan schade brengen op het bedrijf omdat ze niet op de hoogte zijn 
van de licentie en veiligheid, dit noemen we ook wel "Shadow IT" ~\autocite{Yan2021}. Daarom zorgt LCNC, die geautoriseerd is door de IT-sector, enigszins tot veiligheid omdat het
het risico op "Shadow IT" vermindert ~\autocite{Yan2021}. Daarnaast zorgt LCNC er ook voor dat het IT-personeel niet telkens wordt verstoord door de mensen buiten de IT-sector ~\autocite{Yan2021}.
Met deze Low-Code en No-Code platformen kunnen de mensen makkelijk een oplossing ontwikkelen ~\autocite{Yan2021}. Voor de IT-sector is dit ook handig want Low-Code en No-Code platformen bevat 
de internationale standaarden voor veiligheid (ISO/IEC 27001, PCIDSS) ~\autocite{Sufi_2023}. Hedendaags wordt er ook volgens ~\textcite{Sufi_2023} een principe genaamd "Security by Design" toegepast.
Dit principe neemt heel wat zorgen af op het vlak van de veiligheid in de IT voor de Citizen Developers,  ook wel mensen buiten de IT-sector genoemd ~\autocite{Sufi_2023}.

\subsubsection*{Verstaanbaar}
\label{sub:Verstaanbaar}
Doordat moderne Low-Code en No-Code platformen gebruik maken van visuele representatie dat ondersteund wordt door drag-and-drop, is het makkelijk te begrijpen voor de gebruikers ~\autocite{Sufi_2023}.
Hierdoor kunnen Citizen Developers en eindklanten makkelijk zeer complexe applicaties maken zonder enige tot weinig kennis van programmeren ~\autocite{Sufi_2023}.

\subsubsection*{Cloud Forward Approach}
\label{sub:cloud-forward-approach}
Hedendaags beginnen heel wat bedrijven te migreren naar de cloud technologie ~\autocite{Sufi_2023}.
Dit komt omdat de cloud technologie heel wat voordelen biedt zoals schaalbaarheid, flexibiliteit, enzovoort ~\autocite{Sufi_2023}.
Gelukkig zijn de meeste LCNC platformen cloud gebaseerd ~\autocite{Sufi_2023}. 
Hierdoor biedt LCNC snelle strategieën voor het migreren naar de cloud, aan moderne bedrijven ~\autocite{Sufi_2023}.




% Hier beschrijf je de \emph{state-of-the-art} rondom je gekozen onderzoeksdomein, d.w.z.\ een inleidende, doorlopende tekst over het onderzoeksdomein van je bachelorproef. Je steunt daarbij heel sterk op de professionele \emph{vakliteratuur}, en niet zozeer op populariserende teksten voor een breed publiek. Wat is de huidige stand van zaken in dit domein, en wat zijn nog eventuele open vragen (die misschien de aanleiding waren tot je onderzoeksvraag!)?

% Je mag de titel van deze sectie ook aanpassen (literatuurstudie, stand van zaken, enz.). Zijn er al gelijkaardige onderzoeken gevoerd? Wat concluderen ze? Wat is het verschil met jouw onderzoek?

% Verwijs bij elke introductie van een term of bewering over het domein naar de vakliteratuur, bijvoorbeeld~\autocite{Hykes2013}! Denk zeker goed na welke werken je refereert en waarom.

% Draag zorg voor correcte literatuurverwijzingen! Een bronvermelding hoort thuis \emph{binnen} de zin waar je je op die bron baseert, dus niet er buiten! Maak meteen een verwijzing als je gebruik maakt van een bron. Doe dit dus \emph{niet} aan het einde van een lange paragraaf. Baseer nooit teveel aansluitende tekst op eenzelfde bron.

% Als je informatie over bronnen verzamelt in JabRef, zorg er dan voor dat alle nodige info aanwezig is om de bron terug te vinden (zoals uitvoerig besproken in de lessen Research Methods).

% % Voor literatuurverwijzingen zijn er twee belangrijke commando's:
% % \autocite{KEY} => (Auteur, jaartal) Gebruik dit als de naam van de auteur
% %   geen onderdeel is van de zin.
% % \textcite{KEY} => Auteur (jaartal)  Gebruik dit als de auteursnaam wel een
% %   functie heeft in de zin (bv. ``Uit onderzoek door Doll & Hill (1954) bleek
% %   ...'')

% Je mag deze sectie nog verder onderverdelen in subsecties als dit de structuur van de tekst kan verduidelijken.

%---------- Methodologie ------------------------------------------------------
\section{Methodologie}%
\label{sec:methodologie}
Er zal een vergelijkende studie gebeuren tussen twee No-Code en/of Low-Code platforms, namelijk Softr en Stacker.
Deze twee platformen maken het mogelijk om zowel als bedrijf of als een eindklant een Web \& Mobile Portal te maken.
Hierbij moet er ook rekening gehouden worden met de integratie van AirTable of Podio, een database platform. 
Vervolgens zal deze vergelijkende studie ons raadplegen over welk platform, als alternatief voor Quivvy, het meest geschikt is voor het bouwen van Web \& Mobile Portals,
dat zowel goed werkt voor het bedrijf Quivvy als eindklant. Voor de vergelijkende studie zal er opgesplitst worden in verschillende fasen namelijk,
requirements analyse, alternatieven, interessante alternatieven, proof of concept, en conclusies.
\subsection*{Requirements Analyse}
\label{sub:requirements-analyse}
Aan de hand van de co-promotor wordt de criteria opgelijst waaraan de No-Code en/of Low-Code platformen moeten voldoen.
Daarnaast zal ook de co-promotor een lijst geven op welke categorieën de No-Code en/of Low-Code platformen zullen vergeleken worden. 
Vervolgens zal er ook nog een grondige literatuurstudie gebeuren over andere LCNC platformen waarbij onderzoek al gedaan is op verschillende
criteria's. Deze fase zal ongeveer 2 weken duren. Als resultaat zal er een lijst met requirements zijn geordend op prioriteit.
De volgende criteria's zullen onderzocht worden:
\begin{itemize}
  \item Snelheid van het platform
  \item Gebruiksvriendelijkheid van het platform
  \item Integratie met AirTable of Podio
  \item Capaciteit van het platform
  \item Veiligheid
  \item functionaliteiten binnen het platform
\end{itemize}

\subsection*{Alternatieven}
\label{sub:alternatieven}
In deze fase zullen we een grondige onderzoek doen naar andere No-Code en/of Low-Code platformen die voldoen aan de requirements.
Hierbij zullen we ook rekening houden met de criteria's die opgesteld zijn door de co-promotor. Deze fase zal ongeveer 1 week duren. Tenslotte 
zal dit een lijst opleveren met verschillende alternatieven.

\subsection*{Interessante Alternatieven}
\label{sub:interessante-alternatieven}
Hier zal er gekeken worden naar de verschillende alternatieven en zal er een selectie gemaakt worden van de meest interessante.
Uit deze selectie zal er dan 1 No-Code en/of Low-Code platform gekozen worden. Om verder onderzoek over te doen
Deze fase zal ongeveer 2 weken duren.

\subsection*{Proof Of Concept}
\label{sub:proof-of-concept}
Om een goed beeld te krijgen van het No-Code en/of Low-Code platform Softr en Stacker zal er een proof of concept gemaakt worden.
Hiervoor moet men Softr en Stacker installeren. 
Deze LCNC platformen vereisen deze minimum-requirements voor de installatie:
\begin{itemize}
  \item Besturingssysteem: Windows, macOS, of Linux wordt ondersteund.
  \item Processor (CPU): Een dual-core processor of hoger is vereist.
  \item Geheugen (RAM): Minimaal 4 GB RAM of meer wordt aanbevolen.
  \item Opslagruimte: Er moet minimaal 10 GB beschikbare opslagruimte zijn.
  \item Grafische kaart: Een standaard geïntegreerde grafische kaart is meestal voldoende.
  \item Internetverbinding: Een internetverbinding is vereist voor toegang tot cloudservices of updates. 
\end{itemize}
Vervolgens moet men ook rekening houden met een database platform, namelijk AirTable of Podio.
Hiervoor moet men een geldig account hebben voor AirTable of Podio. Deze vereisten zullen vervolgens in een 
document worden opgelijst, met meer details. Dit zal ongeveer 2 dagen duren.
Na de vereisten zal de daadwerkelijke vergelijking gebeuren. Dit zal bestaan uit twee soorten vergelijkingen. Ten eerste
een vergelijking tussen Softr en Stacker waarbij we een simpel Web \& Mobile Portal maken. Hier zal dan gekeken worden naar feiten op vlak van snelheid en andere beschreven criteria's.
Ten tweede een vergelijking tussen Softr en Stacker waarbij we ook portals zullen maken dat uitgevoerd is door vijf gebruikers met weinig tot geen kennis van programmeren, ook wel eindklanten genoemd.
Naast de eindklanten zal dit ook uitgevoerd worden door vijf programmeurs. Hierbij zal er gekeken worden naar de gebruiksvriendelijkheid van het platform en andere benodigde analyse dat niet door feiten kan worden aangetoond.
De eerste vergelijking zal ongeveer 3 weken duren terwijl de tweede 4 weken zal duren, inclusief data verwerking.

\subsection*{Conclusies}
\label{sub:conclusies}
Uiteindelijk zal er met de data van de proof of concept een conclusie worden gemaakt, in de 2 laatste weken, over welk No-Code en/of Low-Code platform het meest geschikt is voor het bedrijf Quivvy en eindklant. 
Waarbij men rekening houdt met de integratie van AirTable of Podio. Dit onderzoek zal niet alles omvatten, maar zal wel een goed beeld geven over de No-Code en/of Low-Code platformen Softr en Stacker.
De aspecten die niet mogelijk zijn om te onderzoeken zijn; een uitgebreid onderzoek naar de gebruiksvriendelijkheid en de capaciteit van het platform.


% Hier beschrijf je hoe je van plan bent het onderzoek te voeren. Welke onderzoekstechniek ga je toepassen om elk van je onderzoeksvragen te beantwoorden? Gebruik je hiervoor literatuurstudie, interviews met belanghebbenden (bv.~voor requirements-analyse), experimenten, simulaties, vergelijkende studie, risico-analyse, PoC, \ldots?

% Valt je onderwerp onder één van de typische soorten bachelorproeven die besproken zijn in de lessen Research Methods (bv.\ vergelijkende studie of risico-analyse)? Zorg er dan ook voor dat we duidelijk de verschillende stappen terug vinden die we verwachten in dit soort onderzoek!

% Vermijd onderzoekstechnieken die geen objectieve, meetbare resultaten kunnen opleveren. Enquêtes, bijvoorbeeld, zijn voor een bachelorproef informatica meestal \textbf{niet geschikt}. De antwoorden zijn eerder meningen dan feiten en in de praktijk blijkt het ook bijzonder moeilijk om voldoende respondenten te vinden. Studenten die een enquête willen voeren, hebben meestal ook geen goede definitie van de populatie, waardoor ook niet kan aangetoond worden dat eventuele resultaten representatief zijn.

% Uit dit onderdeel moet duidelijk naar voor komen dat je bachelorproef ook technisch voldoen\-de diepgang zal bevatten. Het zou niet kloppen als een bachelorproef informatica ook door bv.\ een student marketing zou kunnen uitgevoerd worden.

% Je beschrijft ook al welke tools (hardware, software, diensten, \ldots) je denkt hiervoor te gebruiken of te ontwikkelen.

% Probeer ook een tijdschatting te maken. Hoe lang zal je met elke fase van je onderzoek bezig zijn en wat zijn de concrete \emph{deliverables} in elke fase?

%---------- Verwachte resultaten ----------------------------------------------
\section{Verwacht resultaat, conclusie}%
\label{sec:verwachte_resultaten}
Na grondig onderzoek van de No-Code en/of Low-Code platformen Softr en Stacker komen we uiteindelijk te weten dat ze allebei wel verschillende voordelen hebben, maar ook nadelen.
In de vergelijkende studie werd er gekeken naar verschillende criteria's zoals snelheid, gebruiksvriendelijkheid, enzovoort. Hierbij werd er specifiek een proof of concept gedaan op het vlak van gebruiksvriendelijkheid, om kritisch te bekijken of de LCNC platformen Softr en Stacker wel geschikt zijn voor eindklanten.
Hieruit bleek dat mensen met enig of weinige kennis van programmeren toch een voorkeur gaven aan Softr. Aan de andere kant gaven programmeurs dan weer een voorkeur aan Stacker. De reden hiervoor werd ook in de studie vastgelegd, waaruit bleek dat Stacker complexer was op een manier dat het voor de progammeurs makkelijker was.
Daarnaast was er ook een vergelijking tussen verschillende criteria's die alleen afhangen van feiten. De Softr was duidelijk de winnaar op het vlak van integratie met AirTable of Podio. Daarnaast was het vrij duidelijk dat Softr ook meer functionaliteiten heeft dan Stacker. Maar Stacker kwam wel als winnaar terecht bij de criteria veiligheid.
Uit het onderzoek bleek dat Softr en Stacker qua prijs weinig verschillen. We concluderen dat hoewel Stacker een betere veiligheid heeft en gemakkelijker is voor programmeurs, Softr het meest geschikt is voor het bedrijf Quivvy en eindklanten, met de gegeven criteria's.


% Hier beschrijf je welke resultaten je verwacht. Als je metingen en simulaties uitvoert, kan je hier al mock-ups maken van de grafieken samen met de verwachte conclusies. Benoem zeker al je assen en de onderdelen van de grafiek die je gaat gebruiken. Dit zorgt ervoor dat je concreet weet welk soort data je moet verzamelen en hoe je die moet meten.

% Wat heeft de doelgroep van je onderzoek aan het resultaat? Op welke manier zorgt jouw bachelorproef voor een meerwaarde?

% Hier beschrijf je wat je verwacht uit je onderzoek, met de motivatie waarom. Het is \textbf{niet} erg indien uit je onderzoek andere resultaten en conclusies vloeien dan dat je hier beschrijft: het is dan juist interessant om te onderzoeken waarom jouw hypothesen niet overeenkomen met de resultaten.

