%==============================================================================
% Sjabloon onderzoeksvoorstel bachproef
%==============================================================================
% Gebaseerd op document class `hogent-article'
% zie <https://github.com/HoGentTIN/latex-hogent-article>

% Voor een voorstel in het Engels: voeg de documentclass-optie [english] toe.
% Let op: kan enkel na toestemming van de bachelorproefcoördinator!
\documentclass{hogent-article}

% Invoegen bibliografiebestand
\addbibresource{voorstel.bib}

% Informatie over de opleiding, het vak en soort opdracht
\studyprogramme{Professionele bachelor toegepaste informatica}
\course{Bachelorproef}
\assignmenttype{Onderzoeksvoorstel}
% Voor een voorstel in het Engels, haal de volgende 3 regels uit commentaar
% \studyprogramme{Bachelor of applied information technology}
% \course{Bachelor thesis}
% \assignmenttype{Research proposal}

\academicyear{2023-2024} % TODO: pas het academiejaar aan

% TODO: Werktitel
\title{Portals voor Softwarebedrijf en Eindklant: Een Vergelijkende Analyse van Softr en Stacker in Web \& Mobile Toepassingen}

% TODO: Studentnaam en emailadres invullen
\author{Joeri Verhelst}
\email{joeri.verhelst@student.hogent.be}

% TODO: Medestudent
% Gaat het om een bachelorproef in samenwerking met een student in een andere
% opleiding? Geef dan de naam en emailadres hier
% \author{Yasmine Alaoui (naam opleiding)}
% \email{yasmine.alaoui@student.hogent.be}

% TODO: Geef de co-promotor op
\supervisor[Co-promotor]{M. Demunter (Quivvy, \href{mailto:mike.demunter@quivvy.com}{mike.demunter@quivvy.com})}
% Binnen welke specialisatierichting uit 3TI situeert dit onderzoek zich?
% Kies uit deze lijst:
%
% - Mobile \& Enterprise development
% - AI \& Data Engineering
% - Functional \& Business Analysis
% - System \& Network Administrator
% - Mainframe Expert
% - Als het onderzoek niet past binnen een van deze domeinen specifieer je deze
%   zelf
%
\specialisation{Mobile \& Enterprise development}
\keywords{Software, Mobile \& Web Portals, No-Code, Low-Code}

\begin{document}

\begin{abstract}
  Door de constante verandering in de bussines wereld moeten bedrijven steeds sneller kunnen reageren op de veranderingen en nieuwe technologie snufjes. Eén van deze technologie snufjes, dat de markt overspoeld, is
  Low-Code en No-Code platformen. Maar door de grote hoeveelheid aan platformen is het moeilijk om een keuze te maken. Dit is niets anders dan ook bij het softwarebedrijf Quivvy waarbij men afvraagt welke Low-Code en No-Code plaftorm
  het best geschikt is voor zowel het bedrijf als de einklant, maar ook interageert met AirTable of Podio. In dit onderzoek zullen we twee platformen vergelijken namelijk, Softr en Stacker. Deze platformen zijn gekozen omdat ze de betere LCNC platformen zijn
  op de markt, dat goed kan interageren met AirTable of Podio. Hierbij is er een grondige vergelijkende analyse gedaan tussen de twee plaftormen, maar ook een analyse naar alternatieven. Uit de analyse zijn vereisten opgesteld voor de proof of concept.
  Hierbij zal er 2 POC bestaan, één voor de gebruiksvriendelijkheid en één voor de vereisten dat alleen feiten weergeven. Uit de methodologie bleek dat er nog een goede alternatief was voor het softwarebedrijf Quivvy. In de proof of concept voor de gebruiksvriendelijkheid
  merkte dat Softr toch een betere keuze is dan Stacker. Bij de proof of concept voor de vereisten dat alleen feiten weergeven, had Softr een betere score dan Stacker. Softr is sneller, interageert beter met AirTable en Podio en heeft meer functionaliteiten. Stacker had wel 
  een betere veiligheid. Uit de resultaten kan dan een conclusie getrokken worden op de vraag; welke platform is het beste geschikt voor zowel het bedrijf als de eindklant, dat ook het best interageert met AirTable of Podio. Hierbij kunnen we concluderen dat Softr wel degelijk de betere optie 
  is tussen de twee. In dit onderzoek was het niet mogelijk om een uitgebreide studie te doen over de gebruiksvriendelijkheid van de twee platformen en de capaciteit van de platformen. Ook zijn de alternatieven niet uitgebreid onderzocht. Dit kan een vervolg zijn op dit onderzoek.
\end{abstract}

\tableofcontents

% De hoofdtekst van het voorstel zit in een apart bestand, zodat het makkelijk
% kan opgenomen worden in de bijlagen van de bachelorproef zelf.
%---------- Inleiding ---------------------------------------------------------

\section{Introductie}%
\label{sec:introductie}

Bedrijven aarsen op effeciëntie, kosten  verminderen, en veiligheid. Dit is niets anders dan ook bij het softwarebedrijf genaamd Quivvy waarbij de trend No-Code en Low-Code
platforms voor het bouwen van applicaties de laatste jaren rond dwaalt. Momenteel gebruikt Quivvy FlutterFlow voor het bouwen van Mobile \& Web Portals. 
Hierbij interageert FlutterFlow met AirTable, een database platform. Maar bedrijven zijn voortdurend op zoek naar de beste softwareplatformen om hun bedrijf te runnen.
Daarom zoekt Quivvy naar een alternatief voor het bouwen van Mobile \& Web Portals dat zowel goed werkt voor bedrijf als eindklant, maar ook interageert met AirtTable of Podio.
De markt van No-Code en Low-Code platforms worden overspoeld met verschillende platformen waaronder Microsoft PowerApps, Bubble, Webflow, Softr en Stacker.
Als gevolg dat drie op de vier van de grootste bedrijven tegen 2024 gebruik zullen maken van Low-Code platforms.~\autocite{Moskal_2021} 
Deze No-Code en Low-Code platforms hebben ook een reden tot de populariteit, namelijk dat
deze platforms het mogelijk maken om applicaties te bouwen zonder enige tot weinig kennis van programmeren. Dit is niet het enige waarom deze platforms een trend zijn,
want het zorgt ook voor een snelle ontwikkeling met lage kosten door de effeciënte gebruik van de ontwikkelaars.
Maar welke bestaande softwareplatform is geschikt voor Mobile \& Web portals te creëren, dat zowel goed werkt voor bedrijf als eindklant?
In deze bachelorproef hanteren we het meest geschikte Mobile \& Web Portals voor een softwarebedrijf en eindklant.
Waarbij we een vergelijking maken tussen Softr en Stacker, beide No-Code en/of Low-Code platforms. Daarbij wordt er rekening gehouden met verschillende 
factoren zoals snelheid, gebruiksvriendelijkheid, enzovoort.


%Waarover zal je bachelorproef gaan? Introduceer het thema en zorg dat volgende zaken zeker duidelijk aanwezig zijn:

% \begin{itemize}
%   \item kaderen thema
%   \item de doelgroep
%   \item de probleemstelling en (centrale) onderzoeksvraag
%   \item de onderzoeksdoelstelling
% \end{itemize}

% Denk er aan: een typische bachelorproef is \textit{toegepast onderzoek}, wat betekent dat je start vanuit een concrete probleemsituatie in bedrijfscontext, een \textbf{casus}. Het is belangrijk om je onderwerp goed af te bakenen: je gaat voor die \textit{ene specifieke probleemsituatie} op zoek naar een goede oplossing, op basis van de huidige kennis in het vakgebied.

% De doelgroep moet ook concreet en duidelijk zijn, dus geen algemene of vaag gedefinieerde groepen zoals \emph{bedrijven}, \emph{developers}, \emph{Vlamingen}, enz. Je richt je in elk geval op it-professionals, een bachelorproef is geen populariserende tekst. Eén specifiek bedrijf (die te maken hebben met een concrete probleemsituatie) is dus beter dan \emph{bedrijven} in het algemeen.

% Formuleer duidelijk de onderzoeksvraag! De begeleiders lezen nog steeds te veel voorstellen waarin we geen onderzoeksvraag terugvinden.

% Schrijf ook iets over de doelstelling. Wat zie je als het concrete eindresultaat van je onderzoek, naast de uitgeschreven scriptie? Is het een proof-of-concept, een rapport met aanbevelingen, \ldots Met welk eindresultaat kan je je bachelorproef als een succes beschouwen?

%---------- Stand van zaken ---------------------------------------------------

\section{State-of-the-art}%
\label{sec:state-of-the-art}

 In de software wereld bevindt men in projecten dat er telkens over het budget wordt gegaan. Daarnaast doet het project meestal niet wat de klant verwacht 
 en vervolgens  doet het opgeleverde product niet wat het moet doen ~\autocite{Moskal_2021}. Volgens ~\textcite{Moskal_2021} zijn deze problemen zijn niet alleen maar te zien in de software wereld, 
 maar ook in andere categorieën binnen de IT-sector of bij het implementer en ontwerpen van software systemen. 
De reden waarom No-Code stijgt in populariteit kan men onderverdelen in vier categorieën:
\begin{itemize}
  \item \textbf{Beperkt aantal programmeurs}: 
  Heel wat studenten laten zich afschrikken door het programmeren. 
  Hierdoor is er kort aantal studenten die werkelijk diepe kennis hebben op vlak van programmeren. 
  Maar het vinden een zeer gekwalificeerde programmeur brengt projecten niet altijd tot een goed einde ~\autocite{Moskal_2021}.
  \item \textbf{Technologische turbulentie}: De constante evolutie van programmeertalen zorgt ervoor dat de kennis van programmeurs niet altijd de meest recente is  ~\autocite{Moskal_2021}.
  \item \textbf{Hoge kosten}: De traditionele softwareontwikkeling heeft een grote tol op de kosten van een bedrijf. Daarbij beseffen Software engineers dat het bouwen van een applicatie niet gemakklijk is binnen een budget ~\autocite{Moskal_2021}.
  \item \textbf{Tijdrovend}: De traditionele softwareontwikkeling is een tijdrovend proces. De geschatte tijd om Software te maken op één operating systeem is zes maanden of zelfs lange ~\autocite{Moskal_2021}.
  \item \textbf{Complexe softwareontwikkeling}
\end{itemize} 




% Hier beschrijf je de \emph{state-of-the-art} rondom je gekozen onderzoeksdomein, d.w.z.\ een inleidende, doorlopende tekst over het onderzoeksdomein van je bachelorproef. Je steunt daarbij heel sterk op de professionele \emph{vakliteratuur}, en niet zozeer op populariserende teksten voor een breed publiek. Wat is de huidige stand van zaken in dit domein, en wat zijn nog eventuele open vragen (die misschien de aanleiding waren tot je onderzoeksvraag!)?

% Je mag de titel van deze sectie ook aanpassen (literatuurstudie, stand van zaken, enz.). Zijn er al gelijkaardige onderzoeken gevoerd? Wat concluderen ze? Wat is het verschil met jouw onderzoek?

% Verwijs bij elke introductie van een term of bewering over het domein naar de vakliteratuur, bijvoorbeeld~\autocite{Hykes2013}! Denk zeker goed na welke werken je refereert en waarom.

% Draag zorg voor correcte literatuurverwijzingen! Een bronvermelding hoort thuis \emph{binnen} de zin waar je je op die bron baseert, dus niet er buiten! Maak meteen een verwijzing als je gebruik maakt van een bron. Doe dit dus \emph{niet} aan het einde van een lange paragraaf. Baseer nooit teveel aansluitende tekst op eenzelfde bron.

% Als je informatie over bronnen verzamelt in JabRef, zorg er dan voor dat alle nodige info aanwezig is om de bron terug te vinden (zoals uitvoerig besproken in de lessen Research Methods).

% % Voor literatuurverwijzingen zijn er twee belangrijke commando's:
% % \autocite{KEY} => (Auteur, jaartal) Gebruik dit als de naam van de auteur
% %   geen onderdeel is van de zin.
% % \textcite{KEY} => Auteur (jaartal)  Gebruik dit als de auteursnaam wel een
% %   functie heeft in de zin (bv. ``Uit onderzoek door Doll & Hill (1954) bleek
% %   ...'')

% Je mag deze sectie nog verder onderverdelen in subsecties als dit de structuur van de tekst kan verduidelijken.

%---------- Methodologie ------------------------------------------------------
\section{Methodologie}%
\label{sec:methodologie}

Hier beschrijf je hoe je van plan bent het onderzoek te voeren. Welke onderzoekstechniek ga je toepassen om elk van je onderzoeksvragen te beantwoorden? Gebruik je hiervoor literatuurstudie, interviews met belanghebbenden (bv.~voor requirements-analyse), experimenten, simulaties, vergelijkende studie, risico-analyse, PoC, \ldots?

Valt je onderwerp onder één van de typische soorten bachelorproeven die besproken zijn in de lessen Research Methods (bv.\ vergelijkende studie of risico-analyse)? Zorg er dan ook voor dat we duidelijk de verschillende stappen terug vinden die we verwachten in dit soort onderzoek!

Vermijd onderzoekstechnieken die geen objectieve, meetbare resultaten kunnen opleveren. Enquêtes, bijvoorbeeld, zijn voor een bachelorproef informatica meestal \textbf{niet geschikt}. De antwoorden zijn eerder meningen dan feiten en in de praktijk blijkt het ook bijzonder moeilijk om voldoende respondenten te vinden. Studenten die een enquête willen voeren, hebben meestal ook geen goede definitie van de populatie, waardoor ook niet kan aangetoond worden dat eventuele resultaten representatief zijn.

Uit dit onderdeel moet duidelijk naar voor komen dat je bachelorproef ook technisch voldoen\-de diepgang zal bevatten. Het zou niet kloppen als een bachelorproef informatica ook door bv.\ een student marketing zou kunnen uitgevoerd worden.

Je beschrijft ook al welke tools (hardware, software, diensten, \ldots) je denkt hiervoor te gebruiken of te ontwikkelen.

Probeer ook een tijdschatting te maken. Hoe lang zal je met elke fase van je onderzoek bezig zijn en wat zijn de concrete \emph{deliverables} in elke fase?

%---------- Verwachte resultaten ----------------------------------------------
\section{Verwacht resultaat, conclusie}%
\label{sec:verwachte_resultaten}

Hier beschrijf je welke resultaten je verwacht. Als je metingen en simulaties uitvoert, kan je hier al mock-ups maken van de grafieken samen met de verwachte conclusies. Benoem zeker al je assen en de onderdelen van de grafiek die je gaat gebruiken. Dit zorgt ervoor dat je concreet weet welk soort data je moet verzamelen en hoe je die moet meten.

Wat heeft de doelgroep van je onderzoek aan het resultaat? Op welke manier zorgt jouw bachelorproef voor een meerwaarde?

Hier beschrijf je wat je verwacht uit je onderzoek, met de motivatie waarom. Het is \textbf{niet} erg indien uit je onderzoek andere resultaten en conclusies vloeien dan dat je hier beschrijft: het is dan juist interessant om te onderzoeken waarom jouw hypothesen niet overeenkomen met de resultaten.



\printbibliography[heading=bibintoc]

\end{document}